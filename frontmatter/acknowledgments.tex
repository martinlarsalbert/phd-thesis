This thesis presents research performed since February 2020 at the Division of Marine Technology, Department of Mechanics and Marine Sciences at Chalmers University of Technology and SSPA Sweden AB (\href{www.sspa.se}{www.sspa.se}). Financial support for this research was provided by the DEMOPS project (Development of Methods for Operational Performance of Ships) funded by Swedish Transport Administration (project: FP4 2020) and D2E2F project (Data Driven Energy Efficiency of Ships) funded by Swedish Energy Agency (project: 49301-1).

I had unsuccessfully been applying for funding for PhD studies for a couple of years when Professor Wengang Mao contacting me three years ago with the offer to become a PhD student. I will always be very grateful for this offer and I'm sure that I would otherwise be still searching for funding or given up by now. Wengang has also been my main supervisor during my studies and a guide to the academic research and a tutor in statistical and machine learning methods.  

This gratitude also goes to my examiner and co-supervisor, Professor Jonas W, Ringsberg,
head of the Division of Marine Technology. I have enjoyed our discussions about research methodology and how to organize a paper in academic writing, where his detailed proof reading has also been a great asset.

I also want to thank SSPA Sweden AB for allowing me to be an industrial PhD student within my current employment. Special thanks to Dr. Christian Finnsgård head of the Research Department at SSPA, for his support and good advice throughout the project. I also want to mention all personnel at SSPA who have been involved in creating the model test results, building the ship models, and conducting the experiments.
I'm also grateful to Daiyong who did a great work when preparing Paper \ref{pap:daiyong}.

\vskip 2pc

\noindent \thesisauthor

\noindent \thesiscity, September\  2021  % Since dedication is written a month or more before the actual thesis date, \thesismonth and \thesisyear is not used here.
