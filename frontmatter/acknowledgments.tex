This thesis presents research conducted since February 2020 in the Division of Marine Technology, Department of Mechanics and Marine Sciences at Chalmers University of Technology, and RISE (\href{www.ri.se}{www.ri.se}). Financial support for this research was provided by the DEMOPS project (Development of Methods for Operational Performance of Ships), funded by the Swedish Transport Administration (project: FP4 2020), and the D2E2F project (Data Driven Energy Efficiency of Ships), funded by the Swedish Energy Agency (project: 49301-1).

After several years of unsuccessful attempts to secure funding for my PhD studies, I was contacted by Professor Wengang Mao five years ago with an offer to become a PhD student. I am deeply grateful for this opportunity. Professor Mao has been my main supervisor throughout my studies, guiding me in academic research and teaching me about statistical and machine learning methods. My gratitude also extends to my examiner and co-supervisor, Professor Jonas W. Ringsberg, head of the Division of Marine Technology. I have greatly valued our discussions on research methodology and academic writing, and his meticulous proofreading has been a tremendous asset to our papers.

I would like to thank RISE and the head of the Research Department, Christian Finnsgård, for allowing me to pursue an industrial PhD. I am also grateful to my colleagues at RISE, including Martin Kjellberg, Olov Lundbäck, and others, for the insightful discussions we have had about ship dynamics, which have been invaluable to the literature study of this thesis. Finally, I want to acknowledge all the personnel at RISE who have been involved in model tests, building ship models, and conducting experiments.

\vskip 2pc

\noindent \thesisauthor

\noindent \thesiscity, February\ 2025 % Since dedication is written a month or more before the actual thesis date, \thesismonth and \thesisyear is not used here.
