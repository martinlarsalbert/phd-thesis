
% -- Importance, background/motivation
It is common today, that operational data is recorded onboard ships within the Internet of Ships paradigm. This enables the possibility to build ship digital twins as digital copies of the real ships. Predicting the ships motions with ship dynamics could be an important sub-component of these ship digital twins. Modelling the ship dynamics in full scale sea conditions is a very difficult task, with a lot of uncertainties. These uncertainties can be significantly reduced in a controlled laboratory environment, also assuming calm water conditions. A prediction model should generalize outside the known data to be of any practical use. Grey-box modelling is a technique that combines the operational data with physical principles.  

The objective of this thesis is to 
% -- Objective/scope
\noindent \objective 

% -- Method, developed method, setting
A model development procedure is proposed in this thesis to handle the model uncertainty by a selection of candidate models based on holdout evaluation. The measurement noise is handled by a iterative preprocessor which uses an extended Kalman filter (EKF) and Rauch Tung Striebel (RTS) smoother that uses an initial guessed predictor model from semi-empirical formulas.

% -- Results example
It is shown that the ship roll motion with good accuracy can be described by a quadratic damping model. For the more complex manoeuvring models, multicollinearity is a large problem where the appropriate complexity needs to be selected with the bias-variance tradeoff between underfitting or overfitting the data. 
The proposed model development procedure and parameter estimation method was applied on the wPCC and KVLCC2 test case ships where the holdout turning circle tests were predicted with good accuracy.

% -- Conclusion 
The proposed methods can produce prediction models with very good generalization, given a good model structure selected from the candidate models and an appropriate split in the holdout evaluation of the model development process. 

\vspace{0.1cm}
\textbf{Keywords:} Ship Digital Twin, Ship Manoeuvring, System identification, Inverse Dynamics, Extended Kalman Filter, RTS smoother, Multicollinearity
