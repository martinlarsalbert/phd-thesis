% -- Importance, background/motivation
% -- Objective/scope
This thesis investigates the enhancement of ship manoeuvring models through the integration of prior knowledge embedded in parametric model structures and semi-empirical formulas. The research is driven by the question: How can prior knowledge be used to enhance the generalization of ship manoeuvring models?

% -- Method, developed method, setting
The study begins with a prestudy focusing on one degree of freedom in ship roll motion, aiming to develop parameter identification techniques and propose a parametric model structure with good generalization. This knowledge is then extended to the manoeuvring problem, with objectives including the development of parameter identification techniques for ship manoeuvring models, proposing a generalizable parametric model structure, mitigating multicollinearity, and identifying added masses.

Methodologically, the research employs various parametric model structures for roll motion and manoeuvring, investigated through free running model tests and virtual captive tests (VCT). A novel parameter identification method combining inverse dynamics with an extended Kalman filter (EKF) is proposed. Additionally, a deterministic semi-empirical rudder model is introduced to address multicollinearity issues.

% -- Results example
% -- Conclusion 
Key findings indicate that inverse dynamics regression is an efficient method for parameter identification in parametric models. The proposed quadratic model structure for roll motion demonstrates good generalization, and the new parameter identification method accurately predicts manoeuvring models from standard maneuvers. However, challenges with multicollinearity and the need for more informative data are highlighted. The study concludes that semi-empirical formulas can guide identification towards more physically correct models, and VCT can provide the necessary data for accurate model identification.

% Implications
The implications of this research suggest that integrating semi-empirical rudder models and utilizing VCT can significantly enhance the accuracy and generalization of ship manoeuvring models, contributing to more reliable and physically accurate simulations in maritime engineering.


\vspace{0.3cm}
\noindent\textbf{Keywords:} Manoeuvring, Roll damping, System identification, Extended Kalman filter, Inverse dynamics, Multicollinearity
\cleardoublepage
