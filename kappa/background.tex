\section{Background}
A controllable ship and the knowledge of how to control it are essential for a safe voyage. Ships greater than 100 meters must meet formal manoeuvring requirements \cite{imoStandardsShipManoeuvrability2002}, which are ultimately demonstrated during sea trials. Earlier assessments are often performed before the ship is built, by constructing a model of the ship. Building a physical scale model to conduct free-running model tests (FRMT) is still recognized as the most reliable method \cite{ittcManeuveringCommitteeITTC2008}. Besides the physical scale model, there are also more abstract types of models such as mental models, statistical models, machine learning models, mathematical models, etc.

\say{Loosely speaking, a model is a tool we use to answer questions about the system without having to do an experiment} \cite{ljungModelingIdentificationDynamic2021}. There are many situations where an experiment with the ship is not a desirable option; for instance, it could be too expensive, too slow, too dangerous, or the ship does not exist yet. Computational fluid dynamics (CFD) has been developed to describe the hydrodynamics of ships based on the fundamental principles of physics. However, there are many situations where this is not a feasible model. The calculations could be too expensive, or perhaps the geometries, calculation domain, or boundary conditions cannot be defined with sufficient accuracy. Therefore, in many situations, the lack of a complete physical understanding of the system must be accepted; instead, a data-driven model is used, which mimics the system behavior from observations. These data-driven models will be explored for manoeuvring in this thesis.

Model structures for manoeuvring are often categorized in the literature as either parametric models or non-parametric models. A third category, hybrid models, combines parametric models with non-parametric models. The following definitions have therefore been adopted in this thesis: if the model structure is defined by explicit mathematical formulas that have parameters in it, it is categorized as a parametric model; all other model structures are categorized as either non-parametric or hybrid models.

There are mainly two types of data to identify the models: the captive test (CT) or the free-running test (FT), both of which will be covered in this thesis. Captive model tests (CMT) are the classical way of conducting captive tests, which can be performed in various ways: with an XY-carriage, rotating arm, or planar motion mechanism (PMM). CT can also be performed with CFD in virtual captive tests (VCT). FT data are collected from either model tests, full-scale tests, or in some cases, direct CFD \cite{arakiEstimatingManeuveringCoefficients2012}. CT data is generally more applicable in virtual prototyping when assessing the manoeuvring performance before ships are built. FT data, on the other hand, are generally more applicable for existing ships, in a digital twin context.

The success of system identification methods depends on both the chosen model structure and the quality of the data in terms of measurement accuracy and amount of information. Many papers in the literature handle system identification of parametric models from simulated data. However, this has been considered an irrelevant scenario in this thesis, since the model structure that generated the data is already known beforehand; \say{we identify real objects, not their mathematical model} \cite{millerShipModelIdentification2021}. However, it is still relevant to use simulated data to study the non-parametric models' abilities to find a good model structure. However, while parametric models will be the primary focus of this thesis, other approaches will also be discussed in the subsequent literature review.

