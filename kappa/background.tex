\section{Background}
A controllable ship and the knowledge how to control it are essential for a safe voyage.
There are formal manoeuvring requirements that ships greater than 100 meters must meet \cite{imoStandardsShipManoeuvrability2002}, which are ultimately demonstrated during sea trials. Earlier assessments are often performed before the ship has been built, by building a model of the ship. Building a physical scale model to conduct free-running model tests (FRMT) is still today recognized as the most reliable way \cite{ittcManeuveringCommitteeITTC2008}. Besides the physical scale model, there are also more abstract types of models such as mental models, statistical models, machine learning models, mathematical models, etc.
\say{Loosely speaking a model is a tool we use to answer questions about the system without having to do an experiment} \cite{ljungModelingIdentificationDynamic2021}.
There are many situations where an experiment with the ship is not a desirable option; for instance, it could be too expensive, too slow, too dangerous, or the ship does not exist yet.  
Computational fluid dynamics (CFD) has been developed to describe the hydrodynamics of ships based on the fundamental principles of physics. However, there are many situations where this is not a feasible model. The calculations could be too expensive or perhaps the geometries, calculation domain, or boundary conditions cannot be defined with sufficient accuracy. There are therefore many situations where the lack of a complete physical understanding of the system must be accepted; instead, a data-driven model is used, which mimics the system behavior from observations. These data-driven models will be explored for manoeuvring in this thesis.  

Model structures for manoeuvring are often categorized in the literature as being either parametric models or non-parametric models. A third category, hybrid models, which combines the parametric models with the non-parametric models, is also often used. 
The following definitions have therefore been adopted in this thesis;
if the model structure is defined by explicit mathematical formulas that have parameters in it, it is categorized as a parametric model; all other model structures are categorized as either non-parametric or hybrid models. 

There are mainly two types of data to identify the models: the captive test (CT) or the free-running test (FT) which will both be covered in this thesis. Captive model tests (CMT) are the classical way of captive tests which can be conducted in various ways: with a XY-carriage, rotating arm, or planar motion mechanism (PMM). CT can also be performed with CFD in virtual captive tests (VCT). 
FT data are collected from either: model tests, full-scale tests, or in some cases direct CFD \cite{arakiEstimatingManeuveringCoefficients2012}.
CT data is generally more applicable in virtual prototyping, when assessing the manoeuvring performance before ships are built. The FT data, on the other hand, are generally more applicable for existing ships, in a digital twin context. 

The success of system identification methods depends on both the chosen model structure and the quality of the data in terms of measurement accuracy and amount of information. 
There are many papers in the literature that handle system identification of parametric models from simulated data. However, this has been considered an irrelevant scenario in this thesis, since the model structure that generated the data is already known before hand; \say{we identify real objects, not its mathematical model} \cite{millerShipModelIdentification2021}. However, it is still relevant to use simulated data to study the non-parametric models' abilities to find a good model structure.

