\section{Background}
The ability to understand and ensure the controllability of vessels is essential for achieving safe marine operations. Ships exceeding 100 meters must meet formal manoeuvring requirements \cite{imoStandardsShipManoeuvrability2002}, which are ultimately verified during sea trials. However, preliminary assessments are often performed before ship construction through ship models. Building a physical-scale model to conduct free-running model tests (FRMT) is still recognized as the most reliable method \cite{ittcManeuveringCommitteeITTC2008} for benchmarking performance. However, these physical-scale models are often complemented by more abstract approaches, such as numerical models in computational fluid dynamics (CFD) or data-driven models.

\say{Loosely speaking, a model is a tool we use to answer questions about the system without having to do an experiment} \cite{ljungModelingIdentificationDynamic2021}. There are many situations where an experiment using a full-scale ship is undesirable; for instance, such experiments may be cost prohibitive, time consuming, inherently dangerous, or simply impossible if the ship has not been built. CFD has been developed to describe the hydrodynamics of ships based on the fundamental principles of physics. However, there are many situations where this is infeasible: calculations may be prohibitively expensive, or perhaps the geometries, calculation domain, or boundary conditions may not be definable with sufficient accuracy. Therefore, in many cases, the lack of a complete physical understanding of the system must be accepted, and instead, a data-driven model is used, which mimics the system behavior based on observations. This thesis explores the use of such data-driven models for manoeuvring.

Model structures for manoeuvring are often categorized in the literature as either parametric models or non-parametric models. A third category, hybrid models, combines parametric and non-parametric models. 
Parametric models are characterized by a fixed number of parameters, in contrast to non-parametric models that have a flexible number of parameters, which can grow with the size of the data. 

There are primarily two approaches used to obtain the necessary data for use in model creation: the captive test (CT) and the free-running test (FT). Both methodologies will be discussed in this thesis.  Captive model tests (CMT) are the conventional method for obtaining CT data and can be conducted using various means, such as with an XY-carriage, a rotating arm, or a planar motion mechanism (PMM). Virtual captive tests (VCT) extend this approach by incorporating CFD simulations. FT data are collected from either model tests, full-scale ship trials, or in some cases, direct CFD \cite{arakiEstimatingManeuveringCoefficients2012}. CT data are generally more applicable in virtual prototyping when assessing the manoeuvring performance before ships are built. In contrast, FT data are typically more applicable to existing ships, in a digital twin context.

The success of system identification methods depends on both the chosen model structure and the quality of the data in terms of measurement accuracy and amount of information. Many studies in the literature discuss system identification of parametric models from simulated data. However, this has been considered an irrelevant scenario in this thesis, since the model structure that generated the data is known beforehand; \say{we identify real objects, not their mathematical model} \cite{millerShipModelIdentification2021}. Nevertheless, the use of simulated data in gauging the ability of a non-parametric model to identify a suitable model structure remains relevant. Although parametric models will be the primary focus of this thesis, other approaches will also be discussed in the subsequent literature review.

