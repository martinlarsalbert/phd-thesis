\section{Summary of Paper \ref{pap:pit}}
\subsection*{"\nameref{pap:pit}"}
In order to expand the modelling complexity and uncertainty from Paper \ref{pap:rolldamping}, system identification of the surge, sway and yaw
degrees of freedoms are studied in Paper \ref{pap:pit}. The dynamics is assumed to be described by a ship manoeuvring model (\autoref{sec:manoeuvring model}). The system identification method proposed in Paper \ref{pap:pit} was validated on two case study ships: the wPCC (\autoref{fig:wpcc-mdl}) and the well-known KVLCC2 (\autoref{fig:kvlcc2_hsva}). The models are developed following the process as described in \autoref{sec:model_development_process}. Consequently, both test cases aim to predict turning circle maneuvers. The main dimensions of the two case study ship models are listed in \autoref{tab:cases}, with explanations in \autoref{tab:nomenclature}. The wPCC is a wind-powered car carrier tested at SSPA \cite{alexandersson_wpcc_2022}. This twin screw ship with large rudders has good course stability and symmetric hydrodynamic manoeuvring forces. The KVLCC2 model test data from the Hamburg ship model basin (HSVA) and Maritime research institute Netherlands (MARIN) was made available by SIMMAN2008 conference \cite{stern_experience_2011}. This single screw ship is more course unstable than the wPCC test case, and manoeuvring forces are unsymmetrical due to the single propeller. This instability makes it good as the second test case with parameter estimation on an unsymmetrical model.

\begin{figure}[!htb]
\centering
\includegraphics[width=\linewidth]{kappa/images/wpcc_mdl.png}
\caption{wPCC tested at SSPA. Copyright 2020 by SSPA Sweden AB.}
\label{fig:wpcc-mdl}
\end{figure}

\begin{figure}[!htb]
    \centering
    \begin{subfigure}[b]{0.45\textwidth}
    \centering
    \includegraphics[height=3cm]{kappa/images/kvlcc2_front.png}
    \end{subfigure}
    ~
     \begin{subfigure}[b]{0.45\textwidth}
     \centering
     \includegraphics[height=3cm]{kappa/images/kvlcc2_aft.png}
     \end{subfigure}
    \caption{Ship model used in HSVA and MARIN model tests. Copyright HSVA.}
    \label{fig:kvlcc2_hsva}
\end{figure}


\begin{table}[h]
    \footnotesize
    \caption{Main dimensions of test case ship models.}
    \label{tab:cases}

    \centering
    \begin{tabular}{|p{1.5cm}|p{0.5cm}|p{0.5cm}|p{0.5cm}|p{0.5cm}|p{0.5cm}|p{0.5cm}|p{0.5cm}| p{0.5cm}|p{0.5cm}|p{0.5cm}|p{0.5cm}|p{0.9cm}|p{0.9cm}|}
\hline
&

\(B\) \([m]\)
&

\(D\) \([m]\)
&

\(L\) \([m]\)
&

\(L_{CG}\) \([m]\)
&

\(N_p\)
&

\(c\) \([m]\)
&

\(\alpha\)
&

\(\nabla\) \([m^3]\)
&

\(k_{zz}\)
&

\(m\) \([kg]\)
&

\(w_{p0}\)
&

\(x_{p}\) \([m]\)
&

\(x_{r}\) \([m]\)
\\
\hline
WPCC
&

0.95
&

0.12
&

5.01
&

0.0
&

2
&

0.21
&

41.2
&

0.44
&

0.25
&

441
&

0.15
&

-2.42
&

-2.42
\\

KVLCC2 (HSVA)
&

1.27
&

0.2
&

7.0
&

0.24
&

1
&

0.46
&

45.7
&

3.27
&

0.25
&

3272
&

0.4
&

-3.39
&

-3.5
\\
\hline
\end{tabular}
\end{table}

\begin{table}[h]
\footnotesize
\caption{List of main dimensions symbols.}
    \label{tab:nomenclature}

    \centering
    \begin{tabular}{c c}
\toprule


symbol
&

description
\\


\(B\)
&

Beam
\\


\(D\)
&

Propeller diameter
\\


\(L\)
&

Length between perpendiculars
\\


\(L_{CG}\)
&

Distance \(L/2\) to centre of gravity
\\


\(N_p\)
&

Number of propellers
\\


\(T\)
&

Draught
\\


\(\alpha\)
&

Scale factor
\\


\(\nabla\)
&

Volume displacement
\\


\(k_{zz}\)
&

Radius of gyration / \(L\)
\\


\(m\)
&

Mass (excluding added mass)
\\


\(w_{p0}\)
&

Wake fraction
\\


\(x_{p}\)
&

Longitudinal position of propeller
\\


\(x_{r}\)
&

Longitudinal position of rudder
\\
\bottomrule
\end{tabular}
\end{table}


\noindent The parameter estimation method requires an initial guessed linear manoeuvring model. For these initial models for the two test cases, their hydrodynamic derivatives (\autoref{tab:initial}) are calculated with semi-empirical formulas (\autoref{app:initial_estimates}) taken from \cite{brix_manoeuvring_1993}. 
\begin{table}[h]
    \footnotesize
    \caption{Initial guessed derivatives in linear models (times 1000)}
    \label{tab:initial}
    %\centering
    \begin{tabular}{|T|T|T|T|T|T|T|T|T|T|T|T|}
\hline
 & 

\( N_{\delta} \)
& 

\( N_{r} \)
& 

\( N_{\dot{r}}' \)
& 

\( N_{v} \)
& 

\( N_{\dot{v}}' \)
& 

\( X_{\dot{u}}' \)
& 

\( Y_{\delta} \)
& 

\( Y_{r} \)
& 

\( Y_{\dot{r}}' \)
& 

\( Y_{v} \)
& 

\( Y_{\dot{v}}' \)
\\
\hline

WPCC
&

-1.5
&

-1.719
&

-0.299
&

-3.184
&

-0.128
&

0.179
&

3.0
&

2.402
&

-0.303
&

-9.713
&

-6.109
\\
\hline

KVLCC2 (HSVA)
&

-1.5
&

-3.415
&

-0.822
&

-8.707
&

-1.166
&

1.05
&

3.0
&

4.305
&

-1.271
&

-25.266
&

-15.846
\\
\hline
\end{tabular}

\end{table}




\subsection{The wPCC test case}
\label{\detokenize{05.01_case_studies:the-wpcc-test-scenarios}}

The wPCC test case focuses on predicting forces and moments from the ship hull and the rudders. The propeller force is not part of the prediction model but is taken from the model test measurements.
In the model development process (\autoref{sec:model_development_process}), the model test data used for modeling is split into training, validation and test data sets, 
\begin{itemize}
    \item The training dataset: self-propulsion, pull-out tests, and zigzag10/10 tests to starboard and port.
    \item The validation dataset: three zigzag20/20 tests.
    \item Test dataset: one turning circle test.
\end{itemize}
\noindent This split is also shown in \autoref{fig:wpcc_datasets}. If the manoeuvring model built by the proposed method based on a series of model tests including ZigZag10/10, 20/20 to port and starboard as well as self-propulsion and pull out test \cite{imo_standards_2002} can predict the turning circle maneuver, then it is a capable model.

\begin{figure}[!htb]
\centering
\includegraphics[width=\linewidth]{kappa/images/3.pdf}
\caption{wPCC training, validation and testing datasets.}
\label{fig:wpcc_datasets}
\end{figure}

The LVMM was ruled too simple, so only the AVMM and MAVMM were considered possible manoeuvring models in the cross-validation.
Forces and moment predicted for the validation dataset with the manoeuvring models fitted with proposed parameter estimation on the training set are shown in \autoref{fig:validation-forces}. It can be seen that the fitted AVMM overpredicts the forces by far. Therefore, simulations of the validation cases are only possible using the MAVMM, which is selected as the suitable manoeuvring model for the wPCC.
The simulations are shown for one of the ZigZag20/20 validation cases in \hyperref[\detokenize{06.10_results_wpcc:fig-validation-sim}]{Fig.\@ \ref{\detokenize{06.10_results_wpcc:fig-validation-sim}}}.

\begin{figure}[!htb]
\centering
\includegraphics{kappa/images/7.pdf}
\caption{Validation of force models for wPCC ZigZag20/20.}\label{fig:validation-forces}
\end{figure}

\begin{figure}[!htb]
\centering
\includegraphics{kappa/images/8.pdf}
\caption{Validation with simulations for wPCC ZigZag20/20.}\label{\detokenize{06.10_results_wpcc:fig-validation-sim}}\end{figure}

\noindent The over-prediction of forces with the AVMM can be explained by the large problems with multicollinearity that were encountered when applying the parameter estimation method to the wPCC data. The absolute correlation coefficient between the features in the wPCC yaw moment regression are shown in \hyperref[\detokenize{06.10_results_wpcc:fig-ncorr}]{Fig.\@ \ref{\detokenize{06.10_results_wpcc:fig-ncorr}}}. It can be seen that most of the coefficients have very high absolute correlation (indicated in black). Some of the regressed hydrodynamic derivatives in the AVMM also have a substantial values and large uncertainty.

\begin{figure}[!htb]
\centering
\includegraphics[width=\textwidth]{kappa/images/9.pdf}
\caption{Absolute correlation between the features in the wPCC yaw moment regression of AVMM}\label{\detokenize{06.10_results_wpcc:fig-ncorr}}\end{figure}
\noindent Advance and tactical diameter \cite{imo_standards_2002} from the predicted turning circle with the final manoeuvring model (MAVMM) differs 4\% and 1\% to the model test data as seen in \autoref{\detokenize{06.10_results_wpcc:tab-wpcc-advance}}. Results from the turning circle prediction are also shown in  \autoref{\detokenize{06.10_results_wpcc:fig-track-plot-testing-sim}} and  \autoref{\detokenize{06.10_results_wpcc:fig-testing-sim}}. Monte Carlo simulations with alternative realizations of the regression, considering the uncertainty in the regressed parameters, are also shown in these figures. The alternative realizations have similar simulation results to the model with mean values of the regression (black line).

\begin{figure}[!htb]
\centering
\includegraphics[width=\textwidth]{kappa/images/10.pdf}
\caption{Turning circle test case for wPCC, track plots from model test and simulation.}\label{\detokenize{06.10_results_wpcc:fig-track-plot-testing-sim}}\end{figure}

\begin{figure}[!htb]
\centering
\includegraphics[width=\textwidth]{kappa/images/11.pdf}
\caption{Turning circle test case for wPCC, time series from model test and simulation.}\label{\detokenize{06.10_results_wpcc:fig-testing-sim}}\end{figure}
\begin{table}[!htb]
    \footnotesize
    \caption{wPCC Predicted turning circle advance and tactical diameter compared to SSPA model tests and IMO limit}
    \label{\detokenize{06.10_results_wpcc:tab-wpcc-advance}}
    \centering
    \begin{tabular}{|T|T|T|T|T|}
\hline
&

Advance {[}m{]}
&

Advance (IMO) {[}m{]}
&

Tactical diameter {[}m{]}
&

Tactical diameter (IMO) {[}m{]}
\\
\hline

Model test
&

12.82
&

22.57
&

14.76
&

25.07
\\

Prediction
&

13.3
&

22.57
&

14.93
&

25.07
\\
\hline
\end{tabular}

\end{table}


\clearpage
\noindent The mean values and standard error (se) of the hydrodynamic derivatives expressed with prime units for the wPCC obtained with parameter estimation of MAVMM (\autoref{equation:02.01_manoeuvring models:eqxmartinssimple}, \autoref{equation:02.01_manoeuvring models:eqymartinssimple},  \autoref{equation:02.01_manoeuvring models:eqnmartinssimple}) applied on all the wPCC data (including the turning circle)  are shown in \hyperref[\detokenize{06.10_results_wpcc:wpcc-derivatives}]{Table \ref{\detokenize{06.10_results_wpcc:wpcc-derivatives}}}.

\begin{table}[!htb]
    \footnotesize
    \centering
    \caption{wPCC MAVMM derivatives (prime units times 1000).}
    \label{\detokenize{06.10_results_wpcc:wpcc-derivatives}}
    \begin{tabular}{|T|T|T|T|T|T|T|T|T|}
\hline


name
&

mean
&

se
&

name
&

mean
&

se
&

name
&

mean
&

se
\\
\hline

\( X_{\delta\delta} \)
&

-2.927
&

0.011
&

\( Y_{ur} \)
&

-65.507
&

0.082
&

\( N_{\delta} \)
&

-1.993
&

0.002
\\


\( X_{vr} \)
&

-7.737
&

0.066
&

\( Y_{v} \)
&

-20.347
&

0.016
&

\( N_{T\delta} \)
&

-5.392
&

0.599
\\


\( X_{rr} \)
&

-1.413
&

0.026
&

\( Y_{u} \)
&

-0.027
&

0.001
&

\( N_{r} \)
&

-37.341
&

0.096
\\


\( X_{uu} \)
&

20.124
&

0.137
&

\( Y_{r} \)
&

64.14
&

0.083
&

\( N_{u} \)
&

-0.003
&

0.0
\\


\( X_{u} \)
&

-20.948
&

0.137
&&&&

\( N_{ur} \)
&

35.525
&

0.096
\\
&&&&&&

\( N_{v} \)
&

-0.05
&

0.004
\\
&&&&&&

\( N_{vv\delta} \)
&

-19.051
&

0.054
\\
\hline
\end{tabular}

\end{table}



\subsection{The KVLCC2 test case}
\label{\detokenize{05.01_case_studies:the-kvlcc2-test-scenarios}}

The proposed method is also validated using the KVLCC2 case study ship model.
The propeller is part of the manoeuvring model for this test case, instead of only the hull and rudders, as in the wPCC test case, so that the entire ship can be simulated without additional input.
The model development process as described in \autoref{sec:model_development_process} is applied for the KVLCC2 as well.
Here,
\begin{itemize}
    \item Training dataset: various zigzag tests to starboard and port from model tests carried out at HSVA for the SIMMAN2008 conference \cite{stern_experience_2011}.
    \item Validation dataset: ZigZag35/5 carried out at HSVA for the SIMMAN2008 conference \cite{stern_experience_2011}.
    \item Test dataset: turning circle model tests carried out at MARIN for the SIMMAN2008 conference \cite{stern_experience_2011}
\end{itemize}
\noindent The split can also be seen in \ref{fig:kvlcc2_datasets}. A propeller prediction model is also needed for the KVLCC2, which is developed based on thrust measurements from the model tests as described in the next section.

\begin{figure}[!htb]
\centering
\includegraphics[width=\linewidth]{kappa/images/4.pdf}
\caption{KVLCC2 training, validation and testing datasets.}\label{fig:kvlcc2_datasets}\end{figure}

\newpage
\subsubsection{The KVLCC2 propeller model}
\label{\detokenize{06.20_results_kvlcc2:the-kvlcc2-propeller-model}}\label{\detokenize{06.20_results_kvlcc2:results-propeller-model}}

The coefficients of \(K_T\) (\autoref{equation:02.10_propeller_model:eqkt}) were regressed from the KVLCC2 propeller characteristics from SIMMAN2008 HSVA model tests \cite{stern_experience_2011} (\(k_0\):{0.32419}, \(k_1\):{-0.22091}, \(k_2\):{-0.14905}).
The Polynomial propeller model was developed with polynomial regression and cross-validation on the training and validation datasets to make the best feature selection.
A cross-validation study was carried out on the three candidate propeller models: 
\begin{itemize}
    \item the MMG propeller model
    \item the simple propeller model
    \item the Polynomial propeller model
\end{itemize}
The training and validation sets were made of the entire model test time series from the HSVA model tests. The model tests were divided into the test and validation sets randomly. The random training and validation were repeated 100 times. The Polynomial model was selected, having the highest accuracy. Taylor wake \(w_{p0}\) = {0.4} was used in all three models, the MMG model used \(C_1\)={2.0}, \(C_2\)={1.6} when \(\beta_p>0\) and \(C_2\)={1.1} when \(\beta_p<=0\) \cite{yasukawa_introduction_2015-1}. \hyperref[\detokenize{06.20_results_kvlcc2:fig-propeller-validation}]{Fig.\@ \ref{\detokenize{06.20_results_kvlcc2:fig-propeller-validation}}} shows a small part of the cross-validation and coefficients of the polynomial propeller model fitted on the training and validation dataset for KVLCC2 is shown in \hyperref[\detokenize{06.20_results_kvlcc2:kvlcc2-propeller-model}]{Table \ref{\detokenize{06.20_results_kvlcc2:kvlcc2-propeller-model}}}.

\begin{figure}[!htb]
\centering
\includegraphics{kappa/images/12.pdf}
\caption{Validation of MMG, Simple and Polynomial propeller models for KVLCC2.}\label{\detokenize{06.20_results_kvlcc2:fig-propeller-validation}}\end{figure}
\begin{table}[h]
    \centering
        \caption{KVLCC2 propeller model.}
    \label{\detokenize{06.20_results_kvlcc2:kvlcc2-propeller-model}}
    \begin{tabular}{l l l}
\toprule
\sphinxstyletheadfamily &\sphinxstyletheadfamily 
\sphinxAtStartPar
\(\beta_p>0\)
&\sphinxstyletheadfamily 
\sphinxAtStartPar
\(\beta_p<=0\)
\\
\hline
\sphinxAtStartPar
\(C_1\)
&
\sphinxAtStartPar
-0.1735
&
\sphinxAtStartPar
-0.1066
\\

\sphinxAtStartPar
\(C_2\)
&
\sphinxAtStartPar
0.4589
&
\sphinxAtStartPar
0.0771
\\

\sphinxAtStartPar
\(C_3\)
&
\sphinxAtStartPar
-1.8865
&
\sphinxAtStartPar
1.2958
\\

\sphinxAtStartPar
\(C_4\)
&
\sphinxAtStartPar
0.0515
&
\sphinxAtStartPar
0.0514
\\
\bottomrule
\end{tabular}

\end{table}



\subsubsection{KVLCC2 manoeuvring model}
\label{\detokenize{06.20_results_kvlcc2:kvlcc2-manoeuvring-model}}

The linear manoeuvring model (LVMM) was ruled too simple, for KVLCC2, so only the AVMM and MAVMM were considered possible manoeuvring models in the cross-validation.
The forces and moments applied on the hull, rudder, and propeller predicted with the AVMM and MAVMM fitted with the proposed parameter estimation on the training set are shown in \hyperref[\detokenize{06.20_results_kvlcc2:fig-kvlcc2-validation-forces}]{Fig.\@ \ref{\detokenize{06.20_results_kvlcc2:fig-kvlcc2-validation-forces}}}.
The forces are well predicted with both manoeuvring models. The AVMM is not giving the large over predictions that were seen for wPCC. However, the MAVMM is still slightly better and is therefore selected as the suitable manoeuvring model for the KVLCC2.
Simulations of the validation cases with the MAVMM is shown for one of the ZigZag20/20 validation cases in \hyperref[\detokenize{06.20_results_kvlcc2:fig-kvlcc2-validation-sim}]{Fig.\@ \ref{\detokenize{06.20_results_kvlcc2:fig-kvlcc2-validation-sim}}} and \hyperref[\detokenize{06.20_results_kvlcc2:fig-kvlcc2-validation-sim-error}]{Fig.\@ \ref{\detokenize{06.20_results_kvlcc2:fig-kvlcc2-validation-sim-error}}} where the predicted thrust is also shown.

\begin{figure}[!htb]
\centering
\includegraphics[width=\textwidth]{kappa/images/13.pdf}
\caption{Validation of force models for KVLCC2.}\label{\detokenize{06.20_results_kvlcc2:fig-kvlcc2-validation-forces}}\end{figure}

\begin{figure}[!htb]
\centering
\includegraphics[width=\textwidth]{kappa/images/14.pdf}
\caption{Validation with simulations for KVLCC2.}\label{\detokenize{06.20_results_kvlcc2:fig-kvlcc2-validation-sim}}\end{figure}

\begin{figure}[!htb]
\centering
\includegraphics{kappa/images/15.pdf}
\caption{Validation error (prediction-model test) with simulations for KVLCC2.}\label{\detokenize{06.20_results_kvlcc2:fig-kvlcc2-validation-sim-error}}\end{figure}

\newpage
\noindent Results from the final prediction of the turning circle test are shown in  \hyperref[\detokenize{06.20_results_kvlcc2:fig-kvlcc2-track-plot-testing-sim}]{Fig.\@ \ref{\detokenize{06.20_results_kvlcc2:fig-kvlcc2-track-plot-testing-sim}}}, \hyperref[\detokenize{06.20_results_kvlcc2:fig-kvlcc2-testing-sim}]{Fig.\@ \ref{\detokenize{06.20_results_kvlcc2:fig-kvlcc2-testing-sim}}} and \hyperref[\detokenize{06.20_results_kvlcc2:fig-kvlcc2-testing-sim-error}]{Fig.\@ \ref{\detokenize{06.20_results_kvlcc2:fig-kvlcc2-testing-sim-error}}}. The prediction is conducted using simulation with the MAVMM trained on the training and validation dataset. Monte Carlo simulations with alternative realizations of the regression are also shown in this figure. The alternative realizations are very similar to the model with mean values of the regression (black line).

\begin{figure}[!htb]
\centering
\includegraphics{kappa/images/16.pdf}
\caption{Comparison between predicted Turning circle test with MAVMM trained on HSVA data and MARIN model test results for KVLCC2.}\label{\detokenize{06.20_results_kvlcc2:fig-kvlcc2-track-plot-testing-sim}}\end{figure}

\begin{figure}[!htb]
\centering
\includegraphics{kappa/images/17.pdf}
\caption{Comparison between predicted Turning circle test with MAVMM trained on HSVA data and MARIN model test results for KVLCC2.}\label{\detokenize{06.20_results_kvlcc2:fig-kvlcc2-testing-sim}}\end{figure}

\begin{figure}[!htb]
\centering
\includegraphics{kappa/images/18.pdf}
\caption{The prediction error (prediction-model test) for Turning circle test with MAVMM trained on HSVA data and MARIN model test results for KVLCC2.}\label{\detokenize{06.20_results_kvlcc2:fig-kvlcc2-testing-sim-error}}\end{figure}
\noindent For KVLCC2 comparisons of turning circle advance and tactical diameter compared to the model test result is shown in \hyperref[\detokenize{06.20_results_kvlcc2:tab-kvlcc2-advance}]{Table \ref{\detokenize{06.20_results_kvlcc2:tab-kvlcc2-advance}}}. Predicted advance and tactical diameter differ 2\% and 5\%, which can be considered acceptable, considering the margin to the IMO standard limits, which are also shown in this table. The results are also closer to the model tests than a similar study conducted for the KVLCC2 \cite{he_nonparametric_2022}.
\begin{table}[h]
    \centering
    \footnotesize
        \caption{KVLCC2 Predicted turning circle advance (A) and tactical diameter (TD) compared to MARIN model tests and IMO limit}
    \label{\detokenize{06.20_results_kvlcc2:tab-kvlcc2-advance}}
    \begin{tabular}{|p{0.7cm}|p{1.7cm}|p{1.3cm}|p{1.0cm}|p{1.3cm}|p{1.3cm}|p{1.0cm}|}
\hline
\sphinxstyletheadfamily 
\sphinxAtStartPar
delta
&\sphinxstyletheadfamily 
\sphinxAtStartPar
A (model test) {[}m{]}
&\sphinxstyletheadfamily 
\sphinxAtStartPar
A (prediction) {[}m{]}
&\sphinxstyletheadfamily 
\sphinxAtStartPar
A (IMO) {[}m{]}
&\sphinxstyletheadfamily 
\sphinxAtStartPar
TD (model test) {[}m{]}
&\sphinxstyletheadfamily 
\sphinxAtStartPar
TD (prediction) {[}m{]}
&\sphinxstyletheadfamily 
\sphinxAtStartPar
TD (IMO) {[}m{]}
\\
\hline
\sphinxAtStartPar
35.0
&
\sphinxAtStartPar
21.59
&
\sphinxAtStartPar
21.21
&
\sphinxAtStartPar
31.5
&
\sphinxAtStartPar
21.72
&
\sphinxAtStartPar
23.07
&
\sphinxAtStartPar
35.0
\\

\sphinxAtStartPar
-35.0
&
\sphinxAtStartPar
22.54
&
\sphinxAtStartPar
22.1
&
\sphinxAtStartPar
31.5
&
\sphinxAtStartPar
23.55
&
\sphinxAtStartPar
24.29
&
\sphinxAtStartPar
35.0
\\
\hline
\end{tabular}
\end{table}


\noindent The mean values and standard error (se) of the hydrodynamic derivatives expressed with prime units for the KVLCC2 obtained with parameter estimation of MAVMM (\(\autoref{equation:02.01_manoeuvring models:eqxmartinssimple}\), \(\autoref{equation:02.01_manoeuvring models:eqymartinssimple}\), \(\autoref{equation:02.01_manoeuvring models:eqnmartinssimple}\)) applied on all the HSVA data are shown in \hyperref[\detokenize{06.20_results_kvlcc2:kvlcc2-derivatives}]{Table \ref{\detokenize{06.20_results_kvlcc2:kvlcc2-derivatives}}}.
\begin{table}[h]
    \footnotesize
    \centering
        \caption{KVLCC2 MAVMM derivatives (prime units times 1000)}
    \label{\detokenize{06.20_results_kvlcc2:kvlcc2-derivatives}}
    \begin{tabular}{|T|T|T|T|T|T|T|T|T|}
\hline


name
&

mean
&

se
&

name
&

mean
&

se
&

name
&

mean
&

se
\\
\hline

\( X_{vr} \)
&

-11.454
&

0.272
&

\( Y_{T} \)
&

77.34
&

1.23
&

\( N_{\delta} \)
&

-1.274
&

0.003
\\
\hline

\( X_{rr} \)
&

-1.406
&

0.068
&

\( Y_{r} \)
&

256.065
&

0.654
&

\( N_{r} \)
&

-105.618
&

0.179
\\
\hline

\( X_{\delta\delta} \)
&

-2.719
&

0.013
&

\( Y_{v} \)
&

-24.467
&

0.02
&

\( N_{T} \)
&

-32.523
&

0.274
\\
\hline

\( X_{uu} \)
&

80.508
&

0.618
&

\( Y_{ur} \)
&

-252.991
&

0.658
&

\( N_{u} \)
&

0.063
&

0.001
\\
\hline

\( X_{u} \)
&

-81.415
&

0.618
&

\( Y_{u} \)
&

-0.119
&

0.003
&

\( N_{v} \)
&

-7.156
&

0.016
\\
\hline&&&&&&

\( N_{T\delta} \)
&

-391.596
&

0.941
\\
\hline&&&&&&

\( N_{vv\delta} \)
&

-19.257
&

0.089
\\
\hline&&&&&&

\( N_{ur} \)
&

102.252
&

0.183
\\
\hline
\end{tabular}

\end{table}


\newpage
\subsection{Inverse dynamics}
\label{\detokenize{06.40_results_inverse_dynamics:inverse-dynamics}}\label{\detokenize{06.40_results_inverse_dynamics::doc}}
The capability of the inverse dynamics on simulated data was also investigated in Paper \ref{pap:pit}. The hydrodynamic derivatives within the manoeuvring model can be identified exactly at ideal conditions for the parameter estimation with no measurement noise and a perfect estimator. For example, artificial data from a turning circle test can be simulated by a pre-defined/true manoeuvring model. The hydrodynamic derivatives within the manoeuvring model can be identified with the same values. Results from such a simulation is shown in \hyperref[\detokenize{06.40_results_inverse_dynamics:fig-bar-parameters}]{Fig.\@ \ref{\detokenize{06.40_results_inverse_dynamics:fig-bar-parameters}}} where the regression has identified the true values precisely.
\begin{figure}[!htb]
\centering
\includegraphics{kappa/images/5.pdf}
\caption{True and regressed hydrodynamic derivatives in MAVMM identified with Inverse dynamics and OLS regression on a simulated turning circle with MAVMM.}\label{\detokenize{06.40_results_inverse_dynamics:fig-bar-parameters}}\end{figure}


\subsection{Preprocessing}
\label{\detokenize{06.31_results_noise:preprocessing}}\label{\detokenize{06.31_results_noise::doc}}

The low-pass filter is a prevalent alternative to preprocessing the model test data, as opposed to the EKF used by the proposed parameter estimation.
In order to study which of the filters works best in Paper \ref{pap:pit}, the proposed parameter estimation has been run on the wPCC model test data with the EKF + RTS smoother replaced by a low-pass filter instead. The low-pass filter applies a first-order linear digital Butterworth filter twice, once forward and once backward, to get zero phase \cite{virtanen_scipy_2020}. \hyperref[\detokenize{06.31_results_noise:fig-lowpass-accuracy}]{Fig.\@ \ref{\detokenize{06.31_results_noise:fig-lowpass-accuracy}}} shows the average simulation error \( \overline{RMSE} \) with low-pass filters at various cut-off frequencies for all wPCC model tets. Corresponding error with parameter estimation using EKF + RTS is also shown in the figure. The simulation error for each model test is expressed as Root Mean Square Error \(RMSE\) (\autoref{equation:06.31_results_noise:eqrmse}) of the distance between the position from the model test and simulation.
\begin{equation}\label{equation:06.31_results_noise:eqrmse}
\begin{split}RMSE=\sqrt{ \frac{\sum_{n=1}^{N} (d_n^2) }{N}} \end{split}
\end{equation}

\noindent where \(d_n\) is the euclidean distance for each time step between the model test positions (\(x_0\), \(y_0\)) and the predicted positions.

\begin{figure}[!htb]
\centering
\includegraphics{kappa/images/6.pdf}
\caption{Average simulation error with MAVMM fitted on wPCC model test data using low-pass filters with various cutt off frequency or EKF.}\label{\detokenize{06.31_results_noise:fig-lowpass-accuracy}}\end{figure} 
\noindent Even though high accuracy can be obtained using a low-pass filter as the pre-processor, if an optimal cut-off frequency is selected, its accuracy decreases quickly at lower or higher frequencies. With higher cut-off frequencies, too much of the measurement error remains in the data, resulting in poor performance of the OLS regression. In extreme cases, it is like having no filter at all. Using too low of a cut-off frequency removes too much, including parts of the actual signal. The results show that the low-pass filter with a 7 Hz cut-off frequency has the lowest error among the low-pass filters, but EKF + RTS in the parameter estimation has an even lower error, which is why this is used as the preprocessor in the proposed parameter estimation.
