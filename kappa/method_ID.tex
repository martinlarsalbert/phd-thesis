\section{Inverse dynamics} \label{sec:ID}
Inverse dynamics (ID) was found to be an efficient method to identify the parameters within the roll motion models. ID is a widely used technique in robotics \cite{faberInverseDynamicsMechanical2018, haningerNonparametricInverseDynamic2019, mastalliInverseDynamicsMPCNullspace2023, sunHighorderInverseDynamics2023, kurtzInverseDynamicsTrajectory2023} that is also well applicable to ship dynamics. It can be used to estimate the total forces that act on a ship during motion. The technique can be applied to data from free-model manoeuvring tests or real ship maneuvers. The forces acting on the ship during a maneuver can be estimated with inverse dynamics of the equation of motion (\autoref{eq:eom}) when the mass matrix $\mathbf{M}$ and the acceleration vector $\pmb{\dot{\upsilon}}$ are known. The hydrodynamic damping forces can be calculated by inserting the total force $\mathbf{F}$ from \autoref{eq:F_expanded} into \autoref{eq:eom} and then solving for $X_D$, $Y_D$, and $N_D$ as shown in \autoref{eq:ID_X}-\autoref{eq:ID_N}.
\begin{equation}
    \label{eq:ID_X}
    X_{D} = - X_{\dot{u}} \dot{u} + Y_{\dot{r}} r^{2} + Y_{\dot{v}} r v + \dot{u} m - m r^{2} x_{G} - m r v
\end{equation}
\begin{equation}
    \label{eq:ID_Y}
    Y_{D} = - X_{\dot{u}} r u - Y_{\dot{r}} \dot{r} - Y_{\dot{v}} \dot{v} + \dot{r} m x_{G} + \dot{v} m + m r u
\end{equation}
\begin{equation}
    \label{eq:ID_N}
    N_{D} = I_{z} \dot{r} - N_{\dot{r}} \dot{r} - N_{\dot{v}} \dot{v} + X_{\dot{u}} u v - Y_{\dot{r}} r u - Y_{\dot{v}} u v + \dot{v} m x_{G} + m r u x_{G}
\end{equation}
These expressions are used to estimate the forces acting on the ship during FRMTs, for instance as shown for a turning circle test in \autoref{fig:ID}.
%\begin{figure}[H]
%    \centering
%    \includegraphics[width=\textwidth]{kappa/images/1.pdf}
%    \caption{Forces and moments calculated with inverse dynamics on data from a turning circle test.}
%    \label{fig:ID}
%\end{figure}
\begin{figure}[h]
     \centering
     \begin{subfigure}[b]{0.49\textwidth}
         \centering
         \includesvg{figures/inverse_dynamics_forces.inverse_dynamics_circle.svg}
        \caption{Track plot.}
        \label{fig:sim_optiwise_10_port}
     \end{subfigure}
     \hfill
     \begin{subfigure}[b]{0.49\textwidth}
         \includesvg{figures/inverse_dynamics_forces.inverse_dynamics_forces.svg}
        \caption{Inverse dynamics.}
        \label{fig:sim_optiwise_10_stbd}
     \end{subfigure}
    \caption{Forces and moments calculated with inverse dynamics on data from a turning circle test.}
    \label{fig:ID}
\end{figure}

The estimated inverse dynamics forces were used in Paper \ref{pap:pit} and \ref{pap:physics} as input to an inverse dynamics regression (see \autoref{sec:IDR}). 
Inverse dynamics was also used in Paper \ref{pap:physics} and \ref{pap:vct} to estimate the forces acting on the ship during the FRMTs to be compared with the model force predictions. This is a more informative way to assess model performance than, for instance, using open-loop or closed-loop simulations. The benefit is that the model and the experiment will always be in the same state, which is not the case when simulations are used.