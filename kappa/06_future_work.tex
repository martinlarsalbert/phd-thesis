%%%%%%%%%%%%%%%%%%%%%%%%%%%%%%%
%%%%%%%%%%%%%%%%%%%%%%%%%%%%%%%
\chapter{Future work\label{ch:future_work}}
%%%%%%%%%%%%%%%%%%%%%%%%%%%%%%%
% Loosen up on the assumptions?
%Use a hyphen if "long-term" is an adjective.

\noindent The rigid body assumption during maneuvers is reasonable considering the relatively low accelerations and bending of the hull girder during maneuvers, at least in the absence of waves. However, there are other assumptions and limitations that can pave the way for future research, as described below.

\subsection*{Only data from standard test types such as turning circles or zigzag tests were used in this thesis, since they are commonly available for ships}
It has been shown in this thesis that identifying a physically correct manoeuvring model from data with only standard maneuvers is very difficult, due to high multicollinearity and insufficient persistence of excitation. 
It was shown that prior knowledge about manoeuvring hydrodynamics embedded in the model structure together with good semi-empirical formulas can help to mitigate these problems.
However, it does not solve the problem completely. More informative data is needed to identify a fully physically correct model which could be obtained with other types of maneuvers, such as pseudo random binary sequence (PRBS) \cite{yoonIdentificationHydrodynamicCoefficients2003,wangOptimalDesignExcitation2020}. Studies of system identification on these kinds of informative maneuvers have mainly been conducted with simulated data. Collecting experimental data for these informative maneuvers would be a great contribution. The maneuvers require more space than is available in a model test basin. \textcite{millerShipModelIdentification2021} conducted such tests on a lake and pointed out that they are very difficult and time-consuming to carry out. 
More work is needed to establish reliable experimental research data from maneuvers in which all modes of the manoeuvring dynamics of the ship are excited, which could perhaps be conducted with one of the more well-researched test cases, such as for instance KVLCC2. 

\subsection*{Bayesian modeling}
Compared to the methods used in this thesis that incorporate prior knowledge about ship hydrodynamics, Bayesian modeling offers a more sophisticated approach by expressing prior knowledge as prior probability densities. Informative priors can guide parameter identification towards hydrodynamic derivatives that are physically reasonable, based on prior knowledge from similar ships, even for standard manoeuvres that lack persistence of excitation. However, developing informative priors for hydrodynamic derivatives would require significant research efforts, such as creating a comprehensive database of identified manoeuvring models for numerous ships. This effort would enable the identification of models with much better generalization from standard manoeuvres. Additionally, Bayesian modeling with informative hydrodynamic priors could have valuable applications in the system identification of full-scale ship operations and autonomous ships. 

\subsection*{Calm water assumption}
The sea is never calm, so this assumption within manoeuvring greatly simplifies the real conditions encountered by ships. Relaxing this assumption would significantly complicate system identification. The fluid memory effect would need to be addressed and the constant added mass assumption would no longer be valid. A different approach from what has been presented in this thesis would be required. A data-driven model for viscous manoeuvring forces could potentially be coupled with potential flow calculations, similar to the method used for roll motion in Paper \ref{pap:ikeda}.
    
\subsection*{The free surface effects and the influence of the roll were not included in the VCT data}
It was shown in Paper \ref{pap:vct} that a manoeuvring model could be identified from VCT to predict standard maneuvers with good precision for one of the tested cases, but not for the other ship. It was argued that this discrepancy was due to the unjustified assumption of neglecting the free surface and roll influence for this ship, a statement that requires further investigation to validate. In addition, a better understanding of when these assumptions can be applied is necessary.  
        
