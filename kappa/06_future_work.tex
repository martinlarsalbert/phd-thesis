%%%%%%%%%%%%%%%%%%%%%%%%%%%%%%%
%%%%%%%%%%%%%%%%%%%%%%%%%%%%%%%
\chapter{Future work\label{ch:future_work}}
%%%%%%%%%%%%%%%%%%%%%%%%%%%%%%%
%Use a hyphen if "long-term" is an adjective.
A long-term objective of this research is to develop system identification methods for rigid body ship dynamics of full scale ships in real sea conditions. The study of these dynamics is an important sub-component of ship digital twins, which can be used to investigate alternative scenarios of the real ship's operation. The system identification methods can be used in advanced autopilots or unmanned surface ships (USVs). Multiple aspects of the ship's energy consumption originate from its motions. Components of the motions include added resistance in waves or added resistance from the wind and currents. Prediction models for the ship's dynamics can therefore be used to optimize the energy consumption. The investigation of the ship dynamics in a calm-water laboratory environment conducted in this thesis is one step towards the long-term objective. Possible additions to this research will be discussed further in this section.  

\subsubsection*{\normalfont \color{black} \textbf{System identification of model scale rigid body ship dynamics in waves}}
The calm water assumption used in this thesis reflects a situation that is rare in reality because the sea is never completely calm. The wind, waves, and currents will always influence the movement of real ships. To expand the research beyond the calm water assumption, the development of system identification methods for more realistic conditions is necessary. Entering the seakeeping sub field of ship dynamics introduces new challenges to the system identification compared to the manoeuvring and roll motions, which are studied in this thesis. Seakeeping comprises two more degrees of freedom: heave and pitch. Both of these factors increase the complexity of the models. More importantly, the constant added mass assumption used in this thesis is no longer valid under the influence of wave forces; this means that alternative system identification methods are needed. Wind, waves, and currents all add many uncertainties to the system identification. Conducting system identification of seakeeping model tests data is a way to control these uncertainties. The influence from the waves can be studied in isolation from the wind and currents. The measurement accuracy of the ship and wave motions is also much higher in the laboratory, which makes system identification of model scale rigid body ship dynamics in waves a worthy next step for the research.  

\subsubsection*{\normalfont \color{black} \textbf{System identification of full scale rigid body ship dynamics in wind and current}}
The calm water assumption used in this thesis may be valid for ships operating on inland waters or sheltered coastal areas. The validity is especially likely if the calm water assumption is expanded to include the influence of the wind and currents. System identification of the ship's rigid body dynamics can be conducted on data from full-scale operation of ships in these conditions. This would also be a necessary next step for the research because it adds the uncertainties from the wind and current in full scale while excluding the influence from the waves.

\subsubsection*{\normalfont \color{black} \textbf{System identification of full scale rigid body ship dynamics}}
Given that a satisfactory system identification can be obtained as described by the two steps above, the logical following step would be to combine them in full-scale operation in wind, currents, and waves. To fulfill the requirements of the research project and as a direct, natural continuation of the system identification methods, the data analysis method and established ship dynamic digital twin models can be used to study how ship dynamics can affect a ship's operational performance (fuel consumption and speed loss) in the open sea. This could be another piece of the ship digital twin project. In particular, after establishing the ship digital twin, some online learning and identification methods must be studied and researched to update the digital twin for various applications for ship operation.