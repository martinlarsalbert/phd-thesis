%%%%%%%%%%%%%%%%%%%%%%%%%%%%%%%
%%%%%%%%%%%%%%%%%%%%%%%%%%%%%%%
\chapter{Future work\label{ch:future_work}}
%%%%%%%%%%%%%%%%%%%%%%%%%%%%%%%
% Loosen up on the assumptions?
%Use a hyphen if "long-term" is an adjective.

\noindent The rigid body assumption during maneuvers is reasonable considering the relatively low accelerations and bending of the hull girder during maneuvers, at least in the absence of waves. However, there are other assumptions and limitations that can be addressed through future research, as described below.

\subsection*{Only data from standard test types, such as turning circles or zigzag tests were used in this thesis, since they are commonly available for ships}
It has been shown in this thesis that identifying a physically correct manoeuvring model from data with only standard maneuvers is very difficult, due to high multicollinearity and insufficient persistence of excitation. 
It was further demonstrated that prior knowledge of manoeuvring hydrodynamics embedded in the model structure together with good semi-empirical formulas can help to mitigate, but not completely resolve, these problems. More informative data are needed to identify a fully physically correct model that could be obtained with other types of maneuvers, such as pseudo random binary sequence (PRBS) \cite{yoonIdentificationHydrodynamicCoefficients2003,wangOptimalDesignExcitation2020}. Studies of system identification based on these kinds of informative maneuvers have primarily been conducted with simulated data. Collecting experimental data for these informative maneuvers would be a significant contribution,  since they require more space than is available in a model test basin. \textcite{millerShipModelIdentification2021} conducted such tests on a lake and noted the difficulty and time-consuming nature of this approach.
More work is needed to establish reliable experimental research data from maneuvers in which all modes of the manoeuvring dynamics of the ship are excited, which could perhaps be conducted with one of the more well-researched test cases, such as KVLCC2. 

\subsection*{Bayesian modeling}
In contrast to the methods used in this thesis, 
 which incorporate prior knowledge of ship hydrodynamics, Bayesian modeling offers a more sophisticated approach by expressing prior knowledge as prior probability densities. Informative priors can guide parameter identification toward hydrodynamic derivatives that are physically reasonable, based on prior knowledge from similar ships, even for standard manoeuvres that lack persistence of excitation. However, developing informative priors for hydrodynamic derivatives would require significant research effort, such as creating a comprehensive database of identified manoeuvring models for numerous ships. This effort would enable the identification of models with much better generalization from standard manoeuvres. Additionally, Bayesian modeling with informative hydrodynamic priors could have valuable applications in the system identification of full-scale ship operations and autonomous ships. 

\subsection*{Calm water assumption}
The sea is never calm, so this assumption within manoeuvring greatly simplifies the real-world conditions encountered by ships. Relaxing this assumption would significantly complicate system identification. The fluid-memory effect would need to be addressed, and the assumption of constant added mass would no longer be valid. An alternative approach to those presented in this thesis would be required. A data-driven model for viscous manoeuvring forces could potentially be coupled with potential flow calculations, akin to the method used for roll motion in Paper \ref{pap:ikeda}.
    
\subsection*{The free surface effects and the influence of the roll were not included in the VCT data}
It was shown in Paper \ref{pap:vct} that a manoeuvring model could be identified from VCT to predict standard maneuvers with good precision for only one of the test cases. It was argued that this discrepancy was due to the unjustified assumption of neglecting the free surface and roll influence for this ship. This statement should be validated through further investigation. In addition, a better understanding of when these assumptions can be applied is necessary.  
        
