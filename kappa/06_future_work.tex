%%%%%%%%%%%%%%%%%%%%%%%%%%%%%%%
%%%%%%%%%%%%%%%%%%%%%%%%%%%%%%%
\chapter{Future work\label{ch:future_work}}
%%%%%%%%%%%%%%%%%%%%%%%%%%%%%%%
% Loosen up on the assumptions?
%Use a hyphen if "long-term" is an adjective.

\noindent The rigid body assumption during maneuvers is reasonable considering the relatively low accelerations and bending of the hull girder during maneuvers, at least in the absence of waves. However, there are other assumptions and limitations that can pave the way for future research, as described below.

\subsection*{Only data from standard test types such as turning circles or zigzag tests were used in this thesis, since they are commonly available for ships}
It has been shown in this thesis that identifying a physically correct manoeuvring model from data with only standard maneuvers is very difficult, due to high multicollinearity and insufficient persistence of excitation. 
It was shown that prior knowledge about manoeuvring hydrodynamics embedded in the model structure together with good semi-empirical formulas can help to mitigate these problems.
However, it does not solve the problem completely. More informative data is needed to identify a fully physically correct model. Here are some suggestions for alternative ways to achieve this.
\begin{enumerate}[label=(\roman*),itemsep=1mm]
    
    \item More informative data can be obtained with other types of maneuvers, such as pseudo random binary sequence (PRBS) \cite{yoonIdentificationHydrodynamicCoefficients2003,wangOptimalDesignExcitation2020}. Studies of system identification on these kinds of informative maneuvers have mainly been conducted with simulated data. Collecting experimental data for these informative maneuvers would be a great contribution. The maneuvers require more space than is available in a model test basin. \textcite{millerShipModelIdentification2021} conducted such tests on a lake and pointed out that they are very difficult and time-consuming to carry out. More work is needed to obtain reliable data in which all modes of the ship manoeuvring dynamics is excited.
    
    \item It was shown in Paper \ref{pap:physics} that due to the high correlation between yaw rate and drift, it is hard to make a correct split between the forces that depend on them. It would be good to establish some prior knowledge about how to make this split, based on earlier investigations for other ships, that can be used in a Bayesian way to propose the most likely split during the system identification for new ships.  
        
\end{enumerate}

\subsection*{Calm water assumption}
The sea is never calm, so this assumption within manoeuvring is a great simplification of the real conditions encountered by ships. Loosening up on this assumption would make system identification much harder. The fluid memory effect would have to be handled and the constant added mass assumption can no longer be justified. A different approach to what has been presented in this thesis would have to be adopted. A data-driven model for the viscous manoeuvring forces can perhaps be coupled with potential flow calculations in a similar way as it was done for the roll motion in Paper \ref{pap:ikeda}.  
    
\subsection*{VCT data was assumed to be correct ignoring the free surface effects and the influence of roll}
It was shown in the Paper \ref{pap:vct} that a manoeuvring model could be identified from VCT to predict standard maneuvers with good precision for one of the tested cases, but not the other ship. 
It was argued that this was because the assumed neglection of the free surface and roll influence was not justified for this ship, which is a statement that would need further investigations to prove. A better understanding of when these assumptions can be applied is also needed.   
        
