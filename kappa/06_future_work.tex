%%%%%%%%%%%%%%%%%%%%%%%%%%%%%%%
%%%%%%%%%%%%%%%%%%%%%%%%%%%%%%%
\chapter{Future work\label{ch:future_work}}
%%%%%%%%%%%%%%%%%%%%%%%%%%%%%%%
A long term objective for this research is to develop system identification methods for ship rigid body dynamics of full scale ships in real sea conditions. This is intended as an important sub-component in ship digital twins, which can be used to investigate alternative scenarios of the real ship's operation. The methods can for instance be used in advanced autopilots or unmanned surface ships (USVs). Many aspects of the ship's energy consumption originate from its motions as for instance added resistance in waves, or added resistance from wind and current. Prediction models for the ship's dynamics can therefore be used to optimize the energy consumption. The investigation of the ship dynamics in a calm water laboratory environment conducted in this thesis is one step towards the long term objective. There are many possible next steps in this research, which will be discussed further below.  

\subsubsection*{System identification of model scale ship rigid body dynamics in waves}
The calm water assumption used in this thesis reflects a situation that is very rare in reality, since the sea is never completely calm. There will always be influences from wind, waves and currents acting on real ships. Going beyond the calm water assumption, development of system identification methods for these conditions is a possible next step in this research. Entering the seakeeping sub field of ship dynamics introduces new challenges in the system identification compared to the manoeuvring and roll motions, studied in this thesis. Seakeeping comprises two more degrees of freedoms: heave and pitch, which increases the complexity of the models. But more importantly, the constant added mass assumption used in this thesis is no longer valid under the influence of wave forces, which means that other kinds of system identification methods are needed. Wind, waves and current all add a lot of uncertainties to the system identification. Conducting system identification of seakeeping model tests data, is a way to control these uncertainties. The influence from the waves can be studied in isolation from the wind and current. The measurement accuracy of the ship and wave motions is also much higher in the laboratory, which makes system identification of model scale ship rigid body dynamics in waves an interesting next step for the research.  

\subsubsection*{System identification of full scale ship rigid body dynamics in wind and current}
The calm water assumption used in this thesis can be approximately valid for ships operating on inland waters or sheltered coastal areas. Especially if the calm water assumption is expanded, to also include the influence from wind and current. System identification of the ship rigid body dynamics could be conducted on data from full scale operation of ships in these conditions. This would also be an interesting next step for the research, adding the uncertainties from the wind and current in full scale, but excluding the influence from waves.

\subsubsection*{System identification of full scale ship rigid body dynamics}
Given that a satisfactory system identification can be obtained as described by the two steps above, the logical following step would be to combine them, in full scale operation in wind, current and waves. To reflect to fulfil requirements of the research project and as a direct, natural continuation of your system identification methods, we could exploit the system identification method and established ship dynamic digital twin models, to study how ship dynamics can affect a ship's operational performance (fuel consumption and speed loss) in open sea, by making using of the data analysis method. This can be another piece of the ship digital twin project. Especially, after establishing the ship digital twin, some online learning/identification methods will be studied/researched to update the digital twin for various ship operation related applications.