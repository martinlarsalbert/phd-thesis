\subsection{Inverse dynamics regression} \label{sec:IDR}
Finding the hydrodynamic derivatives can be defined as a linear regression problem following the ''derivation approach'' (see \autoref{sec:derivation_approach}):
\begin{equation}\label{equation:03.01_inverse_dynamics:eqregression}
\begin{split}y = X\gamma + \epsilon\end{split}
\end{equation}

\noindent The label vector \(y\) and feature matrix \(X\) in the regression problem in \autoref{equation:03.01_inverse_dynamics:eqregression} can be calculated if model for the hydrodynamic forces is assumed. For example: the label in the regression of the surge degree of freedom for the MAVMM can be calculated using the inverse dynamics force, which is expressed with primed units:
\begin{equation}\label{equation:03.01_inverse_dynamics:diff_eq_X_y}
\begin{split}\displaystyle y = - X_{\dot{u}} \dot{u}' + \dot{u}' m' - m' r'^{2} x_{G'} - m' r' v'\end{split}
\end{equation}

\noindent The feature matrix \(X\) is expressed as:
\begin{equation}\label{equation:03.01_inverse_dynamics:diff_eq_X_X}
\begin{split}\displaystyle X = \left[\begin{matrix}thrust' & u' & \delta^{2} & r'^{2} & u'^{2} & r' v'\end{matrix}\right]\end{split}
\end{equation}

\noindent The hydrodynamic derivatives in the \(\gamma\) vector (\autoref{equation:03.01_inverse_dynamics:diff_eq_X_beta}) can be estimated with ordinary least squares (OLS) regression.
\begin{equation}\label{equation:03.01_inverse_dynamics:diff_eq_X_beta}
\begin{split}\displaystyle \gamma = \left[\begin{matrix}X_{T}\\X_{u}\\X_{\delta\delta}\\X_{rr}\\X_{uu}\\X_{vr}\end{matrix}\right]\end{split}
\end{equation}
In this regression, the hydrodynamic derivatives are treated as Gaussian random variables. The hydrodynamic derivatives in the manoeuvring model are usually estimated as the mean value of each regressed random variable, which is the most likely estimate. The regression result can be expressed with a multivariate Gaussian distribution, which is defined by the regression’s mean values and covariance matrix. The multivariate Gaussian distribution can be used to conduct Monte Carlo simulations in the study of alternative realizations of the regression.

Strong multicollinearity is a documented problem for the manoeuvring models \cite{luo_parameter_2016, wang_quantifying_2018}.
The thrust coefficient \(X_T\) in the hydrodynamic function \(X_D\) in \autoref{equation:02.01_manoeuvring models:eqxabkowitz} introduces multicollinearity to the regression. This coefficient can instead be calculated from the thrust deduction factor \(t_{df}\):
\begin{equation}\label{equation:03.01_inverse_dynamics:eqXthrust}
\begin{split}\displaystyle X_{T} = 1 - t_{df}\end{split}
\end{equation}

\noindent The \(X_T\) coefficient is excluded from the regression by moving it to the left-hand side of the regression equation \autoref{equation:03.01_inverse_dynamics:eqregression}:
\begin{equation}\label{equation:03.01_inverse_dynamics:eqexclude}
\begin{split}y-X_T \cdot thrust = X \gamma + \epsilon\end{split}
\end{equation}

\noindent Rudder coefficients (\(Y_R\)) from \(Y_D\) equation \autoref{equation:02.01_manoeuvring models:eqyabkowitz}, such as \(Y_{\delta}\) and \(Y_{\delta T}\), have also been excluded by assuming a connection with their \(N_D\) equation counterpart through the rudder lever arm \(x_r\):
\begin{equation}\label{equation:03.01_inverse_dynamics:eqyr}
\begin{split}\displaystyle Y_{R} = \frac{N_{R}}{x_{r'}}\end{split}
\end{equation}

\subsection{Inverse dynamics and regression}
\label{\detokenize{03.01_inverse_dynamics:inverse-dynamics-and-regression}}\label{\detokenize{03.01_inverse_dynamics::doc}}

Each manoeuvring model has some hydrodynamic functions \(X_D(u,v,r,\delta,thrust)\), \(Y_D(u,v,r,\delta,thrust)\), \(N_D(u,v,r,\delta,thrust)\) that are defined as polynomials. The hydrodynamic derivatives in these polynomials can be identified with force regression of measured forces and moments. The measured forces and moments are usually taken from captive model tests (CMT), planar motion mechanism (PMM) tests, or virtual captive tests (VCT). However, motions are recorded when the ship is free in all degrees of freedom. Hence, forces and moments causing ship motion must be estimated by solving the inverse dynamics problem.
The inverse dynamics problem is solved by restructuring the system equation (\autoref{equation:02.01_manoeuvring models:eqacc}) to get the hydrodynamics functions on the left-hand side. If the mass and inertia of the ship with added masses: \(X_{\dot{u}}\), \(Y_{\dot{v}}\), \(Y_{\dot{r}}\), \(N_{\dot{v}}\), and \(N_{\dot{r}}\) are known; the forces in the Prime system can be calculated using \autoref{equation:03.01_inverse_dynamics:eqxd}, \autoref{equation:03.01_inverse_dynamics:eqyd}, and \autoref{equation:03.01_inverse_dynamics:eqnd}.
These forces can be used to regress the hydrodynamic derivatives through the ordinary least square (OLS) method. If the added masses are unknown, they can be calculated using potential flow methods or semi-empirical methods (\autoref{app:initial_estimates}). 
\begin{equation}\label{equation:03.01_inverse_dynamics:eqxd}
\begin{split}\displaystyle \operatorname{X_{D}'}{\left(u',v',r',\delta,thrust' \right)} = - X_{\dot{u}}' \dot{u}' + \dot{u}' m' - m' r'^{2} x_{G}' - m' r' v'\end{split}
\end{equation}\begin{equation}\label{equation:03.01_inverse_dynamics:eqyd}
\begin{split}\displaystyle \operatorname{Y_{D}'}{\left(u',v',r',\delta,thrust' \right)} = - Y_{\dot{r}}' \dot{r}' - Y_{\dot{v}}' \dot{v}' + \dot{r}' m' x_{G}' + \dot{v}' m' + m' r' u'\end{split}
\end{equation}\begin{equation}\label{equation:03.01_inverse_dynamics:eqnd}
\begin{split}\displaystyle \operatorname{N_{D}'}{\left(u',v',r',\delta,thrust' \right)} = I_{z}' \dot{r}' - N_{\dot{r}}' \dot{r}' - N_{\dot{v}}' \dot{v}' + \dot{v}' m' x_{G}' + m' r' u' x_{G}'\end{split}
\end{equation}

An example that includes forces calculated with inverse dynamics from motions in a turning circle test can be seen in \hyperref[\detokenize{03.01_inverse_dynamics:fig-inverse}]{\autoref{\detokenize{03.01_inverse_dynamics:fig-inverse}}}. The forces have been converted to SI units.

\begin{figure}[H]
    \centering
    \includegraphics[width=\textwidth]{kappa/images/1.pdf}
    \caption{Forces and moments calculated with inverse dynamics on data from a turning circle test.}
    \label{\detokenize{03.01_inverse_dynamics:fig-inverse}}
\end{figure}