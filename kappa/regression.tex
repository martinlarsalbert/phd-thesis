\subsection{Inverse dynamics regression} \label{sec:IDR}
Finding the hydrodynamic derivatives can be defined as a linear regression problem following the ''derivation approach'' (see \autoref{sec:derivation_approach}):
\begin{equation}\label{equation:03.01_inverse_dynamics:eqregression}
\begin{split}y = X\gamma + \epsilon\end{split}
\end{equation}

\noindent The label vector \(y\) and feature matrix \(X\) in the regression problem in \autoref{equation:03.01_inverse_dynamics:eqregression} can be calculated if model for the hydrodynamic forces is assumed. For example: the label in the regression of the surge degree of freedom for the MAVMM can be calculated using the inverse dynamics force, which is expressed with primed units:
\begin{equation}\label{equation:03.01_inverse_dynamics:diff_eq_X_y}
\begin{split}\displaystyle y = - X_{\dot{u}} \dot{u}' + \dot{u}' m' - m' r'^{2} x_{G'} - m' r' v'\end{split}
\end{equation}

\noindent The feature matrix \(X\) is expressed as:
\begin{equation}\label{equation:03.01_inverse_dynamics:diff_eq_X_X}
\begin{split}\displaystyle X = \left[\begin{matrix}thrust' & u' & \delta^{2} & r'^{2} & u'^{2} & r' v'\end{matrix}\right]\end{split}
\end{equation}

\noindent The hydrodynamic derivatives in the \(\gamma\) vector (\autoref{equation:03.01_inverse_dynamics:diff_eq_X_beta}) can be estimated with ordinary least squares (OLS) regression.
\begin{equation}\label{equation:03.01_inverse_dynamics:diff_eq_X_beta}
\begin{split}\displaystyle \gamma = \left[\begin{matrix}X_{T}\\X_{u}\\X_{\delta\delta}\\X_{rr}\\X_{uu}\\X_{vr}\end{matrix}\right]\end{split}
\end{equation}
In this regression, the hydrodynamic derivatives are treated as Gaussian random variables. The hydrodynamic derivatives in the manoeuvring model are usually estimated as the mean value of each regressed random variable, which is the most likely estimate. The regression result can be expressed with a multivariate Gaussian distribution, which is defined by the regression’s mean values and covariance matrix. The multivariate Gaussian distribution can be used to conduct Monte Carlo simulations in the study of alternative realizations of the regression.

Strong multicollinearity is a documented problem for the manoeuvring models \cite{luo_parameter_2016, wang_quantifying_2018}.
The thrust coefficient \(X_T\) in the hydrodynamic function \(X_D\) in \autoref{equation:02.01_manoeuvring models:eqxabkowitz} introduces multicollinearity to the regression. This coefficient can instead be calculated from the thrust deduction factor \(t_{df}\):
\begin{equation}\label{equation:03.01_inverse_dynamics:eqXthrust}
\begin{split}\displaystyle X_{T} = 1 - t_{df}\end{split}
\end{equation}

\noindent The \(X_T\) coefficient is excluded from the regression by moving it to the left-hand side of the regression equation \autoref{equation:03.01_inverse_dynamics:eqregression}:
\begin{equation}\label{equation:03.01_inverse_dynamics:eqexclude}
\begin{split}y-X_T \cdot thrust = X \gamma + \epsilon\end{split}
\end{equation}

\noindent Rudder coefficients (\(Y_R\)) from \(Y_D\) equation \autoref{equation:02.01_manoeuvring models:eqyabkowitz}, such as \(Y_{\delta}\) and \(Y_{\delta T}\), have also been excluded by assuming a connection with their \(N_D\) equation counterpart through the rudder lever arm \(x_r\):
\begin{equation}\label{equation:03.01_inverse_dynamics:eqyr}
\begin{split}\displaystyle Y_{R} = \frac{N_{R}}{x_{r'}}\end{split}
\end{equation}