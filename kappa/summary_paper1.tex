%Section Comments: "Second generation intact stability criteria" seems more commonly used. It would help to specify a specific time period instead of saying "a long time."
\section{Summary of Paper \ref{pap:rolldamping}}
\subsection*{"\nameref{pap:rolldamping}"}
\subsection*{Scope and motivations}
The initial step in this research project was to simplify the system identification of the ship dynamics to a single degree of freedom, specifically the roll motion. However, this is still a very important subject where accurate modeling of roll motion is crucial, as \textcite{franceInvestigationHeadSeaParametric2001} demonstrated in their investigation of the APL China casualty in 1998. In this incident, a post-Panamax C11 class container ship lost nearly a third of its containers, most likely due to head sea parametric rolling.

\subsection*{Results and concluding remarks}
The objective of Paper \ref{pap:rolldamping} was to develop parameter identification techniques for roll motion models derived from roll decay model tests. Additionally, it aimed to propose a parametric model structure for roll motion dynamics that generalizes well, based on prior knowledge from these model tests.

The roll damping was studied using time series data from 250 roll decay tests assembled by RISE at the SSPA Maritime Center Maritime Dynamics Laboratory.
System identification was conducted on each of these time series with the linear, quadratic, and cubic models. Simulation result with the identified models for one of the 250 roll-decay tests is compared to the corresponding experimental results in \autoref{fig:roll_decay_compare}. The cubic and quadratic models reproduced the model test well, for this roll decay test, but the linear model was too simple to provide an accurate representation for both smaller and larger roll angles.

A more detailed analysis can be performed by looking at the amplitude decrement $\phi_a$ and roll damping $B$ for each oscillation as shown in \autoref{fig:roll_decay}. \autoref{fig:roll_decay_damping} shows that none of the models had a perfect fit to the damping in this particular example, which seems to be caused by difficulties in determining the damping for smaller amplitudes, where a high scatter can be observed.  
However, a more rational way to assess the goodness of fit was adopted for all 250 roll decay tests, where the coefficient of determination $R^2$ was calculated for each fit as:  
\begin{equation} \label{eq:R2} 
R^2=1-\frac{SS_{res}}{SS_{tot}} R^2=1-\frac{\sum_{i=1}^{n}(\phi_{i}-\hat{\phi}i)^2}{\sum_{i=1}^{n}(\phi_i-\bar \phi)^2} 
\end{equation} 
where $\phi_i$ is the model test roll angle at time step $i$, $\bar \phi$ is the mean roll angle from the model test, and $\hat{\phi}_i$ is the predicted roll angle (with the linear, quadratic, or cubic model). 
The average $R^2$ of all tests was calculated for each model, giving: 0.995 for the cubic model, 0.993 for the quadratic model, and 0.986 for the linear model. These values indicate that the quadratic model is almost as accurate as the cubic model for describing the roll motion. The quadratic model, with fewer parameters than the cubic model, is expected to have a higher level of generalization at the same accuracy and is therefore proposed as the best mathematical model for the roll motion.
\begin{figure}[h!] \centering \includegraphics[width=\linewidth]{kappa/images/roll_decay_model_compare.pdf} \caption{Roll decay estimation with identified cubic, quadratic, and linear models.} \label{fig:roll_decay_compare} \end{figure}
\begin{figure}[h!] \begin{subfigure}[b]{0.45\textwidth} \centering \includegraphics[width=0.9\linewidth]{kappa/images/roll_decay_amplitude.pdf} \caption{Amplitude decrements.} \label{fig:roll_decay_amplitude} \end{subfigure} ~ %add desired spacing between images, e.g., ~, \quad, \qquad, \hfill etc. %(or a blank line to force the subfigure onto a new line) 
\begin{subfigure}[b]{0.45\textwidth} \centering \includegraphics[width=0.9\linewidth]{kappa/images/roll_decay_damping.pdf} \caption{Dampings.} \label{fig:roll_decay_damping} \end{subfigure} \caption{Roll decay model test, linear-, quadratic-, and cubic-model.} \label{fig:roll_decay} \end{figure}
