%Section Comments: "Second generation intact stability criteria" seems more commonly used. It would help to specify a specific time period instead of saying "a long time."
\section{Summary of Paper \ref{pap:rolldamping}}
\subsection*{"\nameref{pap:rolldamping}"}
\subsection*{Scope and motivations}
The initial step in this research project was to simplify the system identification of the ship dynamics to a single degree of freedom, specifically the roll motion. Accurate modeling of the roll motion is crucial, as \textcite{franceInvestigationHeadSeaParametric2001} demonstrated in their investigation of the APL China casualty in 1998. In this incident, a post-Panamax C11 class container ship lost nearly a third of its containers, most likely due to head sea parametric rolling.

\subsection*{Results and concluding remarks}
The objective of Paper \ref{pap:rolldamping} was to develop parameter identification techniques for roll motion models derived from roll decay model tests. Additionally, it aimed to propose a parametric model structure for roll motion dynamics that generalizes well, based on prior knowledge from these model tests.

The roll damping was studied using time series data from 250 roll decay tests (see \autoref{sec:roll}) assembled by RISE at the Maritime Dynamics Laboratory, SSPA Maritime Center.

System identification was conducted on linear, quadratic, and cubic models. Results from the simulations with the identified models (from one of the roll-decay tests) are presented in \autoref{fig:roll_decay_compare}. The cubic and quadratic models reproduced the model test well, but the linear model was too simple to provide an accurate representation for both smaller and larger roll angles. 
The amplitude decrement $\phi_a$ and roll damping $B$ for each oscillation can be visualized, as seen in \autoref{fig:roll_decay}.
\begin{figure}[h!] \centering \includegraphics[width=\linewidth]{kappa/images/roll_decay_model_compare.pdf} \caption{Roll decay estimation with identified cubic, quadratic, and linear models.} \label{fig:roll_decay_compare} \end{figure}
\begin{figure}[h!] \begin{subfigure}[b]{0.45\textwidth} \centering \includegraphics[width=0.9\linewidth]{kappa/images/roll_decay_amplitude.pdf} \caption{Amplitude decrements.} \label{fig:roll_decay_amplitude} \end{subfigure} ~ %add desired spacing between images, e.g., ~, \quad, \qquad, \hfill etc. %(or a blank line to force the subfigure onto a new line) 
\begin{subfigure}[b]{0.45\textwidth} \centering \includegraphics[width=0.9\linewidth]{kappa/images/roll_decay_damping.pdf} \caption{Dampings.} \label{fig:roll_decay_damping} \end{subfigure} \caption{Roll decay model test, linear-, quadratic-, and cubic-model.} \label{fig:roll_decay} \end{figure}

The goodness of fit for the linear, quadratic, and cubic models was expressed using the coefficient of determination: 
\begin{equation} \label{eq:R2} 
R^2=1-\frac{SS_{res}}{SS_{tot}} R^2=1-\frac{\sum_{i=1}^{n}(\phi_{i}-\hat{\phi}i)^2}{\sum_{i=1}^{n}(\phi_i-\bar \phi)^2} 
\end{equation} 
where $\phi_i$ is the model test roll angle at time step $i$, $\bar \phi$ is the mean roll angle from the model test, and $\hat{\phi}_i$ is the predicted roll angle (with the linear, quadratic, or cubic model). The average goodness of fit $R^2$ was 0.995 for the cubic model, 0.993 for the quadratic model, and 0.986 for the linear model. These values indicate that the quadratic model is almost as useful as the cubic model for describing the roll motion. The quadratic model, with fewer parameters than the cubic model, is expected to have a higher level of generalization at the same accuracy and is therefore selected as the best mathematical model for the roll motion.