\section{Manoeuvring model parameter estimation} \label{sec:_VMM}
A new parameter estimation method is proposed in Paper \ref{pap:pit} for the remaining degrees of freedom. A manoeuvring model is used to solve the reversed manoeuvring problem. The problem may consist of predicting unknown forces from known manoeuvring model test data. The hydrodynamic derivatives in the manoeuvring model can be identified through regression of the force polynomials on forces predicted with inverse dynamics (see \autoref{\detokenize{03.01_inverse_dynamics::doc}}).
The measurement noise must be removed prior to the regression of hydrodynamic derivatives in the manoeuvring model. This is conducted by an extended Kalman filter (EKF) and a Rauch Tung Striebel (RTS) smoother (see \autoref{sec:datacleaning}). The EKF requires an accurate manoeuvring model as the predictor.
Therefore, the accurate manoeuvring model is both the input and output of the method. As a solution to this dilemma, a linear manoeuvring model that includes hydrodynamic derivatives estimated with semi-empirical formulas (\autoref{app:initial_estimates}) is used as the initial predictor. Once the regressed manoeuvring model has been obtained, the parameter estimation can be refined, using the regressed manoeuvring model as the predictor model in the EKF, to improve the filter and obtain a more accurate manoeuvring model. The method is summarized in \autoref{fig:greyvmm} and can be repeated several times (indicated by the dashed arrow) for improved accuracy. 
\begin{figure}[h]
    
    \centering
    \begin{tikzpicture}[node distance=1.5cm]
    %\draw (0,0) rectangle (10,10); %create a bounding box to reserve space
    \node (data) [io] {\footnotesize Model test data: $x$, $\delta$, thrust};
    
    \node (EKF) [process, right of=data, xshift=3.0cm] {\footnotesize EFK + RTS};
    \node (predictor) [process, right of=EKF, xshift=1.5cm]{\footnotesize Predictor};
    \node (VMM) [io, right of=predictor, xshift=1.0cm] {\footnotesize initial model};
    
    \node (data_clean) [io, below of=EKF] {\footnotesize \(x,\dot{x},\ddot{x}, \delta, thrust\)};
    
    \node (black-box) [black-box, below of=data_clean] {\footnotesize Regression};
    
    \node (X_D) [io, left of=black-box, xshift=-0.70cm, yshift=0.7cm]{\footnotesize \(X_D\)};
    \node (Y_D) [io, left of=black-box, xshift=-0.70cm, yshift=0cm]{\footnotesize \(Y_D\)};
    \node (N_D) [io, left of=black-box, xshift=-0.70cm, yshift=-0.7cm]{\footnotesize \(N_D\)};
    
    \node (white-box) [white-box, left of=Y_D, xshift=-1.00cm] {\footnotesize Inverse dynamics};
    
    
    %
    %
    \node (coefficients) [io, right of=black-box, xshift=1.5cm] {\footnotesize model$\left(Y_{uv},N_{\delta},...\right)$};
    
    \draw [arrow] (data) -- (EKF);
    \draw [arrow] (predictor) -- (EKF);
    \draw [arrow] (VMM) -- (predictor);
    \draw [arrow] (EKF) -- (data_clean);
    
    \draw [arrow] (data_clean) -| (white-box);
    \draw [arrow] (data_clean) -- (black-box);
    
    \draw [arrow] (white-box) -- (X_D);
    \draw [arrow] (white-box) -- (Y_D);
    \draw [arrow] (white-box) -- (N_D);
    
    \draw [arrow, shorten >=0.5cm] (X_D) -- (black-box);
    \draw [arrow, shorten >=0.2cm] (Y_D)  -- (black-box);
    \draw [arrow, shorten >=0.5cm] (N_D)  -- (black-box);
    
    
    \draw [arrow] (black-box)  -- (coefficients);
    \draw [arrow, dashed] (coefficients)  -- (predictor);
    
    \end{tikzpicture}
    \caption{Method to estimate the manoeuvring model hydrodynamic derivatives.}
    \label{fig:greyvmm}
\end{figure}

\noindent Using semi-empirical formulas (\autoref{app:initial_estimates}) for the initially estimated manoeuvring model adds prior knowledge about the ship dynamics to the regression. An example, with simulation results from the steps in the iteration, is presented in \hyperref[\detokenize{01.01_method:iterations}]{\autoref{\detokenize{01.01_method:iterations}}}.


\begin{figure}[H]
    \centering
    \includegraphics[width=\textwidth]{kappa/images/0.pdf}
    \caption{Simulation with: initial model and first and second iteration of the parameter estimation method.}
    \label{\detokenize{01.01_method:iterations}}
\end{figure}

\subsection{Inverse dynamics and regression}
\label{\detokenize{03.01_inverse_dynamics:inverse-dynamics-and-regression}}\label{\detokenize{03.01_inverse_dynamics::doc}}

Each manoeuvring model has some hydrodynamic functions \(X_D(u,v,r,\delta,thrust)\), \(Y_D(u,v,r,\delta,thrust)\), \(N_D(u,v,r,\delta,thrust)\) that are defined as polynomials. The hydrodynamic derivatives in these polynomials can be identified with force regression of measured forces and moments. The measured forces and moments are usually taken from captive model tests (CMT), planar motion mechanism (PMM) tests, or virtual captive tests (VCT). However, motions are recorded when the ship is free in all degrees of freedom. Hence, forces and moments causing ship motion must be estimated by solving the inverse dynamics problem.
The inverse dynamics problem is solved by restructuring the system equation (\autoref{equation:02.01_manoeuvring models:eqacc}) to get the hydrodynamics functions on the left-hand side. If the mass and inertia of the ship with added masses: \(X_{\dot{u}}\), \(Y_{\dot{v}}\), \(Y_{\dot{r}}\), \(N_{\dot{v}}\), and \(N_{\dot{r}}\) are known; the forces in the Prime system can be calculated using \autoref{equation:03.01_inverse_dynamics:eqxd}, \autoref{equation:03.01_inverse_dynamics:eqyd}, and \autoref{equation:03.01_inverse_dynamics:eqnd}.
These forces can be used to regress the hydrodynamic derivatives through the ordinary least square (OLS) method. If the added masses are unknown, they can be calculated using potential flow methods or semi-empirical methods (\autoref{app:initial_estimates}). 
\begin{equation}\label{equation:03.01_inverse_dynamics:eqxd}
\begin{split}\displaystyle \operatorname{X_{D}'}{\left(u',v',r',\delta,thrust' \right)} = - X_{\dot{u}}' \dot{u}' + \dot{u}' m' - m' r'^{2} x_{G}' - m' r' v'\end{split}
\end{equation}\begin{equation}\label{equation:03.01_inverse_dynamics:eqyd}
\begin{split}\displaystyle \operatorname{Y_{D}'}{\left(u',v',r',\delta,thrust' \right)} = - Y_{\dot{r}}' \dot{r}' - Y_{\dot{v}}' \dot{v}' + \dot{r}' m' x_{G}' + \dot{v}' m' + m' r' u'\end{split}
\end{equation}\begin{equation}\label{equation:03.01_inverse_dynamics:eqnd}
\begin{split}\displaystyle \operatorname{N_{D}'}{\left(u',v',r',\delta,thrust' \right)} = I_{z}' \dot{r}' - N_{\dot{r}}' \dot{r}' - N_{\dot{v}}' \dot{v}' + \dot{v}' m' x_{G}' + m' r' u' x_{G}'\end{split}
\end{equation}

\noindent An example that includes forces calculated with inverse dynamics from motions in a turning circle test can be seen in \hyperref[\detokenize{03.01_inverse_dynamics:fig-inverse}]{\autoref{\detokenize{03.01_inverse_dynamics:fig-inverse}}}. The forces have been converted to SI units.

\begin{figure}[H]
    \centering
    \includegraphics[width=\textwidth]{kappa/images/1.pdf}
    \caption{Forces and moments calculated with inverse dynamics on data from a turning circle test.}
    \label{\detokenize{03.01_inverse_dynamics:fig-inverse}}
\end{figure}