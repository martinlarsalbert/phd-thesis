%"Critic" to what has been done before
\section{Motivation and objective}
\label{sec:motivation}
% Motivation:
System identifications of parametric models has been conducted since the late 1970s from free running tests, and for even longer times from captive tests. The first papers about non-parametric models were published in the late 1990s, with an increasing popularity during the past 15 years, especially within the field of autonomous vessels. Today there are still papers being published about both these approaches, so there seems to be no consensus which one is the better and there are still new findings how to improve the models and also how to combine them in the hybrid models.
Further progress within machine learning can definitely be expected within the coming years, so there is a bright future for the non-parametric models and the hybrid approaches. The lack of informative data and persistence of excitation will however remain a big challenge. One aspect of indirect informative data that is often overlooked is the prior knowledge about ship hydrodynamics from previous experimental works and other physical insights. This indirect informative data are often embedded in the parametric model structures; which parameters should be included or excluded from a model have often been chosen with careful consideration from experimental works or physical reasoning. There are also semi-empirical formulas in the literature that could potentially be used to add more informative data. This is a subject that needs further investigation, and the research question of this thesis has therefore been formulated in the following way: 
% Objective: 
% Objectives --> Conclusions!
% Objective + in order to...
%\vspace{0.2cm}

\begin{tabular}{p{0.93\textwidth}}
    \emph{\researchquestion}
\end{tabular}

To provide a clear path through this research, the research question has been broken down into the following research objectives:

\noindent A prestudy with the initial simplification to only consider one degree of freedom in ship roll motion is first conducted with the following two objectives:
\begin{enumerate}[label=(\Alph*),itemsep=1mm]

    \item Develop parameter identification techniques for roll motion models from FT data.
    
    \item Propose a parametric model structure for roll motion dynamics with good generalization based on prior knowledge from model tests. 

\end{enumerate}

\vspace{0.1 cm}
\noindent The knowledge gained from the prestudy is then expanded to the manoeuvring problem with the following objectives: 
\begin{enumerate}[label=(\Alph*),itemsep=1mm]
    \setcounter{enumi}{2}

    \item Develop parameter identification techniques for ship manoeuvring models from FT data that can generalize from simpler to more complicated maneuvers.

    \item Propose a parametric model structure with good generalization that is identifiable from standard maneuvers. The model structure should be based on physical insights from CFD and FRMT inverse dynamics.

    \item Mitigate multicollinearity and enhance generalization by introducing semi-empirical formulas.
    
\end{enumerate}
