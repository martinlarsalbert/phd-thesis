%"Critic" to what has been done before
\section{Motivation and objective}
\label{sec:motivation}
% Motivation:
System identification of parametric models has been conducted since the late 1970s using free-running tests, and even longer with captive tests. The first papers on non-parametric models were published in the late 1990s, and their popularity has increased over the past 15 years, particularly in the field of autonomous vessels. Today, research continues to be published on both approaches, indicating no clear consensus on which is superior. New findings continue to emerge on how to improve these models and combine them into hybrid models.

Further advancements in machine learning are expected in the coming years, promising a bright future for non-parametric models and hybrid approaches. However, challenges remain, such as the lack of informative data and persistence of excitation, which are crucial for developing physically accurate models with good generalization capabilities—models that perform well on new and unseen data.

One often overlooked aspect of indirect informative data is the prior knowledge of ship hydrodynamics from previous experimental work and other physical insights. This indirect data is frequently embedded in parametric model structures, where the inclusion or exclusion of parameters is often based on careful consideration of experimental results or physical reasoning. Additionally, semi-empirical formulas in the literature could potentially be used to enhance the informative data. This area requires further investigation, which motivates the research question of this thesis as follows: 
% Objective: 
% Objectives --> Conclusions!
% Objective + in order to...
%\vspace{0.2cm}

\vspace{0.1cm}
\begin{tabular}{p{0.93\textwidth}}
    \emph{\researchquestion}
\end{tabular}
\vspace{0.1cm}

\noindent The research question has been broken down into research objectives in \autoref{tab:objectives} to provide a clear path through this study. The first two objectives (A--B) constitute a prestudy, initially simplifying the problem to consider only one degree of freedom in ship roll motion, which was addressed in Paper \ref{pap:rolldamping}. Objectives C--D expand this work to include identification and parametric models for the three degrees of freedom manoeuvring problem. Parameter identification techniques for CT and FT data were developed in Papers \ref{pap:pit} and \ref{pap:vct}, respectively. Parametric model structures were proposed in Paper \ref{pap:pit}, focusing on generalizing from simpler to more complex manoeuvres. The ability of these models to generalize to wind conditions was studied in Paper \ref{pap:physics}. Further work to develop physically accurate models was carried out in Paper \ref{pap:vct}, involving extensive VCT calculations and FT inverse dynamics. Semi-empirical formulas were introduced in Paper \ref{pap:ikeda} for roll damping and in Paper \ref{pap:physics} for rudder forces. 
%
%\vspace{0.2cm}
%\begin{enumerate}[label=(\Alph*),itemsep=1mm]
%
%    \item Develop parameter identification techniques for roll motion models from FT data.
%    
%    \item Propose a parametric model structure for roll motion dynamics with good generalization based on prior knowledge from model tests. 
%    
%    \item Developing parameter identification techniques for ship manoeuvring models from FT data that can generalize from simpler to more complicated maneuvers.
%
%    \item Propose a parametric model structure with good generalization that is identifiable from standard maneuvers. The model structure should be based on physical insights from CFD and FT inverse dynamics.
%
%    \item Introducing semi-empirical formulas to mitigate multicollinearity and enhance generalization.
%    
%\end{enumerate}
\begin{table}[h]
    \centering
    \caption{Research objectives of this thesis A--E including sub objectives 1--3 and the appended papers 1--5 where the objectives are mainly addressed.}
    \label{tab:objectives}
    
    \begin{tabular}{|w{c}{0.30cm}w{c}{0.01cm}p{9cm}|w{c}{0.01cm}|w{c}{0.01cm}|w{c}{0.01cm}|w{c}{0.01cm}|w{c}{0.01cm}|}
     \hline
     ~ & ~ & Objective & 1 & 2 & 3 & 4 & 5 \\
     \hline
     A & ~ & Developing parameter identification techniques for roll motion models from FT data. & \checkmark & ~ & ~ & ~ & ~ \\
     
     \hline
     B & ~ &Proposing a parametric model structure for roll motion dynamics with good generalization based on prior knowledge from model tests. & \checkmark & ~ & ~ & ~ & ~ \\

     \hline
     C & ~ &Developing parameter identification techniques for ship manoeuvring models from: & ~ & ~ & ~ & ~ & ~ \\
     ~ & 1 & \hspace{0.25cm} FT data. & ~ & ~ & \checkmark & ~ & ~ \\
     ~ & 2 & \hspace{0.25cm} CT data. & ~ & ~ & ~ & ~ & \checkmark \\
     
     \hline
     D & ~ &Proposing a parametric model structure with good generalization that is identifiable from standard maneuvers. & ~ & ~ & ~ & ~ & ~ \\
     ~ & 1 & \hspace{0.25cm} Generalize from simpler to more complicated manoeuvres. & ~ & ~ & \checkmark & ~ & ~ \\
     ~ & 2 & \hspace{0.25cm} Generalize to wind conditions. & ~ & ~ & ~ & \checkmark & ~ \\
     ~ & 3 & \hspace{0.25cm} Physical insights from CFD and FT inverse dynamics. & ~ & ~ & ~ & ~ & \checkmark \\

     \hline
     E & ~ & Introducing semi-empirical formulas to mitigate multicollinearity and enhance generalization. & ~ & \checkmark & ~ & \checkmark & ~ \\
     
     
     \hline
    \end{tabular}

\end{table}