%"Critic" to what has been done before
\section{Motivation and objective}
\label{sec:motivation}
% Motivation:
System identifications of parametric models has been conducted since the late 1970s from free running tests, and for even longer times from captive tests. The first papers about non-parametric models were published in the late 1990s, with an increasing popularity during the past 15 years, especially within the field of autonomous vessels. Today there are still papers being published about both these approaches, so there seems to be no consensus which one is the better and there are still new findings how to improve the models and also how to combine them in the hybrid models.
Further progress within machine learning can be expected within the coming years, with a bright future for the non-parametric models and the hybrid approaches. However, the lack of informative data and persistence of excitation will remain a big challenge to find physically correct models with good generalization -- models that can perform well on new and unseen data.
One aspect of indirect informative data that is often overlooked is the prior knowledge about ship hydrodynamics from previous experimental works and other physical insights. This indirect informative data is often embedded in the parametric model structures; which parameters should be included or excluded from a model have often been chosen with careful consideration from experimental works or physical reasoning. There are also semi-empirical formulas in the literature that could potentially be used to add more informative data. This is a subject that needs further investigation, and therefore the research question of this thesis has been formulated in the following way: 
% Objective: 
% Objectives --> Conclusions!
% Objective + in order to...
%\vspace{0.2cm}

\vspace{0.1cm}
\begin{tabular}{p{0.93\textwidth}}
    \emph{\researchquestion}
\end{tabular}
\vspace{0.1cm}

\noindent The research question has been broken down into research objectives in \autoref{tab:objectives} to provide a clear path through this study. The first two objectives (A--B) constitute a prestudy, initially simplifying the problem to consider only one degree of freedom in ship roll motion, which was addressed in Paper \ref{pap:rolldamping}. Objectives C--D expand this work to include identification and parametric models for the three degrees of freedom manoeuvring problem. Parameter identification techniques for VCT and FT data were developed in Papers \ref{pap:pit} and \ref{pap:vct}, respectively. Parametric model structures were proposed in Paper \ref{pap:pit}, focusing on generalizing from simpler to more complex manoeuvres. The ability of these models to generalize to wind conditions was studied in Paper \ref{pap:physics}. Further work to develop physically accurate models was carried out in Paper \ref{pap:vct}, involving extensive VCT calculations and FT inverse dynamics. Semi-empirical formulas were introduced in Paper \ref{pap:ikeda} for roll damping and in Paper \ref{pap:physics} for rudder forces. 
%
%\vspace{0.2cm}
%\begin{enumerate}[label=(\Alph*),itemsep=1mm]
%
%    \item Develop parameter identification techniques for roll motion models from FT data.
%    
%    \item Propose a parametric model structure for roll motion dynamics with good generalization based on prior knowledge from model tests. 
%    
%    \item Developing parameter identification techniques for ship manoeuvring models from FT data that can generalize from simpler to more complicated maneuvers.
%
%    \item Propose a parametric model structure with good generalization that is identifiable from standard maneuvers. The model structure should be based on physical insights from CFD and FT inverse dynamics.
%
%    \item Introducing semi-empirical formulas to mitigate multicollinearity and enhance generalization.
%    
%\end{enumerate}
\begin{table}[h]
    \centering
    \caption{Research objectives of this thesis A--E including sub objectives 1--3 and the appended papers 1--5 where the objectives are mainly addressed.}
    \label{tab:objectives}
    
    \begin{tabular}{|w{c}{0.30cm}w{c}{0.01cm}p{9cm}|w{c}{0.01cm}|w{c}{0.01cm}|w{c}{0.01cm}|w{c}{0.01cm}|w{c}{0.01cm}|}
     \hline
     ~ & ~ & Objective & 1 & 2 & 3 & 4 & 5 \\
     \hline
     A & ~ & Developing parameter identification techniques for roll motion models from FT data. & \checkmark & ~ & ~ & ~ & ~ \\
     
     \hline
     B & ~ &Proposing a parametric model structure for roll motion dynamics with good generalization based on prior knowledge from model tests. & \checkmark & ~ & ~ & ~ & ~ \\

     \hline
     C & ~ &Developing parameter identification techniques for ship manoeuvring models from: & ~ & ~ & ~ & ~ & ~ \\
     ~ & 1 & \hspace{0.25cm} FT data. & ~ & ~ & \checkmark & ~ & ~ \\
     ~ & 2 & \hspace{0.25cm} CT data. & ~ & ~ & ~ & ~ & \checkmark \\
     
     \hline
     D & ~ &Propose a parametric model structure with good generalization that is identifiable from standard maneuvers. & ~ & ~ & ~ & ~ & ~ \\
     ~ & 1 & \hspace{0.25cm} Generalize from simpler to more complicated manoeuvres. & ~ & ~ & \checkmark & ~ & ~ \\
     ~ & 2 & \hspace{0.25cm} Generalize to wind conditions. & ~ & ~ & ~ & \checkmark & ~ \\
     ~ & 3 & \hspace{0.25cm} Physical insights from CFD and FT inverse dynamics. & ~ & ~ & ~ & ~ & \checkmark \\

     \hline
     E & ~ & Introducing semi-empirical formulas to mitigate multicollinearity and enhance generalization. & ~ & \checkmark & ~ & \checkmark & ~ \\
     
     
     \hline
    \end{tabular}

\end{table}