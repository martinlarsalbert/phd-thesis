Ship digital twin (SDT) has a positive trend in the number of publications in recent years (2018-2021). Most of the papers concern ship equipment such as electric power systems, propulsion system, ship hull structure, and marine diesel engines. A small minority of the SDT applications handle ship trajectory, speed, and fuel consumption \cite{assani_ships_2022}.   
Even though SDT is not explicitly mentioned, there are many publications about methods that can be used as SDTs. \textcite{lang_comparison_2022} predicted the propulsion power for a chemical tanker for three test case voyages by using ML black-box modeling. However, the manoeuvres were excluded. \textcite{nielsen_machine_2022} used grey-box modelling for the manoeuvring prediction of a ferry, where a deep learning model (black-box) captures the residues between a first-principles model (white-box) and observed data. These studies demonstrate the vast potential within the field.

Noteworthy publications within the system identification of the ship's manoeuvring dynamics are summarized in \autoref{tab:references} and categorized as black-box or grey-box models.
\input{kappa/references_table} 
The system identification can be applied to full scale data \cite{astrom_identification_1976,revestido_herrero_two-step_2012,perera_system_2015}, which has the highest model uncertainty and measurement uncertainty. Therefore, it is the hardest task but also the most relevant. A method for reducing the uncertainty is using model test data \cite{araki_estimating_2012,luo_parameter_2016,xue_identification_2021,miller_ship_2021, he_nonparametric_2022}. The uncertainty can be further reduced by using simulated data \cite{shi_identification_2009,zhu_parameter_2017,wang_parameter_2021}, which can demonstrate the potential of new methods that have the benefit of the true model being known. One must however be consistent with the main objective of identifying real objects, not only mathematical models \cite{miller_ship_2021}.

Black-box modeling was used in \textcite{he_nonparametric_2022}, using a neural network, and in \textcite{xue_identification_2021}, using a Gaussian process. The nonparametric models are related because the system structure is known but no parameters are required; this is seen in \textcite{pongduang_nonparametric_2020}. However, most of the system identification methods for ship manoeuvring models use grey-box modeling by assuming a predefined mathematical model, which reduces the problem to a parameter estimation.
The Kalman filter (KF) combined with maximum likelihood estimation was proposed in 1976 by \textcite{astrom_identification_1976} to develop a linear manoeuvring model that utilized manually recorded data in 1969 aboard the Atlantic Song freighter. The extended Kalman filter (EKF) can also estimate parameters if the parameters are represented as states of the state space model. This technique was used on a nonlinear Nomoto model \cite{perera_system_2015} and a 3 degree of freedom model (3DOF) \cite{shi_identification_2009}. The EKF was used in \textcite{araki_estimating_2012}, with constrained parameters based on physical reasoning and prior knowledge from constrained least squares regression. The unscented Kalman filter (UKF), which has been proposed as an improvement to the EKF for handling nonlinear systems, was used in \textcite{revestido_herrero_two-step_2012}.
Support vector regression (SVR) has also been investigated by \textcite{luo_parameter_2016}, \textcite{zhu_parameter_2017}, and \textcite{wang_parameter_2021}. A genetic algorithm was used by \textcite{miller_ship_2021} for the system identification of a model test performed on a lake.