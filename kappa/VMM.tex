\section{Vessel manoeuvring models} \label{sec:manoeuvring model}
\label{\detokenize{02.01_manoeuvring models:vessel-manoeuvring-models}}\label{\detokenize{02.01_manoeuvring models:vmm}}\label{\detokenize{02.01_manoeuvring models::doc}}
\vspace{-0.2cm}
Ship manoeuvring is a simplified case of seakeeping. The encountering waves have been removed, assuming calm water conditions. The manoeuvring motions have low frequencies so that added masses and other hydrodynamic derivatives can be assumed as constants  \cite{fossen_handbook_2011}. Three manoeuvring models are used in this thesis: 
\begin{itemize}
    \item Linear (LVMM) \cite{matusiak_dynamics_2021}
    \item Abkowitz (AVMM) \cite{abkowitz_ship_1964}
    \item Modified Abkowitz (MAVMM), which is proposed in Paper \ref{pap:pit}
\end{itemize}

\noindent\autoref{\detokenize{02.01_manoeuvring models:coordinate-system}} shows the reference frames used in the manoeuvring models where \(x_0\) and \(y_0\) and heading \(\Psi\) are the global position and orientation of a ship fix reference frame \(O(x,y,z)\) (or rather \(O(x,y)\) when heave is excluded) with origin at midship. \(u\), \(v\), \(r\), \(X\), \(Y\) and \(N\) are velocities and forces in the ship fix reference frame.



\begin{figure}[H]
    \centering
    \includegraphics[width=0.8\textwidth]{kappa/images/coordinate_system.PNG}
    \caption{Reference frames.}
    \label{\detokenize{02.01_manoeuvring models:coordinate-system}}
\end{figure}

The manoeuvring equation can be described as \cite{fossen_handbook_2011},
\begin{equation}\label{equation:02.01_manoeuvring models:eqqsystem}
\begin{split}\displaystyle \left[\begin{matrix}- X_{\dot{u}} + m & 0 & 0\\0 & - Y_{\dot{v}} + m & - Y_{\dot{r}} + m x_{G}\\0 & - N_{\dot{v}} + m x_{G} & I_{z} - N_{\dot{r}}\end{matrix}\right] \left[\begin{matrix}\dot{u}\\\dot{v}\\\dot{r}\end{matrix}\right] = \left[\begin{matrix}m r^{2} x_{G} + m r v + \operatorname{X_{D}}{\left(u,v,r,\delta,thrust \right)}\\- m r u + \operatorname{Y_{D}}{\left(u,v,r,\delta,thrust \right)}\\- m r u x_{G} + \operatorname{N_{D}}{\left(u,v,r,\delta,thrust \right)}\end{matrix}\right]\end{split}
\end{equation}

\noindent where the first matrix describes the inertia of the ship in the surge, sway and yaw directions. The inertia in air is represented by the mass $m$ and moment of inertia $I_z$. The added mass in water is represented by: $X_{\dot{u}}$, $Y_{\dot{v}}$, $Y_{\dot{r}}$, $N_{\dot{v}}$ and $N_{\dot{r}}$. The hydrodynamic forces from the ship hull and rudder are desribed in the functions $X_D()$, $Y_D()$ and $N_D()$. The accelerations ($\dot{u}$, $\dot{v}$ and $\dot{r}$) can be solved from this equation,
\begin{equation}\label{equation:02.01_manoeuvring models:eqacc}
\begin{split}\displaystyle \dot{\nu} = \left[\begin{matrix}\dot{u}\\\dot{v}\\\dot{r}\end{matrix}\right] = \left[\begin{matrix}\frac{1}{- X_{\dot{u}} + m} & 0 & 0\\0 & - \frac{- I_{z} + N_{\dot{r}}}{S} & - \frac{- Y_{\dot{r}} + m x_{G}}{S}\\0 & - \frac{- N_{\dot{v}} + m x_{G}}{S} & - \frac{Y_{\dot{v}} - m}{S}\end{matrix}\right] \left[\begin{matrix}m r^{2} x_{G} + m r v + \operatorname{X_{D}}{\left(u,v,r,\delta,thrust \right)}\\- m r u + \operatorname{Y_{D}}{\left(u,v,r,\delta,thrust \right)}\\- m r u x_{G} + \operatorname{N_{D}}{\left(u,v,r,\delta,thrust \right)}\end{matrix}\right]\end{split}
\end{equation}

where \(S\) is a helper variable:
\begin{equation}\label{equation:02.01_manoeuvring models:eq_S}
\begin{split}\displaystyle S = - I_{z} Y_{\dot{v}} + I_{z} m + N_{\dot{r}} Y_{\dot{v}} - N_{\dot{r}} m - N_{\dot{v}} Y_{\dot{r}} + N_{\dot{v}} m x_{G} + Y_{\dot{r}} m x_{G} - m^{2} x_{G}^{2}\end{split}
\end{equation}


\noindent A state space model for manoeuvring can now be defined with six states:
\begin{equation}\label{equation:02.01_manoeuvring models:eq_x}
\begin{split}\displaystyle \mathbf{x} = \left[\begin{matrix}x_{0}\\y_{0}\\\Psi\\u\\v\\r\end{matrix}\right]\end{split}
\end{equation}
\noindent where $x_0$, $y_0$ and $\Psi$ are the global coordinates and heading of the ship and $u$, $v$ and $r$ are the velocities as seen in \autoref{\detokenize{02.01_manoeuvring models:coordinate-system}}.
The time derivative of this state \(\dot{\mathbf{x}}\) can be defined by a state transition \(f(\mathbf{x},\mathbf{c})\) using geometrical relations
how global coordinates \(x_0\), \(y_0\) and \(\Psi\) depend on \(u\), \(v\), and \(r\) viz.,
\begin{equation}\label{equation:02.01_manoeuvring models:eqf}
\begin{split}\displaystyle \dot{\mathbf{x}} = f(\mathbf{x},\mathbf{c}) + \mathbf{w}
                                          = \left[\begin{matrix}\dot{x_0}\\ \dot{y_0} \\ \dot{\Psi} \\\dot{u}\\\dot{v}\\\dot{r}\end{matrix}\right] + \mathbf{w}
                                          = \left[\begin{matrix}u \cos{\left(\Psi \right)} - v \sin{\left(\Psi \right)}\\u \sin{\left(\Psi \right)} + v \cos{\left(\Psi \right)}\\r\\\dot{u}\\\dot{v}\\\dot{r}\end{matrix}\right] + \mathbf{w}\end{split}
\end{equation}

where \(\mathbf{c}\) is control inputs (rudder angle \(\delta\) and thrust); the last three derivatives: \(\dot{u}\), \(\dot{v}\), \(\dot{r}\) are calculated with \autoref{equation:02.01_manoeuvring models:eqacc}.
\(\mathbf{w}\) is the process noise, i.e., the difference between the predicted state by the manoeuvring model and the true
state of the system. \(\mathbf{w}\) is unknown when the manoeuvring model is used for manoeuvre predictions and therefore normally
assumed to be zero, but it is an important factor when the manoeuvring model is used in the EKF (see \autoref{sec:datacleaning}).
The manoeuvring simulation can now be conducted by numerical integration of \autoref{equation:02.01_manoeuvring models:eqf}. The main difference between the manoeuvring model:s lies in how the hydrodynamic functions \(X_D(u,v,r,\delta,thrust)\), \(Y_D(u,v,r,\delta,thrust)\), \(N_D(u,v,r,\delta,thrust)\) are defined. These expressions are denoted in prime system ($X_D'$, $Y_D'$, $N_D'$) below for the various manoeuvring models: LVMM, AVMM and MAVMM.

\subsubsection*{{\normalfont \textbf{Linear vessel manoeuvring model (LVMM) \cite{matusiak_dynamics_2021}}}}
\begin{equation}\label{equation:02.01_manoeuvring models:eqxlinear}
\begin{split}\begin{split}
\operatorname{X_{D}'}{\left(u',v',r',\delta\right)} = & X_{\delta} \delta + X_{r} r' + X_{u} u' + X_{v} v' 
\end{split}\end{split}
\end{equation}\begin{equation}\label{equation:02.01_manoeuvring models:eqylinear}
\begin{split}\begin{split}
\operatorname{Y_{D}'}{\left(u',v',r',\delta \right)} = & Y_{\delta} \delta + Y_{r} r' + Y_{u} u' + Y_{v} v' 
\end{split}\end{split}
\end{equation}\begin{equation}\label{equation:02.01_manoeuvring models:eqnlinear}
\begin{split}\begin{split}
\operatorname{N_{D}'}{\left(u',v',r',\delta \right)} = & N_{\delta} \delta + N_{r} r' + N_{u} u' + N_{v} v' 
\end{split}\end{split}
\end{equation}

\subsubsection*{\normalfont \textbf{Abkowitz vessel manoeuvring model (AVMM) \cite{abkowitz_ship_1964}}}
\begin{equation}\label{equation:02.01_manoeuvring models:eqxabkowitz}
\begin{split}
\operatorname{X_{D}'}{\left(u',v',r',\delta,thrust' \right)} = & X_{\delta\delta} \delta^{2} + X_{r\delta} \delta r' + X_{rr} r'^{2} + X_{T} thrust' + X_{u\delta\delta} \delta^{2} u' \\ 
& + X_{ur\delta} \delta r' u' + X_{urr} r'^{2} u' + X_{uuu} u'^{3} + X_{uu} u'^{2} \\ 
& + X_{uv\delta} \delta u' v' + X_{uvr} r' u' v' + X_{uvv} u' v'^{2} \\
& + X_{u} u' + X_{v\delta} \delta v' + X_{vr} r' v' + X_{vv} v'^{2} 
\end{split}
\end{equation}

\begin{equation}\label{equation:02.01_manoeuvring models:eqyabkowitz}
\begin{split}\begin{split}
\operatorname{Y_{D}'}{\left(u',v',r',\delta,thrust' \right)} = & Y_{0uu} u'^{2} + Y_{0u} u' + Y_{0} + Y_{\delta\delta\delta} \delta^{3} + Y_{\delta} \delta + Y_{r\delta\delta} \delta^{2} r' + Y_{rr\delta} \delta r'^{2} \\ & + Y_{rrr} r'^{3} + Y_{r} r' + Y_{T\delta} \delta thrust' + Y_{T} thrust' + Y_{u\delta} \delta u' \\ & + Y_{ur} r' u' + Y_{uu\delta} \delta u'^{2} + Y_{uur} r' u'^{2} + Y_{uuv} u'^{2} v' + Y_{uv} u' v' \\ & + Y_{v\delta\delta} \delta^{2} v' + Y_{vr\delta} \delta r' v' + Y_{vrr} r'^{2} v' + Y_{vv\delta} \delta v'^{2} + Y_{vvr} r' v'^{2} \\ & + Y_{vvv} v'^{3} + Y_{v} v' 
\end{split}\end{split}
\end{equation}\begin{equation}\label{equation:02.01_manoeuvring models:eqnabkowitz}
\begin{split}\begin{split}
\operatorname{N_{D}'}{\left(u',v',r',\delta,thrust' \right)} = & N_{0uu} u'^{2} + N_{0u} u' + N_{0} + N_{\delta\delta\delta} \delta^{3} + N_{\delta} \delta + N_{r\delta\delta} \delta^{2} r' + N_{rr\delta} \delta r'^{2} \\ & + N_{rrr} r'^{3} + N_{r} r' + N_{T\delta} \delta thrust' + N_{T} thrust' + N_{u\delta} \delta u' \\ & + N_{ur} r' u' + N_{uu\delta} \delta u'^{2} + N_{uur} r' u'^{2} + N_{uuv} u'^{2} v' + N_{uv} u' v' \\ & + N_{v\delta\delta} \delta^{2} v' + N_{vr\delta} \delta r' v' + N_{vrr} r'^{2} v' + N_{vv\delta} \delta v'^{2} + N_{vvr} r' v'^{2} \\ & + N_{vvv} v'^{3} + N_{v} v' 
\end{split}\end{split}
\end{equation}

\subsubsection*{\normalfont \textbf{Modified Abkowitz vessel manoeuvring model (MAVMM)}}
\newline
Only the most relevant coefficients in AVMM are included, as proposed in Paper \ref{pap:pit}.
\begin{equation}\label{equation:02.01_manoeuvring models:eqxmartinssimple}
\begin{split}\begin{split}
\operatorname{X_{D}'}{\left(u',v',r',\delta,thrust' \right)} = & X_{\delta\delta} \delta^{2} + X_{rr} r'^{2} + X_{T} thrust' + X_{uu} u'^{2} + X_{u} u' + X_{vr} r' v' 
\end{split}\end{split}
\end{equation}\begin{equation}\label{equation:02.01_manoeuvring models:eqymartinssimple}
\begin{split}\begin{split}
\operatorname{Y_{D}'}{\left(u',v',r',\delta,thrust' \right)} = & Y_{\delta} \delta + Y_{r} r' + Y_{T\delta} \delta thrust' + Y_{T} thrust' + Y_{ur} r' u' + Y_{u} u' \\ & + Y_{vv\delta} \delta v'^{2} + Y_{v} v' 
\end{split}\end{split}
\end{equation}\begin{equation}\label{equation:02.01_manoeuvring models:eqnmartinssimple}
\begin{split}\begin{split}
\operatorname{N_{D}'}{\left(u',v',r',\delta,thrust' \right)} = & N_{\delta} \delta + N_{r} r' + N_{T\delta} \delta thrust' + N_{T} thrust' + N_{ur} r' u' + N_{u} u' \\ & + N_{vv\delta} \delta v'^{2} + N_{v} v' 
\end{split}\end{split}
\end{equation}

The hydrodynamic functions above are expressed using nondimensional units with the prime system, denoted by the prime symbol (\('\)). The quantities are expressed in the prime system, using the denominators in \autoref{tab:prime-system-denominators}. For instance, surge linear velocity \(u\) can be expressed in the prime system as seen in \autoref{equation:02.01_manoeuvring models:eqprime} using the linear velocity denominator.
\begin{equation}\label{equation:02.01_manoeuvring models:eqprime}
\begin{split}\displaystyle u'=\frac{u}{V}\end{split}
\end{equation}

Equations can either be written in the Prime or regular Standard Institute (SI) system. The hydrodynamic derivatives are always expressing forces in the prime system as function of state variables. The (\('\)) sign is therefore implicit and not written out as seen in \autoref{equation:02.01_manoeuvring models:eqderivativeprime}.
\begin{equation}\label{equation:02.01_manoeuvring models:eqderivativeprime}
\begin{split}\displaystyle Y_{\delta'}'=\frac{\partial Y_D'}{\partial \delta'} := Y_{\delta} \end{split}
\end{equation}

The exceptions are the added masses (\(X_{\dot{u}}\), \(Y_{\dot{v}}\), \(Y_{\dot{r}}\), \(N_{\dot{v}}\) and \(N_{\dot{r}}\)) which are expressed in both Prime system or the regular SI system where the (\('\)) sign is therefore
explicitly stated.
There is however a great benefit in expressing the hydrodynamic forces in the prime system. The forces are often nonlinear due to a quadratic relation to the flow velocity, as seen in \autoref{equation:02.01_manoeuvring models:eqquadraticsi}.
\begin{equation}\label{equation:02.01_manoeuvring models:eqquadraticsi}
\begin{split}\displaystyle Y_{D}=Y_{\delta} \cdot \delta \cdot \frac{L^2V^2\rho}{2}\end{split}
\end{equation}
which becomes linear when expressed in the prime system as seen in \autoref{equation:02.01_manoeuvring models:eqquadraticprime}.
\begin{equation}\label{equation:02.01_manoeuvring models:eqquadraticprime}
\begin{split}\displaystyle Y_{D}'=Y_{\delta} \cdot \delta'\end{split}
\end{equation}