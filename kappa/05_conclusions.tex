%%%%%%%%%%%%%%%%%%%%%%%%%%%%%%%
%%%%%%%%%%%%%%%%%%%%%%%%%%%%%%%
\chapter{Conclusions\label{ch:conclusions}}
%%%%%%%%%%%%%%%%%%%%%%%%%%%%%%%

\begin{itemize}
    
    \item The hydrodynamic derivatives within a manoeuvring model can be identified exactly at ideal conditions with no measurement noise and a perfect estimator.
    
    \item System identification methods must handling measurement noise and minimize the process uncertainty to be successful.
    
    \item Prepossessing of measurement data with EKF + RTS run in iteration with initial estimation from semi-empirical formulas, is better than using low-pass filters.
    
    \item Too complex models has great problems with multicollinearity, that need to be handled.
    
    \item Models developed with the proposed system identification method can predict turning circle manoeuvres with less than 5 \% error in advance and tactical diameter for the wPCC and KVLCC2 test cases.
    
\end{itemize}
