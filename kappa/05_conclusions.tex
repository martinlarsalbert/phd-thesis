%%%%%%%%%%%%%%%%%%%%%%%%%%%%%%%
%%%%%%%%%%%%%%%%%%%%%%%%%%%%%%%
\chapter{Conclusions\label{ch:conclusions}}
%%%%%%%%%%%%%%%%%%%%%%%%%%%%%%%
% The conclusions of the stated objectives...
% it was found that...
% it was shown that...
% it was proposed that...

The main conclusions are presented in this section with respect to the main objective of this thesis:
% Objective: 
\begin{tcolorbox}[sharp corners,title=Objective]
\objective
\end{tcolorbox}
\vspace{0.3cm}
\noindent The conclusions are categorized by the goals that comprise this objective.

% RO1
\begin{tcolorbox}[sharp corners,title=RO1]
Establish roll motion and manoeuvring model structures that are very similar to the true model.
\end{tcolorbox}
In Paper \ref{pap:rolldamping} it was attempted to establish a physics-informed model for ship roll motion by combining a semi-empirical deterministic roll damping model with a data-driven model identified on a large number of roll decay tests. However, this was a failed attempt, mainly because most of the investigated ships were outside the limits of the semi-empirical model used.  

Using potential flow calculations as the physics-informed part together with semi-empirical formulas for the viscous parts of the roll damping was shown in Paper \ref{pap:ikeda} to be a very efficient way to accurately predict the ship roll motion.

A thorough investigation with VCT and FRMTs inverse dynamics in Paper \ref{pap:vct} showed that the used simulation model could well describe the hull and rudder forces as well as the ship dynamics for a ship with large rudder such as the Optiwise test case where the exclusion of wave generation and the use of only three degrees of freedom were valid assumptions. 

% RO2
\begin{tcolorbox}[sharp corners,title=RO2]
Develop system identification of roll motion and ship manoeuvring from FRMTs.
\end{tcolorbox}
Two different parameter identification techniques were attempted in Paper \ref{pap:rolldamping} to identify the damping terms, where the so called ''derivation approach'' turned out to be the best. In this approach the differential equations of the models were just treated as a linear regression problem. The identification technique was later referred to as inverse dynamics (ID) regression. A problem with ID is that the whole state of the system must be known; In this case the roll angle including its first and second time derivatives ($\theta,\dot{\theta},\ddot{\theta}$ ) must be known, which is often not the case for model test or ship operational data. Instead, velocity and acceleration needs to be estimated by numerical differentiation or other method.

% RO3
\begin{tcolorbox}[sharp corners,title=RO3]
Develop a method to analyse identified models with FRMTs.
\end{tcolorbox}
The method to use inverse dynamics in combination with the state VCTs was in Paper \ref{pap:vct} found to be a useful tool to test the identified models and see if errors origin from the model structure or are inherent in the VCT data. It was for instance shown that the state VCT did not agree well with the wPCC inverse dynamics forces, which indicates that there was something missing in the wPCC VCT data. This could, for instance, be the lack of wave generation or heeled cases in the VCT data, which might be too much of a simplification for the wPCC. For Optiwise on the other hand, which was run at much lower Froude number and little heel, the state VCT agreed much better with the inverse dynamics forces and better agreement was obtained in the closed loop simulations.

% RO4
\begin{tcolorbox}[sharp corners,title=RO4]
Mitigate the system identification multicollinearity.
\end{tcolorbox}
In Paper \ref{pap:pit} multicollinearity was found to be a significant problem for system identification on data from IMO standard maneuvers, especially for complex models such as the Abkowitz model structure. Consequently, some of the regressed hydrodynamic derivatives had unphysically large values and substantial uncertainties in the regressed polynomials with sums of large counteracting coefficients. However, the models were still mathematically correct, where the regressed polynomials fit the training data well. The model worked as long as the states were similar to the training data. However, when extrapolating, it is easy to imagine that the balance between these massive derivatives could be disturbed, giving significant extrapolation errors very quickly.

%However, multicollinearity is not necessarily a sign that the model structure is wrong, it is rather a sign that the data used do not follow the aspect of persistence of excitation. This means that the structure of the model could be correct, but the data used for identification do not contain enough information to identify all parameters correctly.

It was shown in Paper \ref{pap:pit} that the multicollinearity can be mitigated by reducing the number of hydrodynamic derivatives so that the model could generalize from the zigzag training data to predict turning circles with less than 5\% error in advance and tactical diameter for the two investigated test cases.

However, in Paper \ref{pap:physics}, it was shown that the identified models in Paper \ref{pap:pit} would not generalize well when the ship was exposed to external wind forces. The high correlation between the yaw rates and the drift angles during the maneuvers prevented a correct split between the yaw rate and the drift-dependent coefficients.

It was also shown in Paper \ref{pap:physics} that identification was unable to make a correct separation between the hull and the rudder forces when only the total force of the inverse dynamics was available. 
A deterministic semi-empirical rudder model was proposed instead of the data driven rudder model to alleviate this part of the multicollinearity problem. It was shown that this guided the identification toward a more physically correct model, with lower multicollinearity and better generalization to the wind conditions. There were, however, still problems with multicollinearity from yaw rate and drift which needs to be resolved in other ways, typically with more informative data. 
