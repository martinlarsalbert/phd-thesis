%%%%%%%%%%%%%%%%%%%%%%%%%%%%%%%
%%%%%%%%%%%%%%%%%%%%%%%%%%%%%%%
\chapter{Conclusions\label{ch:conclusions}}
%%%%%%%%%%%%%%%%%%%%%%%%%%%%%%%
% The conclusions of the stated objectives...
% it was found that...
% it was shown that...
% it was proposed that...

%...short summary 
\noindent This thesis investigated the enhancement of ship manoeuvring models through the
integration of prior knowledge embedded in parametric model structures and semi-empirical
formulas. The main conclusion is that physically accurate models
can be derived from parametric model structures when prior knowledge about ship
hydrodynamics and semi-empirical formulas are embedded in the structure, provided
that the observed data are correct and informative, as demonstrated for the Optiwise test case in Paper \ref{pap:vct}.
It was also concluded that physically accurate models could not be identified from standard maneuvers, because they have insufficient informative data. However, by adding a semi-empirical rudder model, the identification process was guided towards a more physically accurate model.

Referring to the objectives listed in \autoref{sec:motivation}, detailed conclusions are presented below.

\subsection*{A parametric model structure and parameter identification technique for roll motion}
As demonstrated in Paper \ref{pap:rolldamping}, inverse dynamics regression (referred to as the derivation approach in that paper) is an efficient method for identifying parameters in parametric roll motion models. The study also showed that 250 roll decay tests were well described by the quadratic model structure. Consequently, this model structure was proposed as a robust framework for system identification of roll motion, offering good generalization.

\subsection*{Parameter identification techniques for ship manoeuvring models}
This thesis proposes parameter identification techniques for both FT and CT (VCT) data.

For FT data, inverse dynamics regression was utilized as the core component in a new parameter identification method, as proposed in Paper \ref{pap:pit}. This method combines inverse dynamics with an EKF in a two-step iterative approach \cite{yoonIdentificationHydrodynamicCoefficients2003}. Paper \ref{pap:pit} revealed that complex manoeuvring model structures, such as the Abkowitz model, cannot be identified from standard maneuvers due to multicollinearity. However, multicollinearity can be mitigated by selecting a model structure that reduces the number of hydrodynamic derivatives. The new method demonstrated the capability to identify manoeuvring models that accurately predicted the FRMT data.

A method was also proposed for CT data, with its novelty lying in the design of the VCT matrix to include the most critical states during maneuvers, covering the relevant parts of the state space. The proposed identification procedure for CT data proved to be highly effective, as demonstrated by the Optiwise test case.

\subsection*{Parametric model structure with good generalization identifiable from standard maneuvers}
To assess generalization from simpler to more complex maneuvers, the new identification method for FT data demonstrated the ability to identify models from zigzag tests that could predict the results of significantly different turning circle tests with less than 5\% error, including the tactical diameter for two investigated ships. Consequently, it was concluded that a model identified from standard maneuvers using the proposed method is capable of simulating other standard maneuvers with reasonably good accuracy.

However, Paper \ref{pap:physics} showed that the identified models in Paper \ref{pap:pit} did not generalize well when the ship was exposed to external wind forces. The model was found to be physically incorrect, despite being mathematically correct. The identification method failed to correctly separate the hull and rudder forces when only the total force of the inverse dynamics was available. To address this, a deterministic semi-empirical rudder model was proposed instead of the data-driven rudder model, which helped guide the identification towards a more physically correct model with lower multicollinearity and better generalization to wind conditions.

Despite these improvements, issues with multicollinearity persisted due to the high correlation between yaw rate and drift during standard maneuvers, preventing the identification of perfectly physically correct models in Paper \ref{pap:physics}. It was demonstrated that these problems could be resolved with additional VCT calculations to obtain more informative data.

\subsection*{Semi-empirical formulas to mitigate multicollinearity and enhance generalization}
There are many semi-empirical formulas available in the literature, containing prior knowledge about hydrodynamics that could potentially help in identifying models.       
Semi-empirical methods to predict roll damping were investigated in Paper \ref{pap:rolldamping} . Predictions with Ikeda's method, including 2D potential flow strip calculations, were in fair agreement with roll decay tests for 15 investigated ships. 
A more in-depth analysis of roll damping was conducted for the KVLCC2 test case in Paper \ref{pap:ikeda}. The 2D potential flow strip calculation was here replaced by a more modern potential flow code. The combined method produced roll motion prediction with high accuracy. It was thereby concluded that Ikeda's method provides a good semi-empirical method for predicting the viscous roll damping.

In Paper \ref{pap:physics}, a new semi-empirical rudder model was proposed for the twin rudder wPCC test case based on various semi-empirical formulas from the literature. The model was in good agreement with the VCT data in Paper \ref{pap:physics} and \ref{pap:vct}. 
A modified quadratic version of the MMG semi-empirical rudder model was proposed in Paper \ref{pap:vct}. 
It was shown that this model could well predict the VCT data and also the measured rudder forces during zigzag model tests for the single rudder Optiwise test case. 
The semi-empirical rudder model from Paper \ref{pap:physics} and the modified quadratic MMG rudder model from Paper \ref{pap:vct} belong to the same family of semi-empirical rudder models which was shown to be very capable of describing the true forces from the rudder/rudders during standard manoeuvres.  