\section{Test cases} \label{sec:test_cases}
Two test cases have been studied in this thesis. The wPCC test case is a ship that was designed for a wind-assisted propulsion system (WAPS) and is capable of operating in a fully sailing mode, a fully motoring mode, and intermediate states. 
However, this thesis only considers the motoring mode. The wPCC design differs slightly from conventional motoring cargo ship designs because of the WAPS. It has two very large rudders, which are two to three times larger than those needed for a conventional ship. The ship also has fins at the bilge to generate extra lift while sailing, as shown on the scale model in \autoref{fig:wPCC}.
\autoref{tab:main_particulars} shows the main particulars of the scale model. 
\begin{figure}[h]
    \centering
    \includegraphics[width=\columnwidth]{figures/5m2.jpg}
    \caption{Scale model of the wPCC used in the model tests. Copyright RISE.}
    \label{fig:wPCC}
\end{figure}

The Optiwse test case is based on a typical VLCC tanker but features a larger rudder size adapted for the WAPS, as shown in the scale model in \autoref{fig:optiwise}. \autoref{tab:main_particulars} shows the main particulars of the scale model. 
\begin{figure}[h]
    \centering
    \includegraphics[width=\columnwidth]{figures/optiwise.jpg}
    \caption{Scale model of the Optiwise used in the model tests. Copyright RISE.}
    \label{fig:optiwise}
\end{figure}
\begin{table}[h]
    \centering
    \caption{Main particulars of the test case scale models.}
    \label{tab:main_particulars}
    \pgfplotstabletypeset[col sep=comma, column type=r, columns={Parameter,Unit, wPCC, Optiwise, Description},
        columns/Parameter/.style={column type=l,string type},
        columns/Unit/.style={column type=l,string type, column name=~},
        columns/wPCC/.style={column type=r},
        columns/Optiwise/.style={column type=r},
        columns/Description/.style={column type=l,string type},
        every head row/.style={before row=\hline,after row=\hline},
        every last row/.style={after row=\hline}
    ]{tables/test_cases.main_particulars.csv}
\end{table}