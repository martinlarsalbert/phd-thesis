\section{Summary of Paper \ref{pap:ikeda}}
\subsection*{"\nameref{pap:ikeda}"}
\subsection*{Scope and motivations}
The impact of roll motions can be seen from the APL China casualty in 1998, where a post-Panamax C11 class container ship lost almost a third of its containers \cite{france_investigation_2001}. Another example is the container ship Svendborg Maersk, were 500 containers were lost overboard and 250 containers were damaged as a result of heavy roll motions during a passage from English Channel to Gibraltar \cite{danish_maritime_accident_investigation_board_marine_2014}.

Analyzing the ship's roll damping is therefore crucial, not the least in the earlier design stages of a ship, where efficient and accurate methods are needed. 
Getting the best possible accuracy with the lowest possible computational cost is an important factor. An explicit semi-empirical formula was proposed in Paper \ref{pap:rolldamping}, based on the simplified Ikeda's method \cite{kawahara_simple_2011}. This is an alternative with very low computational cost. However, it was also found to have poor accuracy, especially for modern ship designs. 

Paper \ref{pap:ikeda} proposes a hybrid method, as a solution to this problem, where the viscous roll damping from Ikeda’s semi-empirical method is injected into an existing 3D unsteady fully nonlinear potential flow (FNPF) method.

\subsection*{Results and concluding remarks}
The viscous roll damping was calculated with Ikeda's method \cite{ikeda_components_1978} for the KVLCC2 test case. Error in the calculation of the eddy damping at zero speed was encountered, which was found to originate from a regression formula from experiments on a number of two-dimensional cylinders with various sections \cite{ikeda_eddy_1978}. A new regression was instead proposed, using a decision tree model, which was found 

\subsection*{Comments}
\clearpage