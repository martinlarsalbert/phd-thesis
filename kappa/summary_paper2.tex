\section{Summary of Paper \ref{pap:ikeda}}
\subsection*{"\nameref{pap:ikeda}"}
\subsection*{Scope and motivations}
The impact of roll motions can be seen from the APL China casualty in 1998, where a post-Panamax C11 class container ship lost almost a third of its containers \cite{france_investigation_2001}. Another example is the container ship Svendborg Maersk, were 500 containers were lost overboard and 250 containers were damaged as a result of heavy roll motions during a passage from English Channel to Gibraltar \cite{.

Getting the best possible accuracy with the lowest possible computational cost is an important factor in the early design stage of ships. An explicit semi-empirical formula was proposed in \nameref{pap:rolldamping}, based on the simplified Ikeda's method \cite{kawahara_simple_2011}. This is an alternative with very low computational cost. However, it was found to have poor accuracy, especially for modern ship designs. \cite{ikeda_components_1978}


Potential flow-based analysis presents such a solution for seakeeping analyses. The accuracy of roll motion in potential flow is however not so good, due to the large influence from vicsous roll damping, which is missing in these calculations.” Page 2

Semi-empirical formulas were developed to estimate the viscous parts, to be used together with the potential flow methods (Ikeda et al., 1978). The older linear methods can today be replaced by more advanced nonlinear potential flow methods. These newer methods still need some injection of semi-empirical viscous damping to give a fair representation of the roll motions.


\subsection*{Results and concluding remarks}
\subsection*{Comments}
\clearpage