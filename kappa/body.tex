%!TEX root = ../main.tex
%%%%%%%%%%%%%%%%%%%%%%%%%%%%%%%
%%%%%%%%%%%%%%%%%%%%%%%%%%%%%%%
\chapter{Introduction}
%%%%%%%%%%%%%%%%%%%%%%%%%%%%%%%
\section{Background}
Modeling of ship’s dynamics has a wide range of useful applications. This can be categorized as: white-box, black-box or grey-box modeling \cite{leifsson_grey-box_2008}. Most of the models in this thesis fall under the grey-box category.

\begin{itemize}
    \item White-box modeling \\
    involves applying physical principles, so that no observed data is required, for instance Computational Fluid Dynamics (CFD). Semi-empirical models where unknown physical constants have been derived from previous experiments, can also be considered as white-box models \cite{leifsson_grey-box_2008}. Ikeda's method to predict roll damping \cite{ikeda_components_1978} is one example of this (see Paper \ref{pap:rolldamping}).  

    \item Black-box modeling \\
    means that parameters do not have physical significance but where the objectives is to find a good model that fits the observed data \cite{lindskog_tools_1995}.
    
    \item Grey-box modeling \\
    is a combination of white-box and black-box modeling methods, so that both a physical model and data is required. This concept is also referred as semi-physical modeling, hybrid modeling or semi-mechanistic modeling \cite{leifsson_grey-box_2008} in the literature. 
\end{itemize}

\noindent The black-box modeling is entirely data driven, which means that no prior understanding of the system generating the data is needed. The main disadvantage is the dependence on the data used to model the system, which can result in limited extrapolation properties beyond the data that it is derived from \cite{leifsson_grey-box_2008}. 
In a grey box model the white and black parts can be combined in several ways using either a serial or parallel approach \cite{leifsson_grey-box_2008}. 
Grey-box modelling is often used in situations where white-box models are not giving the required accuracy by introducing some corrections before or after the white-box in a serial approach. 
Grey-box modeling of a motorcycle shock absorber \cite{beghi_grey-box_2007} is an example of the parallel approach where the low frequency dynamics is handled by a white box and the higher frequencies are handled by a black box.

\section{Literature review}


%"Critic" to what has been done before
\section{Motivation and objective}
A majority of the publications in the field of system identification use simulated data with added white noise. Even though noise has been added, the estimator is still perfect, which means that these methods are not solving the whole problem, where both noise and system model should be unknown. Only when using data obtained from real physical systems, from model tests or full scale ship operation, the whole system identification problem is addressed.

crawling, walking, running...

\section{Assumptions and limitations}
Calm waters...

\section{Outline of the paper}

%%%%%%%%%%%%%%%%%%%%%%%%%%%%%%%
%%%%%%%%%%%%%%%%%%%%%%%%%%%%%%%
\chapter{Ship rigid body dynamics models}
\label{ch:models}
%%%%%%%%%%%%%%%%%%%%%%%%%%%%%%%

The ship's dynamics comprises the forces and motions in the six degrees of freedoms (6DOF): surge, sway, heave, roll, pitch and yaw. Heave and pitch motions are often neglected in calm water conditions, so that a four degrees of freedom (4DOF) model is sufficient to express the ship's dynamics. Physically parameterized models for roll and the Vessel Manoeuvring Model (VMM) for the remaining DOFs are presented in section \ref{sec:roll} and section \ref{sec:VMM}. How the parameters in these models can be identified is presented in \ref{sec:PIT_roll}.

\section{Roll motion} \label{sec:roll}

The roll motion on a straight course in calm water with no external forces can be expressed with Eq.\ref{eq:roll_decay_equation_general_himeno} \cite{himeno_prediction_1981},
\begin{equation} \label{eq:roll_decay_equation_general_himeno}
A_{44} \ddot{\phi} + \operatorname{B_{44}}\left(\dot{\phi}\right) + \operatorname{C_{44}}\left(\phi\right) = 0
\end{equation}


\noindent where $B_{44}\left(\dot{\phi}\right)$ can be expressed as expansion series:  
$ B_{44}\left(\dot{\phi}\right) = B_1\cdot\dot{\phi} + B_2\cdot\dot{\phi}\left|\dot{\phi}\right| + B_3\cdot\dot{\phi}^3 + ... + B_n\cdot\dot{\phi}^n$. Most often, the so-called ``linear model'' (Eq.\ref{eq:roll_decay_equation_himeno_linear}), ``quadratic model'' (Eq.\ref{eq:roll_decay_equation_himeno_quadratic_b}) and ``cubic model'' (Eq.\ref{eq:roll_decay_equation_cubic}) are used to represent $B_{44}(\dot{\phi})$ in by truncating the series to keep only linear, quadratic and cubic terms,

\begin{equation} \label{eq:roll_decay_equation_himeno_linear}
A_{44} \ddot{\phi} + B_{1} \dot{\phi} + C_{1} \phi = 0
\end{equation}

\begin{equation} \label{eq:roll_decay_equation_himeno_quadratic_b}
A_{44} \ddot{\phi} + C_{1} \phi + \left(B_{1} + B_{2} \left|{\dot{\phi}}\right|\right) \dot{\phi} = 0
\end{equation}

\begin{equation} \label{eq:roll_decay_equation_cubic}
A_{44} \ddot{\phi} + \left(B_{1} + B_{2} \left|{\dot{\phi}}\right| + B_{3} \dot{\phi}^{2}\right) \dot{\phi} + \left(C_{1} + C_{3} \phi^{2} + C_{5} \phi^{4}\right) \phi = 0
\end{equation}


\section{Vessel Manoeuvring Models} \label{sec:manoeuvring model}
\label{\detokenize{02.01_VMMs:vessel-manoeuvring-models}}\label{\detokenize{02.01_VMMs:vmm}}\label{\detokenize{02.01_VMMs::doc}}
Ship manoeuvring is a simplified case of seakeeping. The encountering waves have been removed, assuming calm water conditions. The manoeuvring motions have low frequencies so that added masses and other hydrodynamic derivatives can be assumed as constants  \cite{fossen_handbook_2021}. Three manoeuvring models are used in this thesis: 
\begin{itemize}
    \item Linear (LVMM) \cite{matusiak_dynamics_2017}
    \item Abkowitz (AVMM), \cite{abkowitz_ship_1964}
    \item Modified Abkowitz (MAVMM), which is proposed in Paper \ref{pap:pit}
\end{itemize}

\noindent\autoref{\detokenize{02.01_VMMs:coordinate-system}} shows the reference frames used in the VMMs where \(x_0\) and \(y_0\) and heading \(\Psi\) are the global position and orientation of a ship fix reference frame \(O(x,y,z)\) (or rather \(O(x,y)\) when heave is excluded) with origin at midship. \(u\), \(v\), \(r\), \(X\), \(Y\) and \(N\) are velocities and forces in the ship fix reference frame.



\begin{figure}[H]
    \centering
    \includegraphics[width=\textwidth]{kappa/images/coordinate_system.PNG}
    \caption{Reference frames}
    \label{\detokenize{02.01_VMMs:coordinate-system}}
\end{figure}

\noindent The acceleration can be solved from the manoeuvring equation (\autoref{equation:02.01_VMMs:eqqsystem}) \cite{fossen_handbook_2021} as seen in \autoref{equation:02.01_VMMs:eqacc},
\begin{equation}\label{equation:02.01_VMMs:eqqsystem}
\begin{split}\displaystyle \left[\begin{matrix}- X_{\dot{u}} + m & 0 & 0\\0 & - Y_{\dot{v}} + m & - Y_{\dot{r}} + m x_{G}\\0 & - N_{\dot{v}} + m x_{G} & I_{z} - N_{\dot{r}}\end{matrix}\right] \left[\begin{matrix}\dot{u}\\\dot{v}\\\dot{r}\end{matrix}\right] = \left[\begin{matrix}m r^{2} x_{G} + m r v + \operatorname{X_{D}}{\left(u,v,r,\delta,thrust \right)}\\- m r u + \operatorname{Y_{D}}{\left(u,v,r,\delta,thrust \right)}\\- m r u x_{G} + \operatorname{N_{D}}{\left(u,v,r,\delta,thrust \right)}\end{matrix}\right]\end{split}
\end{equation}\begin{equation}\label{equation:02.01_VMMs:eqacc}
\begin{split}\displaystyle \dot{\nu} = \left[\begin{matrix}\dot{u}\\\dot{v}\\\dot{r}\end{matrix}\right] = \left[\begin{matrix}\frac{1}{- X_{\dot{u}} + m} & 0 & 0\\0 & - \frac{- I_{z} + N_{\dot{r}}}{S} & - \frac{- Y_{\dot{r}} + m x_{G}}{S}\\0 & - \frac{- N_{\dot{v}} + m x_{G}}{S} & - \frac{Y_{\dot{v}} - m}{S}\end{matrix}\right] \left[\begin{matrix}m r^{2} x_{G} + m r v + \operatorname{X_{D}}{\left(u,v,r,\delta,thrust \right)}\\- m r u + \operatorname{Y_{D}}{\left(u,v,r,\delta,thrust \right)}\\- m r u x_{G} + \operatorname{N_{D}}{\left(u,v,r,\delta,thrust \right)}\end{matrix}\right]\end{split}
\end{equation}
\sphinxAtStartPar
where \(S\) is a helper variable:
\begin{equation}\label{equation:02.01_VMMs:eq_S}
\begin{split}\displaystyle S = - I_{z} Y_{\dot{v}} + I_{z} m + N_{\dot{r}} Y_{\dot{v}} - N_{\dot{r}} m - N_{\dot{v}} Y_{\dot{r}} + N_{\dot{v}} m x_{G} + Y_{\dot{r}} m x_{G} - m^{2} x_{G}^{2}\end{split}
\end{equation}
\sphinxAtStartPar
A state space model for manoeuvring can now be defined with six states:
\begin{equation}\label{equation:02.01_VMMs:eq_x}
\begin{split}\displaystyle \mathbf{x} = \left[\begin{matrix}x_{0}\\y_{0}\\\Psi\\u\\v\\r\end{matrix}\right]\end{split}
\end{equation}
\sphinxAtStartPar
The time derivative of this state \(\dot{\mathbf{x}}\) can be defined by a state transition \(f(\mathbf{x},\mathbf{c})\) using geometrical relations
how global coordinates \(x_0\), \(y_0\) and \(\Psi\) depend on \(u\), \(v\), and \(r\) viz.,
\begin{equation}\label{equation:02.01_VMMs:eqf}
\begin{split}\displaystyle \dot{\mathbf{x}} = f(\mathbf{x},\mathbf{c}) + \mathbf{w}
                                          = \left[\begin{matrix}\dot{x_0}\\ \dot{y_0} \\ \dot{\Psi} \\\dot{u}\\\dot{v}\\\dot{r}\end{matrix}\right] + \mathbf{w}
                                          = \left[\begin{matrix}u \cos{\left(\Psi \right)} - v \sin{\left(\Psi \right)}\\u \sin{\left(\Psi \right)} + v \cos{\left(\Psi \right)}\\r\\\dot{u}\\\dot{v}\\\dot{r}\end{matrix}\right] + \mathbf{w}\end{split}
\end{equation}
\sphinxAtStartPar
where \(\mathbf{c}\) is control inputs (rudder angle \(\delta\) and thrust); the last three derivatives: \(\dot{u}\), \(\dot{v}\), \(\dot{r}\) are calculated with \autoref{equation:02.01_VMMs:eqacc}.
\(\mathbf{w}\) is the process noise, i.e., the difference between the predicted state by the manoeuvring model and the true
state of the system. \(\mathbf{w}\) is unknown when the manoeuvring model is used for manoeuvre predictions and therefore normally
assumed to be zero, but it is an important factor when the manoeuvring model is used in the EKF (see Section \ref{sec:datacleaning}).
The manoeuvring simulation can now be conducted by numerical integration of \autoref{equation:02.01_VMMs:eqf}. The main difference between the manoeuvring model:s lies in how the hydrodynamic functions \(X_D(u,v,r,\delta,thrust)\), \(Y_D(u,v,r,\delta,thrust)\), \(N_D(u,v,r,\delta,thrust)\) are defined. These expressions are denoted below for the various VMMs: LVMM, AVMM and MAVMM.

\sphinxAtStartPar
LVMM (Linear Vessel Manoeuvring Model) \cite{matusiak_dynamics_2017}:
\begin{equation}\label{equation:02.01_VMMs:eqxlinear}
\begin{split}\begin{split}
\operatorname{X_{D}'}{\left(u',v',r',\delta\right)} = & X_{\delta} \delta + X_{r} r' + X_{u} u' + X_{v} v' 
\end{split}\end{split}
\end{equation}\begin{equation}\label{equation:02.01_VMMs:eqylinear}
\begin{split}\begin{split}
\operatorname{Y_{D}'}{\left(u',v',r',\delta \right)} = & Y_{\delta} \delta + Y_{r} r' + Y_{u} u' + Y_{v} v' 
\end{split}\end{split}
\end{equation}\begin{equation}\label{equation:02.01_VMMs:eqnlinear}
\begin{split}\begin{split}
\operatorname{N_{D}'}{\left(u',v',r',\delta \right)} = & N_{\delta} \delta + N_{r} r' + N_{u} u' + N_{v} v' 
\end{split}\end{split}
\end{equation}
\sphinxAtStartPar
AVMM (Abkowitz Vessel Manoeuvring Model) \cite{abkowitz_ship_1964}:
\begin{equation}\label{equation:02.01_VMMs:eqxabkowitz}
\begin{split}
\operatorname{X_{D}'}{\left(u',v',r',\delta,thrust' \right)} = & X_{\delta\delta} \delta^{2} + X_{r\delta} \delta r' + X_{rr} r'^{2} + X_{T} thrust' + X_{u\delta\delta} \delta^{2} u' \\ 
& + X_{ur\delta} \delta r' u' + X_{urr} r'^{2} u' + X_{uuu} u'^{3} + X_{uu} u'^{2} \\ 
& + X_{uv\delta} \delta u' v' + X_{uvr} r' u' v' + X_{uvv} u' v'^{2} \\
& + X_{u} u' + X_{v\delta} \delta v' + X_{vr} r' v' + X_{vv} v'^{2} 
\end{split}
\end{equation}

\begin{equation}\label{equation:02.01_VMMs:eqyabkowitz}
\begin{split}\begin{split}
\operatorname{Y_{D}'}{\left(u',v',r',\delta,thrust' \right)} = & Y_{0uu} u'^{2} + Y_{0u} u' + Y_{0} + Y_{\delta\delta\delta} \delta^{3} + Y_{\delta} \delta + Y_{r\delta\delta} \delta^{2} r' + Y_{rr\delta} \delta r'^{2} \\ & + Y_{rrr} r'^{3} + Y_{r} r' + Y_{T\delta} \delta thrust' + Y_{T} thrust' + Y_{u\delta} \delta u' \\ & + Y_{ur} r' u' + Y_{uu\delta} \delta u'^{2} + Y_{uur} r' u'^{2} + Y_{uuv} u'^{2} v' + Y_{uv} u' v' \\ & + Y_{v\delta\delta} \delta^{2} v' + Y_{vr\delta} \delta r' v' + Y_{vrr} r'^{2} v' + Y_{vv\delta} \delta v'^{2} + Y_{vvr} r' v'^{2} \\ & + Y_{vvv} v'^{3} + Y_{v} v' 
\end{split}\end{split}
\end{equation}\begin{equation}\label{equation:02.01_VMMs:eqnabkowitz}
\begin{split}\begin{split}
\operatorname{N_{D}'}{\left(u',v',r',\delta,thrust' \right)} = & N_{0uu} u'^{2} + N_{0u} u' + N_{0} + N_{\delta\delta\delta} \delta^{3} + N_{\delta} \delta + N_{r\delta\delta} \delta^{2} r' + N_{rr\delta} \delta r'^{2} \\ & + N_{rrr} r'^{3} + N_{r} r' + N_{T\delta} \delta thrust' + N_{T} thrust' + N_{u\delta} \delta u' \\ & + N_{ur} r' u' + N_{uu\delta} \delta u'^{2} + N_{uur} r' u'^{2} + N_{uuv} u'^{2} v' + N_{uv} u' v' \\ & + N_{v\delta\delta} \delta^{2} v' + N_{vr\delta} \delta r' v' + N_{vrr} r'^{2} v' + N_{vv\delta} \delta v'^{2} + N_{vvr} r' v'^{2} \\ & + N_{vvv} v'^{3} + N_{v} v' 
\end{split}\end{split}
\end{equation}
\sphinxAtStartPar
MAVMM (Modified Abkowitz Vessel Manoeuvring Model, where only the most relevant coefficients in AVMM are included.)
\begin{equation}\label{equation:02.01_VMMs:eqxmartinssimple}
\begin{split}\begin{split}
\operatorname{X_{D}'}{\left(u',v',r',\delta,thrust' \right)} = & X_{\delta\delta} \delta^{2} + X_{rr} r'^{2} + X_{T} thrust' + X_{uu} u'^{2} + X_{u} u' + X_{vr} r' v' 
\end{split}\end{split}
\end{equation}\begin{equation}\label{equation:02.01_VMMs:eqymartinssimple}
\begin{split}\begin{split}
\operatorname{Y_{D}'}{\left(u',v',r',\delta,thrust' \right)} = & Y_{\delta} \delta + Y_{r} r' + Y_{T\delta} \delta thrust' + Y_{T} thrust' + Y_{ur} r' u' \\ & + Y_{u} u' + Y_{vv\delta} \delta v'^{2} + Y_{v} v' 
\end{split}\end{split}
\end{equation}\begin{equation}\label{equation:02.01_VMMs:eqnmartinssimple}
\begin{split}\begin{split}
\operatorname{N_{D}'}{\left(u',v',r',\delta,thrust' \right)} = & N_{\delta} \delta + N_{r} r' + N_{T\delta} \delta thrust' + N_{T} thrust' + N_{ur} r' u' + N_{u} u' \\ & + N_{vv\delta} \delta v'^{2} + N_{v} v' 
\end{split}\end{split}
\end{equation}
\sphinxAtStartPar
The hydrodynamic functions above are expressed using nondimensional units with the prime system, denoted by the prime symbol (\('\)). The quantities are expressed in the prime system, using the denominators in Tab.\ref{tab:my_label}. For instance, surge linear velocity \(u\) can be expressed in the prime system as seen in \autoref{equation:02.01_VMMs:eqprime} using the linear velocity denominator.
\begin{equation}\label{equation:02.01_VMMs:eqprime}
\begin{split}\displaystyle u'=\frac{u}{V}\end{split}
\end{equation}
\sphinxAtStartPar
Equations can either be written in the prime or regular SI system. The hydrodynamic derivatives are always expressing forces in the prime system as function of state variables. The (\('\)) sign is therefore implicit and not written out as seen in \autoref{equation:02.01_VMMs:eqderivativeprime}.
\begin{equation}\label{equation:02.01_VMMs:eqderivativeprime}
\begin{split}\displaystyle Y_{\delta'}'=\frac{\partial Y_D'}{\partial \delta'} := Y_{\delta} \end{split}
\end{equation}
\sphinxAtStartPar
The exceptions are the added masses (\(X_{\dot{u}}\), \(Y_{\dot{v}}\), \(Y_{\dot{r}}\), \(N_{\dot{v}}\) and \(N_{\dot{r}}\)) which are expressed in both Prime system or the regular SI system where the (\('\)) sign is therefore
explicitly stated.
There is however a great benefit in expressing the hydrodynamic forces in the prime system. The forces are often nonlinear due to a quadratic relation to the flow velocity, as seen in \autoref{equation:02.01_VMMs:eqquadraticsi}.
\begin{equation}\label{equation:02.01_VMMs:eqquadraticsi}
\begin{split}\displaystyle Y_{D}=Y_{\delta} \cdot \delta \cdot \frac{L^2V^2\rho}{2}\end{split}
\end{equation}
which becomes linear when expressed in the prime system as seen in \autoref{equation:02.01_VMMs:eqquadraticprime}.
\begin{equation}\label{equation:02.01_VMMs:eqquadraticprime}
\begin{split}\displaystyle Y_{D}'=Y_{\delta} \cdot \delta'\end{split}
\end{equation}


\begin{table}[]
\caption{Prime system denominators}
\label{tab:prime-system-denominators}
\centering
\label{tab:my_label}
\begin{tabular}{|c|c|}
\hline
Quantity &
Denominators
\\
\hline

angle
&

\(1\)
\\


angular
acceleration
&

\(\frac{V^{2}}{L^{2}}\)
\\


angular
velocity
&

\(\frac{V}{L}\)
\\


area
&

\(L^{2}\)
\\


density
&

\(\frac{\rho}{2}\)
\\


force
&

\(\frac{L^{2} V^{2} \rho}{2}\)
\\


frequency
&

\(\frac{V}{L}\)
\\


inertia
moment
&

\(\frac{L^{5} \rho}{2}\)
\\


length
&

\(L\)
\\


linear
acceleration
&

\(\frac{V^{2}}{L}\)
\\


linear
velocity
&

\(V\)
\\


mass
&

\(\frac{L^{3} \rho}{2}\)
\\


moment
&

\(\frac{L^{3} V^{2} \rho}{2}\)
\\


time
&

\(\frac{L}{V}\)
\\


volume
&

\(L^{3}\)
\\
\hline

\end{tabular}


\end{table}

\subsection{The propeller model}
\label{\detokenize{02.10_propeller_model:the-propeller-model}}\label{\detokenize{02.10_propeller_model::doc}}
\sphinxAtStartPar
A propeller model is developed based on Manoeuvring Modeling Group (MMG) model \cite{yasukawa_introduction_2015-1} where the thrust is expressed as:
\begin{equation}\label{equation:02.10_propeller_model:eqT}
\begin{split}\displaystyle thrust = D^{4} K_{T} n^{2} \rho\end{split}
\end{equation}
\sphinxAtStartPar
and the thrust coefficient \(K_T\) is modelled as a second order polynomial:
\begin{equation}\label{equation:02.10_propeller_model:eqkt}
\begin{split}\displaystyle K_{T} = J^{2} k_{2} + J k_{1} + k_{0}\end{split}
\end{equation}
\sphinxAtStartPar
The advance ratio \(J\) is calculated as:
\begin{equation}\label{equation:02.10_propeller_model:eqJ}
\begin{split}\displaystyle J = \frac{u \left(1 - w_{p}\right)}{D n}\end{split}
\end{equation}
\sphinxAtStartPar
where \(D\) is propeller diameter, \(n\) is propeller speed and \(w_p\) is the wake fraction at an oblique inflow to the propeller from the drift angle and the yaw rate. A semi\sphinxhyphen{}empirical formula for \(w_p\) is provided in the MMG model. As an alternative, a simple polynomial is proposed in \autoref{equation:02.10_propeller_model:eqpropellermodel}.
\begin{equation}\label{equation:02.10_propeller_model:eqpropellermodel}
\begin{split}\displaystyle w_{p} = C_{1} \delta + C_{2} \delta^{2} + C_{3} \beta_{p}^{2} + C_{4} u + w_{p0}\end{split}
\end{equation}
\sphinxAtStartPar
\(w_p\) is modeled as a function of rudder angle \(\delta\), to include wake influence from the rudder and ship speed \(u\), to include a speed dependency. The influence from drift angle \(\beta\) and yaw rate \(r\) is expressed by \(\beta_p\) in \autoref{equation:02.10_propeller_model:eqbetap}.
\begin{equation}\label{equation:02.10_propeller_model:eqbetap}
\begin{split}\beta_p=\beta - \frac{r}{V} \cdot x_p \end{split}
\end{equation}
where \(x_p\) is the propeller longitudinal position and \(w_{p0}\) is the regular Taylor wake fraction, applicable to straight ahead steaming with no rudder angle. Similar to the MMG propeller model, two sets of parameters \(C_1\)-\(C_4\) should be used in the propeller model depending on the sign of \(\beta_p\).
%%%%%%%%%%%%%%%%%%%%%%%%%%%%%%%
%%%%%%%%%%%%%%%%%%%%%%%%%%%%%%%
\chapter{Methods\label{ch:methods}}
%%%%%%%%%%%%%%%%%%%%%%%%%%%%%%%
The system identification of ship dynamics can be simplified into parameter identification if parameterized physical models can be assumed. Parameter Identification Techniques (PIT) for roll motion an manoeuvring is presented in \ref{sec:PIT_roll} and \ref{sec:PIT_VMM}. The system identification can be performed by selecting the best model from a collection of candidate models.

\section{Roll damping Parameter Identification} \label{sec:PIT_roll}
\noindent The PIT can be applied to identify the roll damping parameters ($B_1$, $B_2$, $B_3$) and stiffness parameters ($C_1$, $C_3$, $C_5$) in the parameterized roll motion models in Eq.\ref{eq:roll_decay_equation_himeno_linear}, Eq.\ref{eq:roll_decay_equation_himeno_quadratic_b} and Eq.\ref{eq:roll_decay_equation_cubic}. These equations do not have unique solutions, considering that the whole equations can be multiplied by an arbitrary factor to obtain new valid solutions. The inertia is therefore excluded, to obtain unique solutions. This is achieved by normalizing the equations by the total roll inertia $A_{44}$.
The normalized damping and stiffness parameters identified by a PIT can be expressed in dimensional units by multiplication with the normalization factor $A_{44}$. If $A_{44}$ is not known before hand, it can be calculated using Eq.\ref{eq:A_44_eq} \cite{piehl_ship_2016}, assuming that the meta center height $GM$ is known.
\begin{equation} \label{eq:A_44_eq}
A_{44} = \frac{GM g m}{\omega_{0}^{2}}
\end{equation}

\noindent The frequency $\omega_0$ can be obtained with Fast Fourier Transform (FFT) of the roll signal. 

Two different PIT methods have been investigated: the ``derivation approach'' (referred to as PIT in \parencite{imo_1200_2006}) and the ``integration approach'' which is similar to what \parencite{soder_assessment_2019} used. In the derivation approach the first and second roll time derivatives are calculated numerically so that the parameters in the models are the only unknowns. A least squares fit is applied on the roll motion equation to identify any parameter, including nonlinear or frequency parameter. In the integration approach, the parameters are found by solving a nonlinear problem using the least-square method. This approach requires that an ordinary differential equation to be solved for many estimated sets of parameters until the solution converges.

\section{VMM Parameter Identification} \label{sec:PIT_VMM}
In this procedure, a VMM is used to solve the reversed manoeuvring problem, such as predicting unknown forces from known ship manoeuverability. The hydrodynamic derivatives in the VMM can be identified with regression of the force polynomials on forces predicted with inverse dynamics. The Ordinary Least Square (OLS) method regresses the hydrodynamic derivatives. 

\subsection{Inverse dynamics and regression}
\label{\detokenize{03.01_inverse_dynamics:inverse-dynamics-and-regression}}\label{\detokenize{03.01_inverse_dynamics::doc}}
\sphinxAtStartPar
Each manoeuvring model has some hydrodynamic functions \(X_D(u,v,r,\delta,thrust)\), \(Y_D(u,v,r,\delta,thrust)\), \(N_D(u,v,r,\delta,thrust)\) that are defined as polynomials. The hydrodynamic derivatives in these polynomials can be identified with force regression of measured forces and moments. The measured forces and moments are usually taken from Captive Model Tests (CMT), Planar Motion Mechanism (PMM) tests or Virtual Captive Tests (VCT). When the ship is free in all degrees of freedom, as in the present model tests, only
motions are recorded however. Hence, forces and moments causing ship motions need to be estimated by
solving the inverse dynamics problem.
The inverse dynamics is solved by restructuring the system equation (\autoref{equation:02.01_VMMs:eqqsystem}) to get the hydrodynamics functions on the left-hand side. If the mass and inertia of the ship including added masses: \(X_{\dot{u}}\), \(Y_{\dot{v}}\), \(Y_{\dot{r}}\), \(N_{\dot{v}}\) and \(N_{\dot{r}}\), are known, the forces in Prime system can be calculated using \autoref{equation:03.01_inverse_dynamics:eqxd}, \autoref{equation:03.01_inverse_dynamics:eqyd} and \autoref{equation:03.01_inverse_dynamics:eqnd}.
\begin{equation}\label{equation:03.01_inverse_dynamics:eqxd}
\begin{split}\displaystyle \operatorname{X_{D}'}{\left(u',v',r',\delta,thrust' \right)} = - X_{\dot{u}}' \dot{u}' + \dot{u}' m' - m' r'^{2} x_{G}' - m' r' v'\end{split}
\end{equation}\begin{equation}\label{equation:03.01_inverse_dynamics:eqyd}
\begin{split}\displaystyle \operatorname{Y_{D}'}{\left(u',v',r',\delta,thrust' \right)} = - Y_{\dot{r}}' \dot{r}' - Y_{\dot{v}}' \dot{v}' + \dot{r}' m' x_{G}' + \dot{v}' m' + m' r' u'\end{split}
\end{equation}\begin{equation}\label{equation:03.01_inverse_dynamics:eqnd}
\begin{split}\displaystyle \operatorname{N_{D}'}{\left(u',v',r',\delta,thrust' \right)} = I_{z}' \dot{r}' - N_{\dot{r}}' \dot{r}' - N_{\dot{v}}' \dot{v}' + \dot{v}' m' x_{G}' + m' r' u' x_{G}'\end{split}
\end{equation}
\sphinxAtStartPar
An example of forces calculated with inverse dynamics from motions in a turning circle test can be seen in \hyperref[\detokenize{03.01_inverse_dynamics:fig-inverse}]{Fig.\@ \ref{\detokenize{03.01_inverse_dynamics:fig-inverse}}}. The forces have been converted to SI units.

\begin{figure}[H]
    \centering
    \includegraphics[width=\textwidth]{kappa/images/1.pdf}
    \caption{Example of forces and moments calculated with inverse dynamics on data from a turning circle test.}
    \label{\detokenize{03.01_inverse_dynamics:fig-inverse}}
\end{figure}

\section{Data cleaning}
It is possible to do an exact parameter identification on perfect (simulated) data with no noise (see Paper \ref{pap:pit}). However, such data from physical experiments does not exist in reality. The measured data will always contain process noise and measurement noise. In order to mitigate this, the data is preprocessed using an Extended Kalman filter (EKF) and Rauch Tung Striebel (RTS) smoother which are both presented below.

EKF is an extension of the Kalman Filter (KF) to work on nonlinear systems such as the VMMs. The basic idea is that noise can be disregarded if it does not make sense from a physical point of view. If noisy measurement data were perfectly correct, this would mean that the ship has many vibrations that must have originated from tremendous forces, considering the large mass of the ship. The prior understanding of the dynamics suggests that these forces are not present. Therefore, the noise should be considered as measurement noise and should be removed. Low-pass filtering is a common way to remove noise, where motions above some cut-off frequencies are regarded as unphysical measurement noise. The problem with low-pass filter is that it is hard to know what cut-off frequency to choose, either too low: removing part of the signal, or too high: keeping some unfiltered measurement noise in the data. The Kalman filter has a system model that continuously estimates the system’s state that runs in parallel with the measurement data. The filter estimates the current state as a combination of the measurement data and the system model estimate based on belief in the data and the model. If the data has low noise, the estimate turns toward that data. Conversely, if the model gives very good predictions, then that estimate turns towards the model.
The system’s inverse dynamics require the entire states, including positions, velocities, and accelerations, to be known. Only positions are known from the measurements, which means that velocities and accelerations are hidden states that the EKF should estimate.

\section{Iteration}
After choosing a proper VMM model to describe a ship’s manoeuvring performance, the coefficients in the VMM can be estimated by the proposed PIT method in \hyperref[\detokenize{01.01_method:overview}]{Fig.\@ \ref{\detokenize{01.01_method:overview}}}.
The measurement noise needs to be removed if the regression of hydrodynamic derivatives in the VMM should work well. However, filtering with the EKF also needs an accurate VMM as the system model. Therefore the accurate VMM is both the input and output of the PIT. The system model VMM in the EKF is guessed to solve this dilemma. A linear VMM with hydrodynamic derivatives estimated with semi\sphinxhyphen{}empirical formulas is used as the initial guess. Once the regressed VMM has been obtained, the PIT can be rerun using the regressed VMM as the system model in the EKF, to obtain an even better VMM. This procedure can be repeated several times for improved accuracy. Using semi\sphinxhyphen{}empirical formulas for the initially guessed VMM adds prior knowledge about the ship dynamics to the regression. When used with the recursive EKF, this method is an innovation compared to other PIT methods. An example with simulation results from the steps in the iterative EKF is shown in \hyperref[\detokenize{01.01_method:iterations}]{Fig.\@\ref{\detokenize{01.01_method:iterations}}}.

\begin{figure}[h]
    \centering
    \includegraphics[width=\textwidth]{kappa/images/method.png}
    \caption{Flow chart over the proposed PIT}
    \label{\detokenize{01.01_method:overview}}
\end{figure}
\begin{figure}[h]
    \centering
    \includegraphics[width=\textwidth]{kappa/images/0.pdf}
    \caption{Simulation with: initial model, first and second iteration of the PIT}
    \label{\detokenize{01.01_method:iterations}}
\end{figure}

%%%%%%%%%%%%%%%%%%%%%%%%%%%%%%%
%%%%%%%%%%%%%%%%%%%%%%%%%%%%%%%
\chapter{Results\label{ch:results}}
%%%%%%%%%%%%%%%%%%%%%%%%%%%%%%%
This chapter presents a summary of the appended papers, including research activities
and a selection of the important results, and highlights the main achievements.

\section{Summary Paper \ref{pap:rolldamping}}
\subsection*{"\nameref{pap:rolldamping}"}
In Paper \ref{pap:rolldamping}, time series data from 250 roll decay tests (see Fig. \ref{fig:ship_types}) assembled from the Maritime Dynamics Laboratory at SSPA Sweden AB (\href{www.sspa.se}{www.sspa.se}) are used to investigate the roll motion model. 

\begin{figure}[H]
    \centering
    \includegraphics[width=0.5\columnwidth]{kappa/images/ship_types.eps}
    \caption{Number of tests per ship type}
    \label{fig:ship_types}
\end{figure}

\noindent Parameters in the linear (Eq.\ref{eq:roll_decay_equation_himeno_linear}), quadratic (Eq.\ref{eq:roll_decay_equation_himeno_quadratic_b}) and cubic (Eq.\ref{eq:roll_decay_equation_cubic}) roll motion model are identified using the roll motion PIT (see section \ref{sec:PIT_roll}). The quadratic damping model has almost the same accuracy as the cubic model and is therefore sufficient to reproduce most of the roll decay tests. The ''integration approach'' (see section \ref{sec:integration_approach}) to PIT roll motion, produces the most accurate models compared to the ''derivation approach'' (see section \ref{sec:derivation_approach}).

The generic roll damping model (see section \ref{sec:genericrolldampingmodel}) is also developed in Paper \ref{pap:rolldamping}. The roll damping parameters identified from the 250 roll decay tests is used to fit a roll damping prediction model for modern ships using grey-box modelling. The black-box correction model of the output components from the SI method are shown in (Eq.\ref{eq:polynom_correction}),
\begin{equation} \label{eq:polynom_correction}
\hat{B_{e}} = 1.106 \hat{B_{BK}} - 0.9124 \hat{B_{E}} + 4.282 \hat{B_{F}} + 0.7457 \hat{B_{L}} + 0.1844 \hat{B_{W}} + 0.004999 \phi_{a} - 0.0005097
\end{equation}


\noindent Large corrections of the skin friction damping $\hat{B_F}$ and wave damping $\hat{B_W}$ are suggested by this expression. This is because the SI method is not very accurate for this dataset, where most of the ships in the dataset exceed the limits of the method. A pure black-box model is also devloped in Paper \ref{pap:rolldamping} (see Eq.\ref{eq:polynom_complex}),
\begin{equation} \label{eq:polynom_complex}
\begin{aligned} 
 \hat{B_{e}} = - 0.02578 A_{0} V - 0.02705 BK_{B} V + \\ 
 0.008993 BK_{L} V - 0.03191 C_{b} V - 0.2028 OG V + \\ 
 0.003472 V^{2} + \\ 
 0.004234 V \hat{\omega_{0}} - 0.002591 V \phi_{a} - 0.008384 V beam + \\ 
 0.05048 V + \\ 
 0.007814 \hat{\omega_{0}}^{2} + \\ 
 0.03882 \hat{\omega_{0}} \phi_{a} - 0.001069 \\ 
 \end{aligned}
\end{equation}


\noindent The grey-box model and the black-box model above, have about the same accuracy when performing cross-validation on the roll damping dataset.

\section{Summary Paper \ref{pap:daiyong}}
\subsection*{"\nameref{pap:daiyong}"}
Least Square Support Vector Regression (LS-SVR) \cite{brereton_support_2010} is used in Paper \ref{pap:daiyong} to identify the parameters in an Abkowitz Vessel Manoeuvring Model (AVMM) \cite{abkowitz_ship_1964}.  
The data is taken from experimental tests on a lake using a ship model with a scale of 50:1. The configuration of sensors and equipment for the experiment is shown in Fig.\ref{fig:cthmodel}.  
\begin{figure}[H]
    \centering
    \includegraphics[width=\textwidth]{kappa/images/cth_model.png}
    \caption{Configuration of sensors and equipment for the experimental tests.}
    \label{fig:cthmodel}
\end{figure}
\noindent The hydrodynamic derivatives of the AVMM are identified almost perfectly when applied on data from simulations with MSS toolbox Mariner \cite{tristan_matlab_2009}. The PIT does however not work at all when applied on the data obtained from the lake experiments. The PIT is very sensitive to noise due to the differentiation that needs to be conducted to calculate velocities and yaw rate from the measured position and heading. The PIT works better if the data is first cleaned using a proposed preprocessing algorithm together with a Kalman Filter (KF). The simulations with the identified model and the experiments are however still not in very good agreement.     

\section{Summary Paper \ref{pap:pit}}
\subsection*{"\nameref{pap:pit}"}
A method for System Identification of ship manoeuvring dynamics is developed in Paper \ref{pap:pit}. It is shown that the hydrodynamic derivatives within a VMM can be identified exactly at ideal conditions with no measurement noise and a perfect estimator.

It is shown that the proposed prepossessing of measurement data with EKF + RTS run in iteration with initial guess from semi-empirical formulas, is better than using low-pass filters for cleaning.

The new method can predict Turning circles with less than 5 \% error in advance and tactical diameter for the wPCC and KVLCC2 test cases, which should be considered sufficient considering the margin to the corresponding limits in the IMO standard for both ships.
%%%%%%%%%%%%%%%%%%%%%%%%%%%%%%%
%%%%%%%%%%%%%%%%%%%%%%%%%%%%%%%
\chapter{Conclusions\label{ch:conclusions}}
%%%%%%%%%%%%%%%%%%%%%%%%%%%%%%%
The main findings and conclusions are presented below with respect to the research activities described in Section \ref{sec:motivation}.

\subsubsection*{Develop a method to handle noise}
The PIT ''integration approach'' produced better models than the ''derivation approach'' for the roll motion models in Paper \ref{pap:rolldamping}. The ''integration approach'' is very slow and relies on optimization that may or may not converge.
The numerical differentiation that was used in Paper \ref{pap:rolldamping} to estimate the velocities and accelerations, is believed to be the main cause of the poor performance of the much faster and more reliable ''derivation approach''. A similar issue was also encountered in Paper \ref{pap:daiyong}, where the identified hydrodynamic derivatives were very sensitive to the choice of the regularisation factor of the LS-SVR.
The iterative EKF + RTS smoother proposed in Paper \ref{pap:pit}, seems to solve these issues.

\subsubsection*{Propose a methodology to choose the appropriate and robust VMM for extrapolation}
A methodology to select a suitable VMM based on cross-validation is proposed in Paper \ref{pap:pit}. In order to investigate the model's ability to extrapolate and make predictions outside its training data, the validation set should have larger yaw rates, drift angles and rudder angles compared to the training set. The methodology was applied on the wPCC and KVLCC2 test cases, where turning circles where predicted with good accuracy on models trained on zigzag tests. 

\subsubsection*{Develop a PIT for three degrees of freedom (surge, sway and yaw)}
The challenges in identifying the hydrodynamic derivatives of the AVMM on noisy experimental data was shown in Paper \ref{pap:daiyong}. The challenges were mitigated in Paper \ref{pap:pit}, by improved preprocessing of the noisy data and by reduced multicollinearity of the VMM. The multicollinearity of the AVMM was reduced, by excluding hydrodynamic derivatives and the introduction of a propeller thrust model.

\subsubsection*{Develop grey-box model for prediction of modern ship roll             damping parameters}
The SI method, being the white-box physical model in the grey-box model in Paper \ref{pap:rolldamping} has about the same accuracy as the corresponding black-box model, which means that the white-box model is adding very little value, due to poor performance of SI method outside its limits.

\subsubsection*{Develop a PIT for roll motion for roll damping database}
The PIT identified the parameters in the quadratic roll motion model (Eq.\ref{eq:roll_decay_equation_himeno_quadratic_b}) with good accuracy for the 250 roll decay tests obtained from SSPA. Where the linear model (Eq.\ref{eq:roll_decay_equation_himeno_linear}) was ruled out as too simple and the cubic model (Eq.\ref{eq:roll_decay_equation_cubic}) as unnecessary complex.   


%%%%%%%%%%%%%%%%%%%%%%%%%%%%%%%
%%%%%%%%%%%%%%%%%%%%%%%%%%%%%%%
\chapter{Future work\label{ch:future_work}}
%%%%%%%%%%%%%%%%%%%%%%%%%%%%%%%
