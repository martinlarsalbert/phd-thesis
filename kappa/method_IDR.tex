\section{Parameter estimation from inverse dynamics} \label{sec:IDR}
Parameter estimation from CMT or VCT, as described in \autoref{sec:VCT}, is the classic way to identify parameters within a manoeuvring model. However, a model can also be identified from time series obtained with FRMTs or full-scale maneuvers. Instead of, as in the CMT or VCT, having direct measured forces, the forces can instead be estimated with inverse dynamics (see \autoref{sec:ID}). However, inverse dynamics has a basic problem in estimating the rudder forces, which disturbs the estimation of the other maneuvering coefficients \cite{arakiEstimatingManeuveringCoefficients2012}. It may be hard to tell where the forces are generated by just looking at the total force, which introduces a high multicollinearity between the hull and the rudder force during the maneuvers.
This problem can either be resolved by measuring the rudder force, which was done for the Optiwise test case in Paper \ref{pap:vct}. Otherwise, the rudder force needs to be estimated, which was investigated in Paper \ref{pap:physics} by introducing a semi-empirical rudder model (see \autoref{sec:semi-empirical}). The hull forces needed to regress the hull coefficients can be estimated by subtracting the rudder and propeller forces from the total damping forces according to \autoref{eq:X_H_VCT}--\autoref{eq:N_H_VCT}.



