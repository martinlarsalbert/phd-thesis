\section{Parameter estimation from inverse dynamics} \label{sec:IDR}
Parameter estimation from CT data (CMT or VCT), as described in \autoref{sec:VCT}, is the classic approach to identifying parameters within a manoeuvring model. However, a model can also be identified from time series FT data obtained with FRMTs or full-scale maneuvers. Rather than, as in the CMT or VCT, directly measuring forces, they can be estimated through the application of inverse dynamics (see \autoref{sec:ID}). However, inverse dynamics is inherently limited in estimating rudder forces, which affects the estimation accuracy of the other manoeuvring coefficients \cite{arakiEstimatingManeuveringCoefficients2012}. It may be difficult to determine where the forces are generated by solely considering the total force, which introduces a high multicollinearity between the hull and the rudder forces during the maneuvers.
This can be addressed by measuring the rudder force, as demonstrated in the Optiwise test case in Paper \ref{pap:vct}. Otherwise, the rudder force must be estimated, which was investigated in Paper \ref{pap:physics} by introducing a semi-empirical rudder model (see \autoref{sec:semi-empirical}). The hull forces needed to regress the hull coefficients can be estimated by subtracting the rudder and propeller forces from the total damping forces according to \autoref{eq:X_H_VCT}--\autoref{eq:N_H_VCT}.



