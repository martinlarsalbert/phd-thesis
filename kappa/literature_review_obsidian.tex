This literature review explores papers on the use of data-driven models identified from CT or FT data for the manoeuvring of ship. Papers on parametric, non-parametric, or hybrid models are first briefly introduced in \autoref{sec:parametric_models} and \ref{sec:non-parametric_models}. This introduction is followed by a more in-depth review of the models identified from CT or FT data as presented in \autoref{sec:CT} and \ref{sec:FT}.
Papers addressing the system identification of parametric models using simulated data have been deemed irrelevant for this thesis and, consequently, have not been included in this review.

\subsection{Parametric models} \label{sec:parametric_models}
Parametric model structures represent a class of grey-box models where parameterization is based on varying levels of physical insights (greyness) as described by classical manoeuvring models, such as the Nomoto model structure \cite{nomotoSteeringQualitiesShips1957}, the Abkowitz model structure \cite{abkowitzShipHydrodynamicsSteering1964}, and the Norrbin model structure \cite{norrbinTheoryObservationsUse1971}. The Nomoto and Abkowitz model structures are both pure mathematical models. The Nomoto model structure characterizes the yaw dynamics of ships and is particularly useful for predicting a ship's response to steering inputs, for instance in autopilot applications. \textcite{tzengFUNDAMENTALPROPERTIESLINEAR1999} investigated the fundamental properties associated with the Nomoto model. The Abkowitz model structure describes the total forces acting on a ship in three degrees of freedom as a truncated third-order Taylor expansion. \textcite{norrbinTheoryObservationsUse1971} added more physical insight into the model structures; first in the use of second-order modulus functions to model the nonlinearities with coefficients, such as $N_{v|v|}$ and $N_{v|r|}$, later to be replaced by the cross flow drag principle \cite{fossenHandbookMarineCraft2011}. More physical insights were added in the modular model structure of the Manoeuvring Modeling Group (MMG) model structure \cite{ogawaMathematicalModelManoeuvring1978,inouePracticalCalculationMethod1981,yasukawaIntroductionMMGStandard2015}. Instead of describing the total force acting on the ship, the model structure was divided into sub models for the propeller, rudder, and hull.

\subsection{Non-parametric or hybrid models} \label{sec:non-parametric_models}
Advancements in machine learning have enabled the expression of ship manoeuvring through non-parametric models. Non-parametric models may be considered black-box models, which \textcite{ljungPerspectivesSystemIdentification2010} describes as flexible function surfaces.
Examples of non-parametric models include various types of neural networks \cite{rajeshSystemIdentificationNonlinear2008,heBlackBoxModelingShip2020,heNonparametricModelingShip2022}, support vector machine regression (SVM) \cite{chenOnlineModelingPrediction2023,zihaowangKernelbasedSupportVector2020}, or Gaussian process models (GP) \cite{zhangLocallyWeightedNonParametric2021,xueIdentificationPredictionShip2021,xueOnlineIdentificationShip2022}.  

Non-parametric models provide flexibility, enabling them to represent a wide range of hydrodynamic relationships, whereas parametric models may struggle to capture hydrodynamics accurately in certain cases. However, if the assertion by \textcite{revestidoherreroTwostepIdentificationNonlinear2012} is correct, \say{the parametric model structures provide a suitable set of models in which it can be assumed that a true model belongs}, this means that the physical insights from the parametric models might also add valuable prior information to the system identification.

Hybrid models have been developed to integrate parametric and non-parametric approaches. \textcite{wangIncorporatingApproximateDynamics2021} propose a framework in which the foundation is set by the best available parametric model, which is then refined with a neural network. \textcite{nielsenMachineLearningEnhancement2022} use a similar approach. \textcite{dongMathdataIntegratedPrediction2023a} combine an MMG model with an SVM corrector.

\subsection{Captive test papers} \label{sec:CT}
Identifying a manoeuvring model from CMT or VCT is a great challenge, or as \textcite{sutuloAlgorithmOfflineIdentification2014} says that \say{All practical manoeuvring mathematical models are highly schematised and although in principle can be tuned to provide a satisfactory reproduction of the true motion, there are no simple theoretical methods for estimating their parameters}.

\begin{table}[h]
    \centering
    \caption{Averaged overhoot results from \cite{liuPredictionsShipManeuverability2018}}
    \label{tab:liu2018}
    \pgfplotstabletypeset[col sep=comma, column type=c,
        columns/Overshoot/.style={column type=c,string type},
        %columns/Parameter/.style={column type=l,string type},
        %columns/Unit/.style={column type=l,string type,column name=~},
        %columns/Description/.style={column type=l,string type},
        %columns/Value/.style={column type=r, column name=~},
        every head row/.style={before row=\hline,after row=\hline},
        every last row/.style={after row=\hline}
    ]{tables/liu2018.csv}
\end{table}

\subsection{Free running tests (system identification)} \label{sec:FT} % (Parametric models)
Models are often tuned manually before being implemented in bridge simulators, although such approaches are rarely mentioned in the literature \cite{sutuloAlgorithmOfflineIdentification2014}. More structured approaches to system identification for parametric models typically involve some form of Kalman filter (KF) in the process. The use of a KF combined with maximum likelihood estimation was proposed in 1976 by \textcite{astromIdentificationShipSteering1976} to identify a linear manoeuvring model that utilized manually recorded data aboard the Atlantic Song freighter. Currently, the extended Kalman filter (EKF) is the predominant system identification method. It is used to estimate ship state from noisy data during manoeuvres, but it can also estimate model parameters as shown by \textcite{shiIdentificationShipManeuvering2009} and \textcite{pereraSystemIdentificationNonlinear2015}. In this approach, the parameters are updated continuously so that the model can adapt over time in real-time. This approach is quite challenging for larger model structures where many parameters need to be simultaneously estimated. Instead, \textcite{yoonIdentificationHydrodynamicCoefficients2003} introduced an estimation-before-modeling technique  (two-step approach), also used by \textcite{revestidoherreroTwostepIdentificationNonlinear2012}, where only the state of the ship is estimated by EKF and the model parameters are identified by another method.  
Some studies in the literature have not used the EKF: \textcite{tianoMultivariableIdentificationShip1997} used a random search minimization method and also included roll motion in the system identification; \textcite{casadoIdentificationNonlinearShip2005} used the backstepping procedure and the tuning design method; and \textcite{millerShipModelIdentification2021} used a genetic algorithm to identify parameters. Additionally, 
\textcite{chillcceDatadrivenSystemIdentification2023} used numerical calculations of velocities and accelerations by applying the Savitzky–Golay numerical differentiation method \cite{ahnertNumericalDifferentiationExperimental2007}, instead of the EKF. They used an Euler equation-based numerical approach \cite{elmoctarEfficientAccurateApproach2022}  to determine the zero-frequency added masses and a constrained least-squares algorithm for linear regression, akin to the approach by \textcite{arakiEstimatingManeuveringCoefficients2012}.

Multicollinearity in statistical modeling describes a scenario where two or more predictor variables are highly correlated, making it difficult to isolate the individual effects of each predictor on the dependent variable. This issue is particularly relevant in the field of ship manoeuvring modeling, where numerous hydrodynamic coefficients and parameters are involved.
The higher the correlation between the regression variables, or the stronger the multicollinearity, the more difficult it is to identify the regression coefficients separately \cite{yoonIdentificationHydrodynamicCoefficients2003}.
\textcite{wangQuantifyingMulticollinearityShip2018} analyzed the effects of multicollinearity on parameter drift in system identification.  They showed that when predictor variables are highly correlated, the estimates of the model parameters can become unstable and sensitive to small changes in the data. in their work, the variance inflation factor (VIF) was used to quantify the severity of multicollinearity in ship manoeuvring models. VIF measures how much the variance of a regression coefficient is inflated due to multicollinearity; a high VIF indicates a high level of multicollinearity.
Multicollinearity can be partially mitigated by pre-processing the data.
\textcite{luoParameterIdentificationShip2016} addressed the issue of parameter identifiability in ship manoeuvring modeling. This study aimed to reconstruct samples and reduce multicollinearity by employing methods such as the difference method and the additional signal method, thereby improving the feasibility of system identification.
\textcite{xuUncertaintyAnalysisHydrodynamic2019} introduced methods to address the uncertainty caused by multicollinearity, such as truncated singular value decomposition and Tikhonov regularization. These techniques help in stabilizing the parameter estimates and improving the robustness of the model.

Model structure selection is a more pragmatic approach to addressing multicollinearity, by reducing the number of parameters in the model, ensuring that they are identifiable from the available data. \textcite{luoParameterIdentificationShip2016} reduced the number of parameters based on physical considerations. \textcite{costaRobustParameterEstimation2021} applied truncated singular value decomposition, while\textcite{liuPhysicsinformedIdentificationMarine2024} used sparse identification of nonlinear dynamics (SINDy, as in \cite{bruntonDiscoveringGoverningEquations2016}, to reduce the number of model parameters. 
\textcite{abkowitzMEASUREMENTHYDRODYNAMICCHARACTERISTICS1980} addressed multicollinearity through elimination of “inconvenient” terms; however, this approach could lead to models with limited applicability, as certain regression terms may only become significant under specific conditions, such as sailing in wind. This is perhaps the key limitation of model structure selection – that the generalization of a model may suffer when parameters are excluded.

The most effective approach to mitigating multicollinearity is to obtain more informative data with sufficient persistence of excitation. This requires input signals in system identification to be rich in frequency content, ensuring that all system modes are adequately excited. This ensures that the system's response contains enough information to uniquely identify the system parameters. Without persistence of excitation, the identified model may not accurately represent the ship's behaviour in all scenarios.

\textcite{yoonIdentificationHydrodynamicCoefficients2003} discussed the importance of designing experiments that ensure persistence of excitation. They suggested using specific input scenarios that maximize the information content of the data, such as D-optimal designs. An optimal experimental design is easier to obtain for captive tests, where the state of the ship can be varied freely. Although \textcite{wangOptimalDesignExcitation2020} and \textcite{millerShipModelIdentification2021} suggested that a pseudo-random sequence (PRS) can be used for free running tests, data from these kinds of tests are very rare. A model basin is too small, and full-scale tests of this kind are also very rare. 
Data for the mandatory zigzag and turning circle standard manoeuvres \cite{imoStandardsShipManoeuvrability2002} are much more readily available, which explains the frequent use of standard manoeuvres for system identification in many studies in the literature. However, these manoeuvres are not sufficiently rich to guarantee reliable estimation of all regression coefficients \cite{sutuloAlgorithmOfflineIdentification2014}, which poses a substantial challenge for system identification.

%\subsection{System identification with non-parametric models}
%The multicollinearity is not a problem for the non-parametric models. However, informative data with persistence of excitation is till a requirement since even the most clever model, cannot see behind corners. 