This literature review covers papers on data-driven models for the manoeuvring of ships identified from CT or FT data. Papers on parametric, non-parametric, or hybrid models are first briefly introduced in \autoref{sec:parametric_models} and \ref{sec:non-parametric_models}. After that, a more in-depth review of the models identified from CT or FT data is presented in \autoref{sec:CT} and \ref{sec:FT}.
Papers addressing the system identification of parametric models using simulated data have been deemed irrelevant for this thesis and, consequently, have not been included in this review.

\subsection{Parametric models} \label{sec:parametric_models}
Parametric model structures are a kind of grey-box models that are parameterizations based on various levels of physical insights (greyness) described by the classical manoeuvring models such as the Nomoto model structure \cite{nomotoSteeringQualitiesShips1957}, the Abkowitz model structure \cite{abkowitzShipHydrodynamicsSteering1964}, and the Norrbin model structure \cite{norrbinTheoryObservationsUse1971}. The Nomoto and Abkowitz model structures are pure mathematical models. The Nomoto model structure describes the ship's yaw dynamics and is particularly useful for predicting a ship's response to steering inputs, for instance in autopilot applications. \textcite{tzengFUNDAMENTALPROPERTIESLINEAR1999} investigated the fundamental properties associated with the Nomoto model. The Abkowitz model structure describes the total forces acting on the ship in three degrees of freedom as a truncated 3rd order Taylor expansion. \textcite{norrbinTheoryObservationsUse1971} added more physical insight into the model structures; first in the use of 2nd order modulus functions  to model the nonlinearities with coefficients such as $N_{v|v|}$ and $N_{v|r|}$, later to be replaced by the cross flow drag principle \cite{fossenHandbookMarineCraft2011}. More physical insights where added in the modular model structure of the Manoeuvring Modeling Group (MMG) model structure \cite{ogawaMathematicalModelManoeuvring1978,inouePracticalCalculationMethod1981,yasukawaIntroductionMMGStandard2015}. Instead of describing the total force acting on the ship, the model structure was instead divided into sub models for the propeller, rudder, and hull.

\subsection{Non-parametric or hybrid models} \label{sec:non-parametric_models}
The advancements of machine learning has enabled the possibility to express the ship manoeuvring with non-parametric models. The non-parametric models can also be referred as black-box models, which \textcite{ljungPerspectivesSystemIdentification2010} describes as flexible function surfaces.
Examples of non-parametric models are various types of neural networks \cite{rajeshSystemIdentificationNonlinear2008,heBlackBoxModelingShip2020,heNonparametricModelingShip2022}, support vector machine regression (SVM) \cite{chenOnlineModelingPrediction2023,zihaowangKernelbasedSupportVector2020}, or Gaussian process models (GP) \cite{zhangLocallyWeightedNonParametric2021,xueIdentificationPredictionShip2021,xueOnlineIdentificationShip2022}.  

The non-parametric models have a potential benefit with their flexibility, so that any kind of hydrodynamic relationship can be described; in contrast to the parametric models that might be incapable of describing the hydrodynamics correctly for some cases. However, if the statement made by \textcite{revestidoherreroTwostepIdentificationNonlinear2012} is true \say{the parametric model structures provide a suitable set of models in which it can be assumed that a true model belongs}, this means that the physical insights from the parametric models might also add valuable prior information to the system identification.

Hybrid models have been developed to bridge the parametric models and non-parametric models. \textcite{wangIncorporatingApproximateDynamics2021} proposes a foundation as the best available parametric model which is corrected with a neural network. \textcite{nielsenMachineLearningEnhancement2022} used a similar approach. \textcite{dongMathdataIntegratedPrediction2023a} uses the MMG model together with a SVM corrector.

\subsection{Captive test papers} \label{sec:CT}
Identifying a manoeuvring model from CMT or VCT is a great challenge, or as \textcite{sutuloAlgorithmOfflineIdentification2014} says that \say{All practical manoeuvring mathematical models are highly schematised and although in principle can be tuned to provide a satisfactory reproduction of the true motion, there are no simple theoretical methods for estimating their parameters}.

\begin{table}[h]
    \centering
    \caption{Averaged overhoot results from \cite{liuPredictionsShipManeuverability2018}}
    \label{tab:liu2018}
    \pgfplotstabletypeset[col sep=comma, column type=c,
        columns/Overshoot/.style={column type=c,string type},
        %columns/Parameter/.style={column type=l,string type},
        %columns/Unit/.style={column type=l,string type,column name=~},
        %columns/Description/.style={column type=l,string type},
        %columns/Value/.style={column type=r, column name=~},
        every head row/.style={before row=\hline,after row=\hline},
        every last row/.style={after row=\hline}
    ]{tables/liu2018.csv}
\end{table}

\subsection{Free running tests (system identification)} \label{sec:FT} % (Parametric models)
Often the models are tuned manually before being implemented in bridge simulators although such approaches are rarely even mentioned in the literature \cite{sutuloAlgorithmOfflineIdentification2014}. More structured ways of system identification of parametric models typically involve some kind of Kalman filter (KF) in the process. KF combined with maximum likelihood estimation was proposed in 1976 by \textcite{astromIdentificationShipSteering1976} to identify a linear manoeuvring model that utilized manually recorded data aboard the Atlantic Song freighter. The extended Kalman filter (EKF) is the predominant system identification method. It is used to estimate the states of the ship from noisy data during the manoeuvres, but it can also estimate the model parameters as shown by \textcite{shiIdentificationShipManeuvering2009} and \textcite{pereraSystemIdentificationNonlinear2015}. With this approach, the parameters are updated continuously so that the model can adopt over time in an online manner. This approach is quite challenging for larger model structures where many parameters need to be estimated at the same time. Instead, \textcite{yoonIdentificationHydrodynamicCoefficients2003} introduced an estimation before modelling technique  (two-step approach) also used by \textcite{revestidoherreroTwostepIdentificationNonlinear2012}, where only the state of the ship is estimated by EKF and the model parameters are identified by some other method.  
There are also a few papers that do not use the EKF. \textcite{tianoMultivariableIdentificationShip1997} used a random search minimization method and also included roll motions in the system identification. \textcite{casadoIdentificationNonlinearShip2005} used the backstepping procedure and the tuning design method and \textcite{millerShipModelIdentification2021} used a genetic algorithm to identify parameters. 
\textcite{chillcceDatadrivenSystemIdentification2023} used numerical calculations of velocities and accelerations using the Savitzky–Golay numerical differentiation method \cite{ahnertNumericalDifferentiationExperimental2007}, instead of the EKF. They used an Euler equation-based numerical approach \cite{elmoctarEfficientAccurateApproach2022}  to determine the zero-frequency added masses and a constrained least squares algorithm for the linear regression similar to \textcite{arakiEstimatingManeuveringCoefficients2012}.

Multicollinearity refers to a situation in statistical modelling where two or more predictor variables are highly correlated, making it difficult to isolate the individual effects of each predictor on the dependent variable. This issue is particularly relevant in the field of ship manoeuvring modelling, where numerous hydrodynamic coefficients and parameters are involved.
The higher the correlation between the regression variables, or the stronger the multicollinearity exists, the more difficult it is to identify the regression coefficients separately \cite{yoonIdentificationHydrodynamicCoefficients2003}.
\textcite{wangQuantifyingMulticollinearityShip2018} discussed how multicollinearity can lead to parameter drift in system identification. When predictor variables are highly correlated, the estimates of the model parameters can become unstable and sensitive to small changes in the data. They used the variance inflation factor (VIF) to quantify the severity of multicollinearity in ship manoeuvring models. VIF measures how much the variance of a regression coefficient is inflated due to multicollinearity. A high VIF indicates a high level of multicollinearity.
The multicollinearity can to some extent be handled by pre-processing the data.
\textcite{luoParameterIdentificationShip2016} addressed the issue of parameter identifiability in ship manoeuvring modelling. By employing methods such as difference method and additional signal method, the study aimed to reconstruct samples and reduce multicollinearity, thereby improving the feasibility of system identification.
\textcite{xuUncertaintyAnalysisHydrodynamic2019} introduced methods like truncated singular value decomposition and Tikhonov regularization to handle the uncertainty caused by multicollinearity. These techniques help in stabilizing the parameter estimates and improving the robustness of the model.

Model structure selection is a more pragmatic way to handle multicollinearity, by reducing the number of parameters in the model to be identifiable from the data at hand.  \textcite{luoParameterIdentificationShip2016} reduced the number of parameters based on physical considerations. \textcite{costaRobustParameterEstimation2021} used the truncated singular value decomposition, and \textcite{liuPhysicsinformedIdentificationMarine2024} used sparse identification of nonlinear dynamics (SINDy) \cite{bruntonDiscoveringGoverningEquations2016}, to reduce the number of model parameters. 
\textcite{abkowitzMEASUREMENTHYDRODYNAMICCHARACTERISTICS1980} was only able to fight the multicollinearity through elimination of “inconvenient” terms which, however, could lead to models with limited applicability as certain regression terms may only become significant in special conditions like sailing in wind. This is perhaps the biggest drawback with the model structure selection – that the generalization of a model may suffer when parameters are excluded.

The best option to mitigate multicollinearity is to get more informative data with persistence of excitation including conditions where the input signals used in system identification are sufficiently rich in frequency content to excite all the modes of the system. This ensures that the system's response contains enough information to uniquely identify the system parameters. Without persistence of excitation, the identified model may not accurately represent the ship's behaviour in all scenarios.

\textcite{yoonIdentificationHydrodynamicCoefficients2003} discuss the importance of designing experiments that ensure persistence of excitation. They suggest using specific input scenarios that maximize the information content of the data, such as D-optimal designs. An optimal experimental design is easier to obtain for captive tests, where the state of the ship can be varied freely. However, \textcite{wangOptimalDesignExcitation2020} and \textcite{millerShipModelIdentification2021} suggest that a pseudo-random sequence (PRS) can be used for free running tests.  However, data from these kind of tests are very rare. A model basin is too small, and full scale tests of this kind are also very hard to find. 
Data for the mandatory zigzag and turning circle standard manoeuvres \cite{imoStandardsShipManoeuvrability2002} are much easier to find, which is why almost all papers use standard manoeuvres for the system identification. However, these manoeuvres are not rich enough to guarantee reliable estimation of all regression coefficients \cite{sutuloAlgorithmOfflineIdentification2014}, which poses a great challenge for the system identification.

%\subsection{System identification with non-parametric models}
%The multicollinearity is not a problem for the non-parametric models. However, informative data with persistence of excitation is till a requirement since even the most clever model, cannot see behind corners. 