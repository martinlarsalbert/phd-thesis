We build mental models to understand the world. So does the helmsman of a ship, where the understanding of manoeuvring dynamics is crucial for a safe voyage. \say{Loosely speaking a model is a tool we use to answer questions about the system without having to do an experiment} \cite{ljungModelingIdentificationDynamic2021}. \say{The model is a relationship between observed quantities. In loose terms, a model allows for prediction of properties or behaviours of the object} \cite{ljungPerspectivesSystemIdentification2010}.
Mathematical model structures have been developed to give a more strict definition of the manoeuvring dynamics, to enable predictions with simulations. This literature review will cover papers about these model structures and ways to identify them from various types of data sources.

Beyond the helmsman's will of a safe voyage, there are formal manoeuvring requirements that ships greater than 100 meters must meet  \cite{imoStandardsShipManoeuvrability2002} that are ultimately demonstrated during the sea trials. Earlier assessments are often carried out before the ship has been build where free-running model tests (FRMT) are often recognized as the most reliable way \cite{ittcManeuveringCommitteeITTC2008}. However, there are other situations when building a mathematical model of the ship instead of a physical scale model is a better option.  There are problems with scale effects in the model tests that could potentially be handled in a mathematical model.  However, the largest benefit with mathematical models is the low computational costs, after that a mathematical model has been established; Numerous scenarios can be predicted in a simulator.

The model structures for manoeuvring can be categorized as either parametric models or non-parametric models. The hybrid models is a third category which combines parametric models with non-parametric models. A literature review of these three items follows in \autoref{sec:parametric_models} -- \autoref{sec:hybrid_models} below.

\subsection{Parametric models} \label{sec:parametric_models}
Parametric model structures are a kind of grey-box models that are parameterizations based on various levels of physical insights (greyness) described by the classical manoeuvring models such as the Nomoto model structure \cite{nomotoSteeringQualitiesShips1957}, the Abkowitz model structure \cite{abkowitzShipHydrodynamicsSteering1964}, and the Norrbin model structure \cite{norrbinTheoryObservationsUse1971}. The Nomoto and Abkowitz model structures are pure mathematical models. The Nomoto model structure describes the ship's yaw dynamics and is particularly useful for predicting a ship's response to steering inputs, for instance in autopilot applications. \textcite{tzengFUNDAMENTALPROPERTIESLINEAR1999} investigated the fundamental properties associated with the Nomoto model. The Abkowitz model structure describes the total forces acting on the ship in 3DOF as a truncated 3rd order Taylor expansion. \textcite{norrbinTheoryObservationsUse1971} added more physical insights into the model structures; Firstly in the use of 2nd order modulus functions  to model the nonlinearities with coefficients such as $N_{v|v|}$ and $N_{v|r|}$, later to be replaced by the cross flow drag principle \cite{fossenHandbookMarineCraft2011}. More physical insights where added in the modular model structure of the Maneuvering Modeling Group (MMG) model structure \cite{ogawaMathematicalModelManoeuvring1978,inouePracticalCalculationMethod1981,yasukawaIntroductionMMGStandard2015}. Instead of describing the total force acting on the ship, the model structure was instead divided into sub models for the propeller, rudder and hull.

\subsection{Non-parametric models} \label{sec:non-parametric_models}
The advancements of machine learning has enabled the possibility to express the ship manoeuvring with non-parametric models. The non-parametric models can also be referred as black-box models, which \textcite{ljungPerspectivesSystemIdentification2010} describes as flexible function surfaces.
Examples of non-parametric models are various types of neural networks \cite{rajeshSystemIdentificationNonlinear2008,heBlackBoxModelingShip2020,heNonparametricModelingShip2022}, support vector machine regression (SVM) \cite{chenOnlineModelingPrediction2023,zihaowangKernelbasedSupportVector2020}, or Gaussian process models (GP) \cite{zhangLocallyWeightedNonParametric2021,xueIdentificationPredictionShip2021,xueOnlineIdentificationShip2022}.  

The non-parametric models have a potential benefit with their flexibility, so that any kind of hydrodynamic relationship can be described; In contrast to the parametric models that might be incapable of describing the hydrodynamics correctly for some cases. However, if the statement made by \textcite{revestidoherreroTwostepIdentificationNonlinear2012} is true \say{the parametric model structures provide a suitable set of models in which it can be assumed that a true model belongs}, this means that the physical insights from the parametric models might also add valuable prior information to the system identification.

\subsection{Hybrid models} \label{sec:hybrid_models}
Hybrid models have been developed to bridge the parametric models and non-parametric models. \textcite{wangIncorporatingApproximateDynamics2021} proposes a foundation as the best available parametric model which is corrected with a neural network. \textcite{nielsenMachineLearningEnhancement2022} uses a similar approach. \textcite{dongMathdataIntegratedPrediction2023a} uses the MMG model together with a SVM corrector.

\subsection{Data}
Only data driven, or partly data driven models are considered in this review, which means that data, of some sort, is always needed for the identification of the models. There are mainly two types of data: either from captive test (CT) or free running test (FT). These data types divide the papers into two main categories. In the CT category, models are identified from forces and moments obtained with captive tests. Captive model tests (CMT) are the classical way for captive tests which can be conducted in various ways: with a XY-carriage, rotating arm, or planar motion mechanism (PMM). CMT can also be conducted with CFD in virtual captive tests (VCT). 

In the FT category of papers, models have been identified with system identification from free running tests, from either: model tests, full scale tests or in some cases direct CFD \cite{arakiEstimatingManeuveringCoefficients2012}.  However, the methods from the CT papers are generally more applicable in the virtual prototyping – when assessing the manoeuvring performance before ships are build. The FT methods, on the other hand, are generally more applicable for existing ships – in a digital twin context.   

\subsection{Captive test papers}
\textcolor{red}{ToDo...}

\subsection{Free running test papers}
System identification has been applied on both idealized data from simulations, or real data from either model tests or full scale measurements. The success of the system identification methods depend on both the chosen model structure and the quality of the data, in terms of measurement accuracy and amount of information. 
There are many papers in the literature that handle system identification of parametric models from simulated data. 
In these papers the model structure that generated data is already known before hand. This is not a relevant scenario, and the papers have therefore been excluded from this review. Or as \textcite{millerShipModelIdentification2021} formulated it: \say{we identify real objects, not its mathematical model}. However, papers that use non-parametric models on simulated data have been included in the review, since it is still relevant to study their ability to find the model structure from simulated data.

\subsection{System identification} % (Parametric models)
Often the models are tuned manually before being implemented in bridge simulators although such approaches are rarely even mentioned in the literature \cite{sutuloAlgorithmOfflineIdentification2014}. More structured ways of system identification of parametric models typically involve some kind of Kalman filter (KF) in the process. KF combined with maximum likelihood estimation was proposed in 1976 by \textcite{astromIdentificationShipSteering1976} to identify a linear manoeuvring model that utilized manually recorded data aboard the Atlantic Song freighter. The extended Kalman filter (EKF) is the predominant system identification method. It is used to estimate the states of the ship from noisy data during the manoeuvres, but it can also estimate the model parameters as shown by \textcite{shiIdentificationShipManeuvering2009} and \textcite{pereraSystemIdentificationNonlinear2015}. With this approach, the parameters are updated continuously so that the model can adopt over time in an online manner. This approach is quite challenging for larger model structures where many parameters need to be estimated on the same time. Instead, \textcite{yoonIdentificationHydrodynamicCoefficients2003} introduced an estimation before modelling technique  (two-step approach) also used by \textcite{revestidoherreroTwostepIdentificationNonlinear2012} where only the state of the ship is estimated by EKF and the model parameters are identified by some other method .  
There are also a few papers that do not use the EKF. \textcite{tianoMultivariableIdentificationShip1997} used a random search minimization method and also included roll motions in the system identification. \textcite{casadoIdentificationNonlinearShip2005} used the backstepping procedure and the tuning design method and \textcite{millerShipModelIdentification2021} used a genetic algorithm to identify parameters.

\subsection{Multicollinearity}
Multicollinearity refers to a situation in statistical modelling where two or more predictor variables are highly correlated, making it difficult to isolate the individual effects of each predictor on the dependent variable. This issue is particularly relevant in the field of ship manoeuvring modelling, where numerous hydrodynamic coefficients and parameters are involved.
The higher correlation of parameter is, or the stronger multicollinearity exists, the more difficult it is to identify regression coefficients separately \cite{yoonIdentificationHydrodynamicCoefficients2003}.
\textcite{wangQuantifyingMulticollinearityShip2018} discuss how multicollinearity can lead to parameter drift in system identification. When predictor variables are highly correlated, the estimates of the model parameters can become unstable and sensitive to small changes in the data. They use the variance inflation factor (VIF) to quantify the severity of multicollinearity in ship manoeuvring models. VIF measures how much the variance of a regression coefficient is inflated due to multicollinearity. A high VIF indicates a high level of multicollinearity.
The multicollinearity can to some extent be handled by pre-processing the data.
\textcite{luoParameterIdentificationShip2016} addresses the issue of parameter identifiability in ship manoeuvring modelling. By employing methods such as difference method and additional signal method, the study aims to reconstruct samples and reduce multicollinearity, thereby improving the feasibility of system identification.
\textcite{xuUncertaintyAnalysisHydrodynamic2019} introduce methods like truncated singular value decomposition and Tikhonov regularization to handle the uncertainty caused by multicollinearity. These techniques help in stabilizing the parameter estimates and improving the robustness of the model.

Model structure selection is a more pragmatic way to handle the multicollinearity, by reducing the number of parameters in the model to be identifiable from the data at hand.  \textcite{luoParameterIdentificationShip2016} reduced the number of parameters based on physical considerations. \textcite{costaRobustParameterEstimation2021} used the truncated singular value decomposition, and \textcite{liuPhysicsinformedIdentificationMarine2024} used sparse identification of nonlinear dynamics (SINDy) \cite{bruntonDiscoveringGoverningEquations2016}, to reduce the number of model parameters. 
\textcite{abkowitzMEASUREMENTHYDRODYNAMICCHARACTERISTICS1980} was only able to fight the multicollinearity through elimination of “inconvenient” terms which, however, could lead to models with limited applicability as certain regression terms may only become significant in special conditions like sailing in wind. This is perhaps the biggest drawback with the model structure selection – that the generalization of a model may suffer when parameters are excluded.

The best option to mitigate multicollinearity is to get more informative data with persistence of excitation including conditions where the input signals used in system identification are sufficiently rich in frequency content to excite all the modes of the system. This ensures that the system's response contains enough information to uniquely identify the system parameters. Without persistence of excitation, the identified model may not accurately represent the ship's behaviour in all scenarios.

\textcite{yoonIdentificationHydrodynamicCoefficients2003} discuss the importance of designing experiments that ensure persistence of excitation. They suggest using specific input scenarios that maximize the information content of the data, such as D-optimal designs. An optimal experimental design is easier to obtain for captive tests, where the state of the ship can be varied freely. However, \textcite{wangOptimalDesignExcitation2020} and \textcite{millerShipModelIdentification2021} suggest that a pseudo-random sequence (PRS) can be used for free running tests.  However, data from these kind of tests are very rare. A model basin is too small, and full scale tests of this kind are also very hard to find. 
Data for the mandatory zigzag and turning circle standard manoeuvres \cite{imoStandardsShipManoeuvrability2002} are much easier to find, which is why almost all papers use these manoeuvres for the system identification. However, these manoeuvres are not rich enough to guarantee reliable estimation of all regression coefficients \cite{sutuloAlgorithmOfflineIdentification2014}, which poses a great challenge for the system identification.

%\subsection{System identification with non-parametric models}
%The multicollinearity is not a problem for the non-parametric models. However, informative data with persistence of excitation is till a requirement since even the most clever model, cannot see behind corners. 