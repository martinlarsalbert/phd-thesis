\section{Propeller model} \label{sec:propeller}
The surge forces from the propeller are taken as the propeller thrust multiplied by a thrust deduction factor $t_{df}$ (\autoref{eq:X_P}).
\begin{equation}
    \label{eq:X_P}
    X_P = (1-t_{df})T
\end{equation}
The propeller thrust $T$ is taken as the measured thrust from VCT or FRMTs in this thesis, to reduce the uncertainty of the complex interaction between the propeller, rudder, and hull. The propeller also generates side forces, especially for yaw rates, which have a small stabilizing effect on the ship. The stabilizing propeller moment can be around 5\% of the rudder yawing moment, as shown in \autoref{fig:propeller_size_force}.
\begin{figure}[h]
    \centering
    \includesvg{figures/model_propeller_side_force.propeller_size_force.svg}
    %\caption{Typical yawing moments from rudder and propeller for various yaw rates.}

    \begin{minipage}[t]{5.25in}
    \caption{Typical yawing moments from rudder and propeller for various yaw rates.}
	\label{fig:propeller_size_force}
    \end{minipage}
        
\end{figure}
This effect is not explicitly modeled in this thesis, so that $Y_P=0$,$N_P=0$. The propeller side force is however included in the VCT data, so that propeller side force will be included in the hull coefficients instead.
