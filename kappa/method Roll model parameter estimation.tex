\section{Roll model parameter estimation} \label{sec:_roll}
\noindent Damping parameters can be estimated directly from the VCT forces, as demonstrated in \autoref{sec:VCT}. However, this approach is not feasible when the ship is free to move, necessitating the use of FT time series data to estimate the parameters. In such cases, the entire dynamics must be considered. The damping parameters ($B_1$, $B_2$, $B_3$) and stiffness parameters ($C_1$, $C_3$, $C_5$) can be identified from the parametric linear, quadratic, and cubic roll motion model structures presented in the previous chapter (\autoref{eq:roll_decay_equation_himeno_linear}, \autoref{eq:roll_decay_equation_himeno_quadratic_b}, and \autoref{eq:roll_decay_equation_cubic}).
These equations do not have unique solutions because each equation can be multiplied by an arbitrary factor to yield a new valid solution. To obtain unique solutions, inertia is excluded by normalizing the equations with the total roll inertia $A_{44}$, as shown in \autoref{eq:roll_decay_nonedim_a44} for the linear model.

\input{equations/roll_decay_nonedim_A44}

\noindent The identified normalized damping and stiffness parameters $B_{1A}$ and $C_{1A}$ can be expressed in dimensional units by multiplication with the normalization factor $A_{44}$. If $A_{44}$ is unknown before hand, it can be calculated using \autoref{eq:A_44_eq} \cite{piehlShipRollDamping2016}, assuming that the meta center height $GM$ is known.
\input{equations/A_44_eq}
The frequency $\omega_0$ can be obtained with Fast Fourier transform (FFT) of the roll signal. 
Two methods for parameter estimation have been investigated: the \say{derivation approach}, referred to in \textcite{imo1200InterimGuidelines2006}, and the \say{integration approach} used in \textcite{soderAssessmentShipRoll2019} which are both described in the next subsections. 

%\subsection{Inverse dynamics regression}\label{sec:derivation_approach}
In inverse dynamics regression (called derivation approach in Paper \ref{pap:rolldamping}), \autoref{eq:roll_decay_nonedim_a44} is treated as a linear regression problem, where the states ($\phi$, $\dot{\phi}$, $\ddot{\phi}$) are known and the parameters $B_1$ and $C_1$ must be regressed. Only roll angle $\phi$ is known from the experimental data, which means that the velocity and acceleration $\dot{\phi}$, $\ddot{\phi}$ also must be approximated (note that this is done with numerical differentiation in Paper \ref{pap:rolldamping} and with the extended Kalman filter (EKF) in Paper \ref{pap:pit}).
A least squares fit is applied to the roll motion equation to identify the damping and stiffness parameters.

%\subsection{Integration approach}\label{sec:integration_approach}
In the integration approach, \autoref{eq:roll_decay_nonedim_a44} is solved as an ordinary differential equation (ODE) for many estimated sets of parameters until the solution converges. This method is time-consuming, and convergence is not guaranteed. However, the advantage is that only roll angle $\phi$ is needed.