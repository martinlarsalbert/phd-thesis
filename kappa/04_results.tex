%%%%%%%%%%%%%%%%%%%%%%%%%%%%%%%
%%%%%%%%%%%%%%%%%%%%%%%%%%%%%%%
\chapter{Results\label{ch:results}}
%%%%%%%%%%%%%%%%%%%%%%%%%%%%%%%
This chapter presents a summary of the appended papers, including research activities
and a selection of the important results, and highlights the main achievements. A model for the roll motion is developed in Paper \ref{pap:rolldamping}. A manoeuvring model is developed in Paper \ref{pap:daiyong} and Paper \ref{pap:pit} and the model generalization is addressed in Paper \ref{pap:pit}.

\section{Summary of Paper \ref{pap:rolldamping}}
\subsection*{"\nameref{pap:rolldamping}"}
System identification of ship roll motion, including roll damping and stiffness, is developed in in Paper \ref{pap:rolldamping}. In the second generation of intact stability criteria, the IMO addressed the importance of ships having sufficient roll damping to avoid large roll motions, parametric rolling, and excessive acceleration \parencite{imo_finalization_2016}. These phenomena have been well known for a very long time. Parametric roll was observed already by \parencite{froude_rolling_1861} and has been on the agenda of the marine research community since the early 1950s \parencite{galeazzi_early_2013}; it has received much more attention since \parencite{france_investigation_2001} showed that the APL China casualty in 1998, where a post-Panamax C11 class container ship lost almost a third of its containers, was most likely caused by head sea parametric rolling. The damping of roll motion plays an important part during the above-mentioned phenomena. It has been shown that the relatively small difference in the roll damping prediction they obtained with small method variation, could mean the difference between severe roll angles and hardly noticeable motions \parencite{soder_ikeda_2019}.

The objective in Paper \ref{pap:rolldamping} was therefore to improve the roll damping predictions for modern ships. The roll damping was studied using time series data from 250 (see Fig. \ref{fig:ship_types}) roll decay tests (see \autoref{sec:roll}) assembled from the Maritime Dynamics Laboratory at SSPA Sweden AB (\href{www.sspa.se}{www.sspa.se}).

\begin{figure}[H]
    \centering
    \includegraphics[width=0.5\columnwidth]{kappa/images/ship_types.eps}
    \caption{Number of tests per ship type}
    \label{fig:ship_types}
\end{figure}

\noindent The work was broken down to the following subtasks: 
\begin{itemize}
    \item Find the mathematical model that best describes the roll motion
    \item Identify the parameters in this model for all the tests
    \item Compare the identified parameters with state of art prediction
    \item Develop a generic roll damping model for all ships, using the identified parameters
    \begin{itemize}
        \item Grey-box model
        \item Black-box model
    \end{itemize}
\end{itemize}

\noindent The work is also summarized in \autoref{fig:paper1_overview}. System identification on the time series from the roll decay database was performed with the linear (\autoref{eq:roll_decay_equation_himeno_linear}), quadratic (\autoref{eq:roll_decay_equation_himeno_quadratic_b}) and cubic model (\autoref{eq:roll_decay_equation_cubic}). Roll damping parameters identified from the model that was found to be the best was used to build a roll damping database. The identified roll dampings could then be compared with corresponding predictions with the Simplified-Ikedas method \cite{kawahara_simple_2011}, being the state of art prediction for ship roll damping.
The generic roll damping model was then developed as a grey-box model, including the Simplified-Ikedas method as the white part, and a pure black-box model.
\begin{figure}[H]
    \centering
    \includegraphics[width=\linewidth]{kappa/images/workflow.pdf}
    \caption{Overview of the work conducted for Paper \ref{pap:rolldamping}}
    \label{fig:paper1_overview}
\end{figure}

\subsection{Best mathematical model for the roll motion}
System identification on the linear, quadratic and cubic model was conducted using both the ''integration approach'' (described in \autoref{sec:integration_approach}) and the ''derivation approach'' (described in \autoref{sec:derivation_approach}).
Results from simulations with the identified models is shown for one of the roll-decay tests in \autoref{fig:roll_decay_compare}. It can be seen that the cubic and quadratic model reproduce the model test well and that the the linear model is too simple to have a good representation for both smaller and larger roll angles.

\begin{figure}[H]
    \centering
    \includegraphics[width=\linewidth]{kappa/images/roll_decay_model_compare.pdf}
    \caption{Roll decay estimation with identified cubic, quadratic and linear model.}
    \label{fig:roll_decay_compare}
\end{figure}

\begin{figure}[H]
    \begin{subfigure}[b]{0.45\textwidth}
        \centering
        \includegraphics[]{kappa/images/roll_decay_amplitude.eps}
        \caption{Amplitude decrements}
        \label{fig:roll_decay_amplitude}
    \end{subfigure}
        ~ %add desired spacing between images, e. g. ~, \quad, \qquad, \hfill etc. 
      %(or a blank line to force the subfigure onto a new line)
    \begin{subfigure}[b]{0.45\textwidth}
        \centering
        \includegraphics[]{kappa/images//roll_decay_damping.eps}
        \caption{Dampings}
        \label{fig:roll_decay_damping}
    \end{subfigure}
    \caption{Roll decay model test, linear-, quadratic- and cubic-model}
    \label{fig:roll_decay}
\end{figure}

The best parameter estimations were obtained using the ''integration approach''.

\subsection{Generic roll damping model}
\label{sec:genericrolldampingmodel}
A serial grey-box model for ship roll damping (see Fig.\ref{fig:greyrolldamping}) is also developed in Paper \ref{pap:rolldamping}. 
This is expanding the system identification, not only focusing on one ship, but rather all modern ships, by a prediction model of the damping coefficients from the  applied on a whole database of roll decay tests. 
Simplified Ikeda's method \cite{kawahara_simple_2011} is used as the white box model, which is combined with a following black-box correction model.

\begin{figure}[H]
    
    \centering
    \begin{tikzpicture}[node distance=2cm]
    \node (white-box) [white-box] {Simplified Ikeda};
    \node (B_BK) [io, right of=white-box, xshift=0.90cm, yshift=1.5cm] {$\hat{B_{BK}}$};
    \node (B_E) [io, right of=white-box, xshift=0.75cm, yshift=0.75cm] {$\hat{B_{E}}$};
    \node (B_F) [io, right of=white-box, xshift=0.75cm, yshift=0cm] {$\hat{B_{F}}$};
    \node (B_L) [io, right of=white-box, xshift=0.75cm, yshift=-0.75cm] {$\hat{B_{L}}$};
    \node (B_W) [io, right of=white-box, xshift=0.75cm, yshift=-1.5cm] {$\hat{B_{W}}$};
    
    
    \node (black-box) [black-box, right of=B_F, xshift=0.75cm] {Black-box};
    \draw [arrow] (white-box) -- (B_BK);
    \draw [arrow] (white-box) -- (B_E);
    \draw [arrow] (white-box) -- (B_F);
    \draw [arrow] (white-box) -- (B_L);
    \draw [arrow] (white-box) -- (B_W);
    
    \draw [arrow] (B_BK) -- (black-box);
    \draw [arrow] (B_E)  -- (black-box);
    \draw [arrow] (B_F)  -- (black-box);
    \draw [arrow] (B_L)  -- (black-box);
    \draw [arrow] (B_W)  -- (black-box);
    
    
    \node (B) [io, right of=black-box, xshift=0.75cm, yshift=0cm] {$B$};
    \draw [arrow] (black-box)  -- (B);
    
    \end{tikzpicture}
    \caption{Grey-box model to predict roll damping}
    \label{fig:greyrolldamping}
\end{figure}

\noindent The roll damping data set, obtained from the roll motion investigation, is used to train the black-box part of the grey-box model. The black-box correction model of the output components from the Simplified Ikeda's method are shown in (Eq.\ref{eq:polynom_correction}),
\begin{equation} \label{eq:polynom_correction}
\hat{B_{e}} = 1.106 \hat{B_{BK}} - 0.9124 \hat{B_{E}} + 4.282 \hat{B_{F}} + 0.7457 \hat{B_{L}} + 0.1844 \hat{B_{W}} + 0.004999 \phi_{a} - 0.0005097
\end{equation}


\noindent Large corrections of the skin friction damping $\hat{B_F}$ and wave damping $\hat{B_W}$ are suggested by this expression. This is because the Simplified Ikeda's method is not very accurate for this dataset, where most of the ships in the dataset exceed the limits of the method. A pure black-box model is also devloped in Paper \ref{pap:rolldamping} (see Eq.\ref{eq:polynom_complex}),
\begin{equation} \label{eq:polynom_complex}
\begin{aligned} 
 \hat{B_{e}} = - 0.02578 A_{0} V - 0.02705 BK_{B} V + \\ 
 0.008993 BK_{L} V - 0.03191 C_{b} V - 0.2028 OG V + \\ 
 0.003472 V^{2} + \\ 
 0.004234 V \hat{\omega_{0}} - 0.002591 V \phi_{a} - 0.008384 V beam + \\ 
 0.05048 V + \\ 
 0.007814 \hat{\omega_{0}}^{2} + \\ 
 0.03882 \hat{\omega_{0}} \phi_{a} - 0.001069 \\ 
 \end{aligned}
\end{equation}


\noindent The grey-box model and the black-box model above, have about the same accuracy when performing cross-validation on the roll damping dataset. The linearized equivalent damping $B_e$ is calculated with \autoref{eq:B_e_equation} \cite{himeno_prediction_1981}. The damping and frequency is nondimensionalized with \autoref{eq:be_eqvalent} and \autoref{eq:omega0_hat_equation} \cite{himeno_prediction_1981}.

\begin{equation} \label{eq:B_e_equation}
B_{e} = B_{1} + \frac{8 B_{2} \omega_{0} \phi_{a}}{3 \pi}
\end{equation}


\begin{equation} \label{eq:be_eqvalent}
    \hat{B_e} = \frac{B_e}{\rho \bigtriangledown Beam^2} \sqrt{\frac{Beam}{2g}},
\end{equation}

\begin{equation} \label{eq:omega0_hat_equation}
\omega_{hat} = \frac{\sqrt{2} \omega_{0} \sqrt{\frac{beam}{g}}}{2}
\end{equation}


\section{Summary of Paper \ref{pap:daiyong}}
\subsection*{"\nameref{pap:daiyong}"}
Least Square Support Vector Regression (LS-SVR) \cite{brereton_support_2010} is used in Paper \ref{pap:daiyong} to identify the parameters in the AVMM.  
The data is taken from experimental tests on a lake using a ship model with a scale of 50:1. The configuration of sensors and equipment for the experiment is shown in Fig.\ref{fig:cthmodel}.  
\begin{figure}[H]
    \centering
    \includegraphics[width=\textwidth]{kappa/images/cth_model.png}
    \caption{Configuration of sensors and equipment for the experimental tests.}
    \label{fig:cthmodel}
\end{figure}
\noindent The hydrodynamic derivatives of the AVMM are identified almost perfectly when applied on data from simulations with MSS toolbox Mariner \cite{tristan_matlab_2009}. The  does however not work at all when applied on the data obtained from the lake experiments as seen in Fig.\ref{fig:daiyong_extrapolation}. 

\begin{figure}[H]
    \centering
    \includegraphics[width=\linewidth]{kappa/images/daiyong_extrapolation.jpeg}
    \caption{Prediction with AVMM of zigzag lake experiments.}
    \label{fig:daiyong_extrapolation}
\end{figure}

\noindent The  is very sensitive to noise due to the differentiation that needs to be conducted to calculate velocities and yaw rate from the measured position and heading. The  works better if the data is first cleaned using a proposed preprocessing algorithm together with a Kalman Filter (KF). The simulations with the identified model and the experiments were however still not in very good agreement.     

\section{Summary of Paper \ref{pap:pit}}
\subsection*{"\nameref{pap:pit}"}
A new method for System Identification of ship manoeuvring dynamics is developed in Paper \ref{pap:pit}. The system model for the ship manoeuvring is assumed to be represented by a manoeuvring model (see Section \ref{sec:manoeuvring model}). The appropriate manoeuvring model for the specific ship and data is selected from a set of candidate VMMs, with varying complexity as a function of its number of hydrodynamic derivatives. The appropriate manoeuvring model is selected to give a robust model that can make predictions outside the domain covered by the available training data. A cross validation scheme is proposed to be used in the selection. In this scheme, the validation set has larger drift angles, yaw rates and rudder angles compared to the training set as seen in the example in Fig.\ref{fig:cross_validation}.
\begin{figure}[H]
    \centering
    \includegraphics[width=\linewidth]{kappa/images/3.pdf}
    \caption{wPCC training, validation and testing datasets.}
    \label{fig:cross_validation}
\end{figure}
\noindent A new  for VMMs (see Section \ref{sec:_VMM}), with large focus on the data cleaning, is also proposed in Paper \ref{pap:pit}. This  is investigated together with the method to select a appropriate and robust manoeuvring model for two very different cases. The wPCC test case which is a twin screw Pure Car Truck Carrier and the KVLCC2 being a single screw Very Large Crude Carrier.    
\begin{itemize}
    
    \item It is shown that the hydrodynamic derivatives within a manoeuvring model can be identified exactly at ideal conditions with no measurement noise and a perfect estimator.
    
    \item It is shown that the proposed prepossessing of measurement data with EKF + RTS run in iteration with initial guess from semi-empirical formulas, is better than using low-pass filters for cleaning. \autoref{fig:lowpass-accuracy} shows the average simulation error with the  using various low-pass filters or the EKF + RTS smoother.
    \begin{figure}[h]
        \centering
        \includegraphics{kappa/images/6.pdf}
        \caption{Average simulation error with MAVMM fitted on wPCC model test data using low-pass filters with various cutt off frequency or EKF.}
        \label{fig:lowpass-accuracy}
    \end{figure}
    
    \item The  has large problems with multicollinearity for the AVMM. The absolute correction between the features is shown in \autoref{fig:ncorr}. This was less of a problem for the less complex model MAVMM.
    \begin{figure}[h]
        \centering
        \includegraphics{kappa/images/9.pdf}
        \caption{Absolute correlation between the features in the wPCC yaw moment regression of AVMM}
        \label{fig:ncorr}
    \end{figure}
    
    \item The new method can predict turning circle manoeuvres with less than 5 \% error in advance and tactical diameter for the wPCC and KVLCC2 test cases. Track plot of the wPCC result is shown in Fig.\ref{fig:turning_circle_wpcc}.
    \begin{figure}[h]
    \centering
    \includegraphics{kappa/images/10.pdf}
    \caption{Track plots of the turning circle test case for wPCC from Model test and simulation (Test). Also simulations with alternative realizations of the regression (Monte Carlo) are shown in this figure.}
    \label{fig:turning_circle_wpcc}
    \end{figure}
    
\end{itemize}






