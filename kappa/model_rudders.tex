\section{Rudder models} \label{sec:rudders}
It has become evident during this investigation that an accurate rudder model is central to achieving high accuracy in the overall manoeuvring model. Polynomial rudder models were used in Paper \ref{pap:pit}, which well describe the rudder forces if a sufficiently high polynomial degree is used. However, these models introduce multicollinearity into the model, which poses a significant challenge in system identification, particularly when estimating parameters using inverse dynamics  (see \autoref{sec:IDR}). Since only the total force acting on the ship can be observed, it becomes difficult to separate hull-generated forces from rudder-induced forces as shown in \autoref{fig:hull_rudder_forces}. Instead of using a data-driven rudder model, a semi-empirical deterministic rudder model (see \autoref{sec:semi-empirical}) was therefore introduced in Paper \ref{pap:physics}. The rudder forces were calculated on the basis of the rudder's characteristics and established coefficients from the literature.
\begin{figure}[h]
    \centering
    \includesvg{figures/multicollinearity.svg}
    \caption{Multicollinearity between hull and rudder forces.}
    \label{fig:hull_rudder_forces}
\end{figure}
A third rudder model was introduced in Paper \ref{pap:vct} as a modified version of the MMG rudder model \cite{yasukawaIntroductionMMGStandard2015}. This model was found to be easier to adopt when very rich information about the rudder forces was available from the VCT data.
\subsection{Semi-empirical rudder model} \label{sec:semi-empirical}
The semi-empirical rudder model was proposed in Paper \ref{pap:physics}  as is a lifting line model similar to \textcite{kjellberg_sailing_2023}, \textcite{matusiak_dynamics_2021}, and \textcite{hughes_tempest_2011} that is primarily based on the rudder wind tunnel tests conducted by \textcite{whicker_free-stream_1958}. The surge and sway forces are expressed as rudder lift $L_R$ and rudder drag $D_R$, which are projected on the ship through the rudder inflow angle $\alpha_f$ (see \autoref{eq:X_R_semiempirical}, \autoref{eq:Y_R_semiempirical}, and \autoref{fig:inflow}).
This angle is the sum of the initial inflow to the rudder at a straight course $\gamma_0$ and the inflow to the rudder $\gamma$ due to propeller-induced speed, drift angle, and yaw rate of the ship, as shown in \autoref{eq:gamma_semiempirical}.
%
\begin{figure}[h]
    \centering
    \includesvg[pretex=\centering\fontsize{10}{11}]{figures/rudder_flow.svg}
    \caption{Inflow to the rudder.}
    \label{fig:inflow}
\end{figure}
%
\begin{equation}
    \label{eq:X_R_semiempirical}
    X_{R} = - F_{N} \left(t_{R} - 1\right) \sin{\left(\delta \right)}
\end{equation}
%
\begin{equation}
    \label{eq:Y_R_semiempirical}
    Y_{R } = D_{R } \sin{\left(\alpha_{f } \right)} + L_{R } \cos{\left(\alpha_{f } \right)}
\end{equation}
%
\begin{equation}
    \label{eq:alpha_f_semiempirical}
    \alpha_{f } = \gamma_{0 } + \gamma_{}
\end{equation}
%
\begin{equation}
    \label{eq:gamma_semiempirical}
    \gamma_{} = \operatorname{atan}{\left(\frac{V_{R y }}{V_{R x C }} \right)}
\end{equation}

The transverse velocity at the rudder $V_{Ry}$ is calculated by multiplying the ship's yaw rate $r$ and transverse velocity $v$ by their flow straightening values $\kappa_{rtot}$ and $\kappa_{vtot}$ (\autoref{eq:V_R_y_semiempirical}). The flow straightening values have linear and nonlinear dependencies of the geometric inflow angle $\gamma_g$ (\autoref{eq:gamma_g_semiempirical}), as calculated in \autoref{eq:kappa_r_tot_semiempirical} with $\kappa_r,\kappa_{r \gamma g}$ and \autoref{eq:kappa_v_tot_semiempirical} with $\kappa_v,\kappa_{v \gamma g}$, respectively, so that the flow straightening may vary for different inflow angles, which is an enhancement of the MMG model.
The axial velocity at the rudder $V_{RxC}$, including the velocity of the propeller race, is presented in \autoref{sec:velocity_in_the_propeller_slip_stream}.
\begin{equation}
    \label{eq:V_R_y_semiempirical}
    V_{R y } = - \kappa_{r tot } r x_{R} - \kappa_{v tot } v
\end{equation}
%
\begin{equation}
    \label{eq:kappa_r_tot_semiempirical}
    \kappa_{r tot } = \kappa_{r} + \kappa_{r \gamma g} \left|{\gamma_{g }}\right|
\end{equation}
%
\begin{equation}
    \label{eq:kappa_v_tot_semiempirical}
    \kappa_{v tot } = \kappa_{v} + \kappa_{v \gamma g} \left|{\gamma_{g }}\right|
\end{equation}
%
\begin{equation}
    \label{eq:gamma_g_semiempirical}
    \gamma_{g } = \operatorname{atan}{\left(\frac{- r x_{R} - v}{V_{R x C }} \right)}
\end{equation}
The yawing moment is modeled as the sway force multiplied by the lever arm $x_R$, as in \autoref{eq:N_R_semiempirical}.
\begin{equation}
    \label{eq:N_R_semiempirical}
    N_{R} = Y_{R} x_{r}
\end{equation}
%
%
\subsubsection{Rudder lift}
\label{sec:rudder lift}
With inspiration from the work of \textcite{villa_numerical_2020}, the total rudder lift is calculated as the sum of the lift at the rudder areas that are covered by the propeller $L_{RC}$ and that at the uncovered area $L_{RU}$, as shown in \autoref{eq:L_R_semiempirical} and \autoref{fig:rudder_coverage}.
\begin{equation}
    \label{eq:L_R_semiempirical}
    L_{R } = L_{R C } + L_{R U }
\end{equation}
%
\begin{figure}[h]
    \centering
    \includesvg{figures/rudder_coverage.svg}
    \caption{Rudder areas covered and uncovered by the propeller.}
    \label{fig:rudder_coverage}
\end{figure}
%
The lift forces are calculated (\autoref{eq:L_R_U_semiempirical} and \autoref{eq:L_R_C_semiempirical}) with the lift coefficient $C_L$. These equations are essentially the same except that the lift at the covered part $L_{RC}$ is diminished by the factor $\lambda_R$ (\autoref{eq:lambda_R_semiempirical}) because of the limited radius of the propeller slipstream in the lateral direction \cite{brix_manoeuvring_1993} (See \autoref{sec:rudder_models} for further details).
\begin{equation}
    \label{eq:L_R_U_semiempirical}
    L_{R U } = \frac{A_{R U} C_{L } V_{R U }^{2} \rho}{2}
\end{equation}
%
\begin{equation}
    \label{eq:L_R_C_semiempirical}
    L_{R C } = \frac{A_{R C} C_{L } V_{R C }^{2} \lambda_{R } \rho}{2}
\end{equation}
The velocities of the uncovered $V_{RU}$ and covered $V_{RC}$ parts of the rudder are calculated according to \ref{sec:velocity_outside_the_propeller_slip_stream} and \ref{sec:velocity_in_the_propeller_slip_stream}.
For a nonstalling rudder, the lift coefficient $C_L$ is calculated according to \textcite{whicker_free-stream_1958} with the additional parameter $K_{gap}$ as shown in \autoref{eq:C_L_semiempirical}.
\begin{equation}
    \label{eq:C_L_semiempirical}
    C_{L } = K_{gap } \left(\alpha_{} \frac{\partial C_L}{\partial \alpha} + \frac{C_{DC} \alpha_{} \left|{\alpha_{}}\right|}{AR_{e }}\right)
\end{equation}
%
\begin{equation}
    \label{eq:alpha_semiempirical}
    \alpha_{} = \delta + \gamma_{0 } + \gamma_{}
\end{equation}
The effective aspect ratio $AR_e$ accounts for the mirror image effect when the rudder is flush with the hull, and it is typically assumed to be twice the geometric aspect ratio $AR_g$ (\autoref{eq:AR_e_semiempirical} and \autoref{eq:AR_g_semiempirical}) \cite{hughes_tempest_2011}.
However, The wPCC rudder is not flush with the hull, so a gap is created between the rudder and rudder horn at larger rudder angles, reducing the pressure difference between the high- and low-pressure sides in the upper part of the rudder. \textcite{matusiak_dynamics_2021} proposed that the gap effect can be modeled as a reduced aspect ratio. Instead, this paper opts for a more straightforward approach based on experience. A factor $K_{gap}$ is introduced, calculated according to \autoref{eq:lambda_gap_semiempirical}. The gap effect is only activated above a threshold rudder angle $\delta_{lim}$, and the strength of the gap effect is modeled by a factor $s$, as in \autoref{fig:gap}.
%
\begin{equation}
    \label{eq:AR_g_semiempirical}
    AR_{g } = \frac{b_{R}^{2}}{A_{R}}
\end{equation}
%
\begin{equation}
    \label{eq:AR_e_semiempirical}
    AR_{e } = 2 AR_{g }
\end{equation}
%
\begin{equation}
    \label{eq:lambda_gap_semiempirical}
    K_{gap} = \begin{cases} 1 & \text{for}\: \delta_{lim} > \left|{\delta}\right| \\s \left(- \delta_{lim} + \left|{\delta}\right|\right)^{2} + 1 & \text{otherwise} \end{cases}
\end{equation}
\begin{figure}[h]
    \centering
    \includesvg[width=0.5\columnwidth]{figures/gap_effect.gap.svg}
    \caption{Rudder lift is reduced by the gap between the rudder and rudder horn for larger rudder angles.}
    \label{fig:gap}
\end{figure}

The lift slope of the rudder $\frac{\partial C_L}{\partial \alpha}$ is calculated using \autoref{eq:dC_L_dalpha_semiempirical}, where $a_0$ is the section lift curve slope (\autoref{eq:a_0_semiempirical}) and $\Omega$ is the sweep angle of the quarter chord line \cite{lewis_principles_1989}.
\begin{equation}
    \label{eq:dC_L_dalpha_semiempirical}
    \frac{\partial C_L}{\partial \alpha} = \frac{AR_{e } a_{0 }}{\sqrt{\frac{AR_{e }^{2}}{\cos^{4}{\left(\Omega \right)}} + 4} \cos{\left(\Omega \right)} + 1.8}
\end{equation}
%
%
\begin{equation}
    \label{eq:a_0_semiempirical}
    a_{0 } = 1.8 \pi
\end{equation}
%
Additionally, a small nonlinear part to $C_L$ is modeled by the cross-flow drag coefficient $C_{DC}$, which is calculated for a rudder with squared tip using \autoref{eq:C_D_crossflow_semiempirical}, where the taper ratio $\lambda$ is the ratio between the chords at the tip and the root of the rudder (\autoref{eq:lambda__semiempirical}) \cite{hughes_tempest_2011}. 
\begin{equation}
    \label{eq:C_D_crossflow_semiempirical}
    C_{DC} = 1.6 \lambda^{} + 0.1
\end{equation}
%
\begin{equation}
    \label{eq:lambda__semiempirical}
    \lambda^{} = \frac{c_{t}}{c_{r}}
\end{equation}
%
%
\subsubsection{Rudder drag}
\label{sec:rudder_drag}
The total rudder drag $D_R$ is calculated as a sum of the contributions from the parts covered and uncovered by the propeller, as in \autoref{eq:D_R_semiempirical}.
\begin{equation}
    \label{eq:D_R_semiempirical}
    D_{R } = 0.5 \rho \left(A_{R C} C_{D C } V_{R C }^{2} + A_{R U} C_{D U } V_{R U }^{2}\right)
\end{equation}
The drag coefficients $C_{DC}$ and $C_{DU}$ are calculated with semi-empirical formulas according to \ref{sec:CD}.


\subsection{Modified quadratic MMG rudder model}
A modified quadratic MMG rudder model is proposed in Paper \ref{pap:vct}, with two enhancements to the original MMG rudder model \cite{yasukawaIntroductionMMGStandard2015}. The first enhancement is to add the rudder initial inflow angle $\gamma_0$ to the calculation of the effective inflow angle to the rudder $\alpha_R$, by replacing equation 21 in \textcite{yasukawaIntroductionMMGStandard2015} with the modified equation (\autoref{eq:alpha_R2}). This allows the rudder model to produce a side force in the straight ahead condition, due to unsymmetrical flow from the propeller. 
\begin{equation}
    \label{eq:alpha_R2}
    %\alpha_{R} = \delta + \gamma_{0} + \operatorname{atan}{\left(\frac{v_{R}}{u_{R}} \right)}
    \alpha_{R} = \delta + \underbrace{\gamma_{0}}_{\text{ proposed}} + \operatorname{atan}{\left(\frac{v_{R}}{u_{R}} \right)}
\end{equation}
The rudder transverse velocity $v_R$ is calculated as:
\begin{equation}
    \label{eq:gamma_R2}
    v_{R} = V \beta_{R} \gamma_{R}
\end{equation}
where $\beta_R$ is the effective inflow angle to the rudder as function of both the drift angle $\beta$ and yaw rate $r$:
\begin{equation}
    \label{eq:beta_R}
    \beta_{R} = \beta - \frac{l_{R} r}{V}
\end{equation}
where $l_R$ is a lever arm to the rudder that is treated as an experimental constant.
The other enhancement is to allow for a quadratic relationship between the flow straightening coefficient $\gamma_R$ and the effective inflow angle $\beta_R$, by introducing two new coefficients $\gamma_{R2neg}$, and $\gamma_{R2pos}$ as shown in:  
\begin{equation}
    \label{eq:gamma_R2}
    \gamma_{R} = \begin{cases} \overbrace{\gamma_{R2 neg} \left|{\beta_{R}}\right|}^{proposed} +   
    \gamma_{R neg} & \text{for}\: \beta_{R} \leq 0 \\\gamma_{R2 pos} \left| 
    {\beta_{R}}\right| + \gamma_{R pos} & \text{otherwise} \end{cases}
\end{equation}
