\section{Prime system with perturbed surge velocity} \label{sec:prime_system}
Some variables in the equations in this thesis are expressed using non-dimensional units with the prime system, denoted by the prime symbol ($'$). Variables are converted from SI units to the prime system using the denominators in \autoref{tab:prime-system-denominators} for the corresponding physical quantity, where $U$ and $L$ are the velocity and length between the perpendiculars of the ship, respectively, and $\rho$ is the water density.
For the calculation of surge velocity $u'$, the perturbed velocity $(u-U_0)$ about a nominal speed $U_0$ is used, as in \autoref{eq:u_prime}, to avoid a $u'$ of 1 for all speeds when the ship is on a straight course (where $u=U$), as in a resistance or self-propulsion test. The usage of the perturbed velocity, therefore, allows for higher order resistance terms in the model, such as $X_{u}$, which are otherwise not possible. 
\begin{equation}
    \label{eq:u_prime}
    u' = \frac{u-U_0}{U}
\end{equation}
For a nondimensional model, $U_0$ is instead expressed as a Froude number within the model (\autoref{eq:Fn0}), and this paper uses $F_{n0}=0.02$.
\begin{equation}
    \label{eq:Fn0}
    F_{n0} = \frac{U_0}{\sqrt{g \cdot L}}
\end{equation}
\begin{table}[h]
    \centering
    \caption{Scalings with prime system.}
    \label{tab:prime-system-denominators}
    \pgfplotstabletypeset[col sep=comma,
        columns={Physical quantity,SI unit,Denominator},
        columns/SI unit/.style={string type},
        columns/Physical quantity/.style={string type},
        columns/Denominator/.style={string type},
        column type=l,	% specify the align method
        every head row/.style={before row=\hline,after row=\hline},	% style the first row
        every last row/.style={after row=\hline},	% style the last row
    ]{tables/prime_system.prime_system.csv"}
\end{table}