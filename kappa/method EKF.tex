\section{Data cleaning}
\label{sec:datacleaning}
It is possible to perform an exact parameter estimation on flawless (simulated) data with no noise (see Paper \ref{pap:pit}). However, such ideal data from physical experiments do not exist in reality. Measured data will always contain both process noise and measurement noise. Even very moderate measurement noise can create significant problems for inverse dynamics in cases where noise is amplified when velocities and accelerations are differentiated from measured positions.
Mitigation of the effects of noise can be achieved if the data are pre-processed using the extended Kalman filter (EKF) \cite{brownIntroductionRandomSignals1997} and the Rauch-Tung-Striebel (RTS) smoother \cite{rauchMaximumLikelihoodEstimates1965}, both of which are presented below.

EKF is an extension of the Kalman filter (KF) that is used for nonlinear systems, such as manoeuvring models. The premise is that noise can be neglected in the absence of a plausible physical explanation. For instance, if noisy measurement data were entirely correct,it would imply that large ship vibrations resulted from large high-frequency forces (given the size of the ship). A prior understanding of the system dynamics suggests that these forces are not present. Therefore, the noise should be considered measurement noise and should be removed. Low-pass filtering is commonly used to remove noise, where motions above a cutoff frequency are considered unphysical measurement noise. However, choosing this cutoff frequency is difficult. It is often  either too low (removing some of the signal) or too high (retaining some unfiltered measurement noise in the data). The Kalman filter has a predictor model, a manoeuvring model in this case, that continuously estimates the system’s state that runs parallel with the measurement data. The filter estimates the current state as a combination of the measurement data and the predictor model estimate based on the possible validity of the data and the model. If the data has low noise, the estimate is closer to that data. Conversely, if the model provides very accurate predictions, then that estimate is closer to the model.
The system’s inverse dynamics require everything about the state (positions, velocities, and accelerations) to be known. Only positions are known from the measurements, so the velocities and accelerations are instead estimated by the EKF.

The EKF is recursive and updates estimates in real-time as new measurements become available. It uses past measurements to predict states in the near future, making it useful for online applications, such as autopilots or autonomous ships. However, this real-time constraint is unnecessary for the estimation of pre-existing data, where an entire time series of existing measurements is available. In such cases, knowledge of both past and future data can be used to improve the filter. Future time steps can be included by applying the RTS smoother after the filter. The RTS smoother algorithm runs the EKF in reverse to account for future time steps. The EKF recursive algorithm used is summarized in the pseudo-code in \autoref{fig:ek-algorithm} \cite{brownIntroductionRandomSignals1997}.
\clearpage
\label{04.01_EK:ek-algorithm}
\begin{sphinxadmonition}{note}{Algorithm 3.1 (Discrete\sphinxhyphen{}time extended Kalman filter)}



\sphinxAtStartPar
\sphinxstylestrong{Inputs} Initial values: \(x_0\), \(P_0\), \(C_d\), \(R_d\), \(Q_d\), \(E_d\)

\sphinxAtStartPar
\sphinxstylestrong{Output} Estimated states: \(\hat{x}\), estimated state covariances \(\hat{P}\)
\begin{enumerate}
\sphinxsetlistlabels{\arabic}{enumi}{enumii}{}{.}%
\item {} 
\sphinxAtStartPar
Initial values:
\begin{enumerate}
\sphinxsetlistlabels{\arabic}{enumii}{enumiii}{}{.}%
\item {} 
\sphinxAtStartPar
\(\hat{x}[0] = x_0\)

\item {} 
\sphinxAtStartPar
\(\hat{P}[0] = P_0\)

\end{enumerate}

\item {} 
\sphinxAtStartPar
For \(k\) in \(n\) measurements (time steps)
\begin{enumerate}
\sphinxsetlistlabels{\arabic}{enumii}{enumiii}{}{.}%
\item {} 
\sphinxAtStartPar
KF gain
\begin{enumerate}
\sphinxsetlistlabels{\arabic}{enumiii}{enumiv}{}{.}%
\item {} 
\sphinxAtStartPar
\(K[k]=\hat{P}[k] C_d^T \left(C_d \hat{P}[k] C_d^T + R_d\right)^{-1}\)

\item {} 
\sphinxAtStartPar
\(I_{KC} = I_n - K[k] C_d\)

\end{enumerate}

\item {} 
\sphinxAtStartPar
Update
\begin{enumerate}
\sphinxsetlistlabels{\arabic}{enumiii}{enumiv}{}{.}%
\item {} 
\sphinxAtStartPar
State corrector
\(\hat{x}[k] = \hat{x}[k] + K[k] (y - C_d \hat{x}[k]) \)

\item {} 
\sphinxAtStartPar
Covariance corrector
\(\hat{P}[k] = I_{KC} \cdot \hat{P}[k] I_{KC}^T + K[k] R_d K^T \)

\end{enumerate}

\item {} 
\sphinxAtStartPar
Predict
\begin{enumerate}
\sphinxsetlistlabels{\arabic}{enumiii}{enumiv}{}{.}%
\item {} 
\sphinxAtStartPar
State predictor
\(\hat{x}[k+1] = \hat{x}[k] + h \cdot \hat{f}(\hat{x}[k], c[k])\)

\item {} 
\sphinxAtStartPar
Covariance predictor
\(\hat{P}[k+1] = A_d[k]  \hat{P}[k] A_d[k]^T + E_d Q_d E_d^T \)

\end{enumerate}

\end{enumerate}

\end{enumerate}
\end{sphinxadmonition}
Here, \(n\) is the number of states (6 in this case), and \(\mathbf{I_n}\) is an $n$ by $n$ identity matrix.
The transition matrix is calculated for each iteration using the Jacobian of the transition model:
\begin{equation}\label{equation:04.01_EK:eqjacobi}
\begin{split}\mathbf{A_d}[k] = \mathbf{I_n} + h \left. \frac{\partial f \left(\mathbf{x}[k],\mathbf{c}[k] \right)}{\partial \mathbf{x}[k]} \right|_{\mathbf{x}[k]=\mathbf{\hat{x}}[k]}\end{split}
\end{equation}
This formulation and the fact that the nonlinear transition model is used directly as the predictor are the key differences between the EKF and the linear KF. Please note the linear approximation in \autoref{equation:04.01_EK:eqjacobi} around the current state. This approximation can cause instability if the real system and the linearized system deviate significantly when large time steps are used on a very nonlinear system. The unscented Kalman filter, which was used in \textcite{revestidoherreroTwostepIdentificationNonlinear2012}, is an alternative that can be applied in such situations.

The output from the filter contains the estimated states: \(\mathbf{\hat{x}}\) and the estimated state covariance matrix \(\mathbf{\hat{P}}\). \(\mathbf{\hat{x}}\) represents the most likely estimates, but these estimates have uncertainty that is expressed in \(\mathbf{\hat{P}}\).
The state of the system is described by the ship's position, heading, velocities, and yaw velocity:
\begin{equation}\label{equation:04.01_EK:eqstates}
\begin{split}\mathbf{x} = [x_0,y_0,\psi,u,v,r]^T\end{split}
\end{equation}

The initial state \(x_0\) is taken as the mean value of the first five measurements, where the velocities are estimated using numeric differentiation.
\(\mathbf{C_d}\) selects the measured states (\(x_0\), \(y_0\), \(\psi\)):
\begin{equation}\label{equation:04.01_EK:eqcd}
\begin{split}\displaystyle \mathbf{C_{d}} = h \left[\begin{matrix}1 & 0 & 0 & 0 & 0 & 0\\0 & 1 & 0 & 0 & 0 & 0\\0 & 0 & 1 & 0 & 0 & 0\end{matrix}\right]\end{split}
\end{equation}
\(\mathbf{E_d}\) selects the hidden states (\(u\), \(v\), \(r\)):
\begin{equation}\label{equation:04.01_EK:eqed}
\begin{split}\displaystyle \mathbf{E_{d}} = h \left[\begin{matrix}0 & 0 & 0\\0 & 0 & 0\\0 & 0 & 0\\1 & 0 & 0\\0 & 1 & 0\\0 & 0 & 1\end{matrix}\right]\end{split}
\end{equation}
where \(h\) is the discrete time step, \(\mathbf{R_d}\) describes the covariance matrix of the measurement, \(\mathbf{Q_d}\) is the covariance matrix of the process model, and \(\mathbf{P_0}\) is the initial state covariance.
Selecting appropriate values for these three matrices is the most challenging aspect of optimizing EKF performance. The matrix \(\mathbf{R_d}\) should reflect the expected measurement noise,  and  \(\mathbf{Q_d}\) should account for the uncertainty introduced by the process model (manoeuvring model) .