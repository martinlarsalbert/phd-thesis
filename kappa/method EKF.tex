\section{Data cleaning}
\label{sec:datacleaning}
It is possible to do an exact parameter estimation on flawless (simulated) data with no noise (see Paper \ref{pap:pit}). However, such data from physical experiments does not exist in reality. The measured data will always contain process noise and measurement noise. In order to mitigate the effects of noise, the data is pre-processed using the extended Kalman filter (EKF) \cite{brownIntroductionRandomSignals1997} and the Rauch Tung Striebel (RTS) smoother \cite{rauchMaximumLikelihoodEstimates1965}, which are both presented below.

EKF is an extension of the Kalman filter (KF) that is used to work on nonlinear systems such as the manoeuvring models. The premise is that noise can be neglected if it has no rational physical explanation. For instance, if noisy measurement data would be  completely correct, that would mean that large ship vibrations must have originated from large high frequency forces considering the large mass of the ship. A prior understanding of the dynamics suggests that these forces are not present. Therefore, the noise should be considered as measurement noise and should be removed. Low-pass filtering is commonly used to remove noise; motions above a cutoff frequency are considered unphysical measurement noise. The problem with low-pass filtering is that choosing the cutoff frequency is difficult. It is often  either too low (removing some of the signal) or too high (keeping some unfiltered measurement noise in the data). The Kalman filter has a predictor model, a manoeuvring model in this case, that continuously estimates the system’s state that runs parallel with the measurement data. The filter estimates the current state as a combination of the measurement data and the predictor model estimate based on the possible validity of the data and the model. If the data has low noise, the estimate is closer to that data. Conversely, if the model provides very accurate predictions, then that estimate is closer to the model.
The system’s inverse dynamics require everything about the state (positions, velocities, and accelerations) to be known. Only positions are known from the measurements, but the velocities and accelerations are instead estimated by the EKF.

The EKF is recursive and can be ran online; it continuously makes new estimates as new measurements arrive. The EKF uses passed measurements to estimate states in the near future. This property is commonly used for online applications such as autopilots or autonomous ships. This restriction is unnecessary for the estimation for already existing data, where an entire time series of existing measurements is available. The knowledge of both past and future data can be used to improve the filter. An EKF filter can include future time steps by adding the RTS smoother after the filter. The RTS smoother is an algorithm that runs the EKF backward to account for future time steps. The used EKF recursive algorithm used is summarized in the pseudo-code in \autoref{fig:ek-algorithm} \cite{brownIntroductionRandomSignals1997}.
\clearpage
\begin{figure}[h]
\noindent\begin{sphinxadmonition}{note}{Algorithm 3.1 (Discrete\sphinxhyphen{}time extended Kalman filter)} \\
\sphinxstylestrong{Inputs} \\
\hspace*{0.5cm} Initial values: \(\mathbf{x_0}\), \(\mathbf{P_0}\) \\
\hspace*{0.5cm} Filter parameters: \(\mathbf{C_d}\), \(\mathbf{R_d}\), \(\mathbf{Q_d}\), \(\mathbf{E_d}\) \\
\hspace*{0.5cm} Data: \(\mathbf{y}\), \(\mathbf{c}\) \\
\sphinxstylestrong{Output} \\
\hspace*{0.5cm} Estimated states: \(\mathbf{\hat{x}}\), estimated state covariances \(\mathbf{\hat{P}}\)
\vspace{0.2cm}
\begin{enumerate}
\sphinxsetlistlabels{\arabic}{enumi}{enumii}{}{.}%
\item {} 
Initial values:
\begin{enumerate}
\sphinxsetlistlabels{\arabic}{enumii}{enumiii}{}{.}%
\item {} 

\(\mathbf{\hat{x}}[0] = \mathbf{x_0}\)

\item {} 

\(\mathbf{\hat{P}}[0] = \mathbf{P_0}\)

\end{enumerate}

\item {} 

For \(k\) in \(n\) measurements (time steps)
\begin{enumerate}
\sphinxsetlistlabels{\arabic}{enumii}{enumiii}{}{.}%
\item {} 

KF gain
\begin{enumerate}
\sphinxsetlistlabels{\arabic}{enumiii}{enumiv}{}{.}%
\item {} 

\(\mathbf{K}[k]=\mathbf{\hat{P}}[k] \mathbf{C_d}^T \left(\mathbf{C_d} \mathbf{\hat{P}}[k] \mathbf{C_d}^T + \mathbf{R_d}\right)^{-1}\)

\item {} 

\(\mathbf{I_{KC}} = \mathbf{I_n} - \mathbf{K}[k] \mathbf{C_d}\)

\end{enumerate}

\item {} 

Update
\begin{enumerate}
\sphinxsetlistlabels{\arabic}{enumiii}{enumiv}{}{.}%
\item {} 

State corrector
\(\mathbf{\hat{x}}[k] = \mathbf{\hat{x}}[k] + \mathbf{K}[k] (\mathbf{y} - \mathbf{C_d} \mathbf{\hat{x}}[k]) \)

\item {} 

Covariance corrector
\(\mathbf{\hat{P}}[k] = \mathbf{I_{KC}} \cdot \mathbf{\hat{P}}[k] \mathbf{I_{KC}}^T + \mathbf{K}[k] \mathbf{R_d} \mathbf{K}^T \)

\end{enumerate}

\item {} 

Predict
\begin{enumerate}
\sphinxsetlistlabels{\arabic}{enumiii}{enumiv}{}{.}%
\item {} 

State predictor
\(\mathbf{\hat{x}}[k+1] = \mathbf{\hat{x}}[k] + h \cdot \mathbf{\hat{f}}(\mathbf{\hat{x}}[k], \mathbf{c}[k])\)

\item {} 

Covariance predictor
\(\mathbf{\hat{P}}[k+1] = \mathbf{A_d}[k]  \mathbf{\hat{P}}[k] \mathbf{A_d}[k]^T + \mathbf{E_d} \mathbf{Q_d} \mathbf{E_d}^T \)

\end{enumerate}

\end{enumerate}

\end{enumerate}
\end{sphinxadmonition}
\caption{Algorithm for a discrete time extended Kalman filter.}
\label{fig:ek-algorithm} 
\end{figure}
Here, \(n\) is number of states (6 in this case), \(\mathbf{I_n}\) is an $n$ by $n$ identity matrix.
The transition matrix is calculated for each iteration using a Jacobian of the transition model:
\begin{equation}\label{equation:04.01_EK:eqjacobi}
\begin{split}\mathbf{A_d}[k] = \mathbf{I_n} + h \left. \frac{\partial f \left(\mathbf{x}[k],\mathbf{c}[k] \right)}{\partial \mathbf{x}[k]} \right|_{\mathbf{x}[k]=\mathbf{\hat{x}}[k]}\end{split}
\end{equation}
This part and the fact that the nonlinear transition model is used directly as the predictor are the extension part of the EKF compared to the linear KF. Please note the linear approximation in \autoref{equation:04.01_EK:eqjacobi} around the current state. This approximation can cause stability problems if the real system and the linearized system deviates too much, when large time steps are used on a very nonlinear system. The unscented Kalman filter, which was used in \textcite{revestidoherreroTwostepIdentificationNonlinear2012}, is an alternative that can be used in these situations.

The output from the filter contains the estimated states: \(\mathbf{\hat{x}}\) and estimated state covariance matrix \(\mathbf{\hat{P}}\). \(\mathbf{\hat{x}}\) represent the most likely estimates, but the estimates have uncertainty that is expressed in \(\mathbf{\hat{P}}\).
The state of the system is described by the ships position, heading, velocities and yaw velocity:
\begin{equation}\label{equation:04.01_EK:eqstates}
\begin{split}\mathbf{x} = [x_0,y_0,\psi,u,v,r]^T\end{split}
\end{equation}

The initial state \(x_0\) is taken as the mean value of the first five measurements, where the velocities are estimated with numeric differentiation.
\(\mathbf{C_d}\) selects the measured states (\(x_0\), \(y_0\), \(\psi\)):
\begin{equation}\label{equation:04.01_EK:eqcd}
\begin{split}\displaystyle \mathbf{C_{d}} = h \left[\begin{matrix}1 & 0 & 0 & 0 & 0 & 0\\0 & 1 & 0 & 0 & 0 & 0\\0 & 0 & 1 & 0 & 0 & 0\end{matrix}\right]\end{split}
\end{equation}
\(\mathbf{E_d}\) selects the hidden states (\(u\), \(v\), \(r\)):
\begin{equation}\label{equation:04.01_EK:eqed}
\begin{split}\displaystyle \mathbf{E_{d}} = h \left[\begin{matrix}0 & 0 & 0\\0 & 0 & 0\\0 & 0 & 0\\1 & 0 & 0\\0 & 1 & 0\\0 & 0 & 1\end{matrix}\right]\end{split}
\end{equation}
where \(h\) is the discrete time step, \(\mathbf{R_d}\) describes the covariance matrix of the measurement, \(\mathbf{Q_d}\) is the covariance matrix of the process model, and \(\mathbf{P_0}\) is the initial state covariance.
Selecting good values for these three matrices is the most complicated part of getting the EKF to work well. The amount of expected measurement noise in the data should be inserted in \(\mathbf{R_d}\), and the amount of error generated by the process model (manoeuvring model) should be estimated in \(\mathbf{Q_d}\). The choices for these matrices depend on the reliability of the present data and the present process model.