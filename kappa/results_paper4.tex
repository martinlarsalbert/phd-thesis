\section{Summary of Paper \ref{pap:physics}}
\subsection*{"\nameref{pap:physics}"}
\subsection*{Scope and Motivations}
It was shown in Paper \ref{pap:pit} that it is possible to identify a model from calm water free running model test with inverse dynamics regression (\autoref{sec:IDR}) together with a cross validation technique (\autoref{sec:cross_validation}) to unsure good generalization (\autoref{sec:generalization}) so that the model can predict other kinds of maneuvers with very good accuracy. It was however soon discovered that these models did not generalize well when wind forces were added to the simulations. This problem was addressed in Paper \ref{pap:physics}.

A brief description of the workflow of Paper \ref{pap:physics} is shown in \autoref{fig:methodology}.
The PI and PU models are identified on free running model tests via inverse dynamics and regression. To assess the physical correctness, a reference model is established, where the PI model is instead identified on a VCT dataset. This reference model, based on CFD, is assumed to be a sufficiently correct representation of the ship's physics.
Verification and comparisons between the models are carried out on the free sailing model tests.
\begin{figure}[h]
  \centering
  %\includesvg[width=\columnwidth, pretex=\scriptsize, height=12cm]{figures/methodology2.svg}
  \includesvg[pretex=\centering\fontsize{7.5}{8}]{kappa/images/methodology2.svg}
  \caption{Research workflow, describing how the reference model is identified with regression of VCT data and the PI and PU models are identified with regression of inverse dynamics forces from model tests. Results are then gathered to assess the parameter drift, physical correctness and generalization of the models.}
  \label{fig:methodology}
\end{figure}
It was investigated if the introduction of a deterministic semi-empirical rudder model in the PI model would reduce the multicollinearity and enhance the generalization.
\subsection*{Results and Conclusions}

\subsection*{Comments}
\clearpage