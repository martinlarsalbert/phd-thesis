%%%%%%%%%%%%%%%%%%%%%%%%%%%%%%%
%%%%%%%%%%%%%%%%%%%%%%%%%%%%%%%
\chapter{Discussion\label{ch:discussion}}
%%%%%%%%%%%%%%%%%%%%%%%%%%%%%%%
This thesis has demonstrated that identifying parametric manoeuvring models from standard manoeuvres is challenged by high multicollinearity among many model parameters. This issue is well-documented; \textcite{yoonIdentificationHydrodynamicCoefficients2003} highlighted the difficulties in separately determining regression coefficients, and \textcite{wangQuantifyingMulticollinearityShip2018} discussed how multicollinearity can lead to parameter drift, resulting in unphysical models, as shown in this thesis.

\autoref{fig:handle_multicollinearity} presents a proposed flowchart for mitigating multicollinearity, addressing the challenges of correctly separating hull and rudder forces, as well as drift and yaw rate-dependent forces.
It was shown in Paper \ref{pap:physics} that a more accurate separation between hull and rudder forces can be achieved by introducing a deterministic semi-empirical rudder model, resulting in a more physically accurate model. Another effective approach is to measure the rudder forces, which yielded excellent results for the Optiwise test case in Paper \ref{pap:vct}.
\begin{figure}[h]
    
    \centering
        \begin{tikzpicture}[node distance=1.5cm]
    
    \node (multicollinearity) [problem] {\footnotesize multicollinearity};
    \node (hull_vs_rudder) [item, below left of=multicollinearity, xshift=-1.5cm] {\footnotesize hull vs. rudder};
    \node (beta_vs_r) [item, below right of=multicollinearity, xshift=1.0cm] {\footnotesize $\beta$ vs. $r$};
    
    \node (semi_empirical) [solution, below left of=hull_vs_rudder, xshift=-1.75cm] {\footnotesize semi-empirical};
    \node (rudder_measure) [solution, below left of=hull_vs_rudder, xshift=1cm] {\footnotesize measure};
    \node (truncate) [solution, below left of=beta_vs_r, xshift=-1.5cm] {\footnotesize truncate};

    \node (generalization) [problem, below left of=truncate, xshift=-0.25cm] {\footnotesize generalization};

    \node (more_data) [wish, below left of=beta_vs_r, xshift=1cm] {\footnotesize more data};
    \node (future) [future, below right of=beta_vs_r, xshift=1.5cm] {\footnotesize Bayesian?};
        
    \node (informative_manoeuvres) [solution, below left of=more_data, xshift=0.25cm, yshift=-0.25cm, align=left] {\footnotesize informative \\ \footnotesize manoeuvres};
    \node (VCT) [solution, below right of=more_data, xshift=0.50cm] {\footnotesize VCT};
    
    
    %Connect
    \draw [arrow] (multicollinearity) -- (hull_vs_rudder);
    \draw [arrow] (multicollinearity) -- (beta_vs_r);

    \draw [arrow] (hull_vs_rudder) -- (semi_empirical);
    \draw [arrow] (hull_vs_rudder) -- (rudder_measure);
    
    \draw [arrow] (beta_vs_r) -- (truncate);
    \draw [arrow] (beta_vs_r) -- (more_data);
    \draw [arrow] (beta_vs_r) -- (future);

    \draw [arrow] (more_data) -- (VCT);
    \draw [arrow] (more_data) -- (informative_manoeuvres);
    
    \draw [arrow] (truncate) -- (generalization);
        
    
    \end{tikzpicture}
    \caption{Flowchart for mitigating multicollinearity based on the research presented in this thesis.}
    \label{fig:handle_multicollinearity}
\end{figure}

Mitigating multicollinearity between drift and yaw rate-dependent parameters during standard manoeuvres is more challenging. \textcite{abkowitzMEASUREMENTHYDRODYNAMICCHARACTERISTICS1980}, \textcite{luoParameterIdentificationShip2016}, \textcite{xuUncertaintyAnalysisHydrodynamic2019}, \textcite{liuPhysicsinformedIdentificationMarine2024}, and Paper \ref{pap:pit} addressed this by truncating polynomials using various methods to select which parameters to remove. This approach is valid if the truncation method correctly identifies which parameters are identifiable from the data. However, model generalization inevitably suffers when parameters are removed, as certain regression terms may only become significant under specific conditions, such as sailing in wind. Truncated models may perform well when simulating conditions similar to the data, such as other standard manoeuvres, but excessive extrapolation should be avoided, as shown in Paper \ref{pap:physics}.

More informative data are needed to mitigate multicollinearity without reducing model generalization. \textcite{yoonIdentificationHydrodynamicCoefficients2003} emphasized the importance of designing experiments that ensure persistence of excitation. They suggested using specific input scenarios that maximize data information content, such as D-optimal designs. \textcite{wangOptimalDesignExcitation2020} and \textcite{millerShipModelIdentification2021} proposed using a pseudo-random sequence (PRS) for this purpose. These manoeuvres require more space than a model test basin provides, necessitating full-scale ship tests at sea or radio-controlled models on a lake.

Another option for gathering more informative data is conducting VCT calculations. Paper \ref{pap:vct} demonstrated that a physically accurate model can be identified in this way, effectively overcoming the multicollinearity problem while maintaining good model generalization.

If a VCT is infeasible, prior knowledge of drift and yaw rate forces can be incorporated, akin to the semi-empirical rudder model. \textcite{chillcceDatadrivenSystemIdentification2023} used a constrained least-squares algorithm, defining the sign and boundaries of the hydrodynamic derivatives based on prior knowledge from VCT. \textcite{taimuri6DoFManeuveringModel2020} proposed calculating hydrodynamic derivatives from semi-empirical formulas combined with corrections from a reference ship. Bayesian modeling offers a more refined approach to expressing prior knowledge as prior probability densities. \textcite{xueHydrodynamicParameterIdentification2020} used a Bayesian approach for parameter identification in manoeuvring models, employing an optimizer to suggest priors. A more accurate, though demanding, method would be to specify priors based on the estimation of hydrodynamic derivatives from many ships, which could be an interesting topic for future work, as discussed in further detail in \autoref{ch:future_work}.
