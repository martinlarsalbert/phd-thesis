\section{Assumptions and limitations}
% Assumptions --> Future work
%Calm waters...
In this thesis, the following assumptions are made.
\begin{enumerate}[label=(\Roman*),itemsep=1mm]
    \item The rigid body assumption simplifies the ship to a stiff body that does not transform under the influence of forces.

    \item Calm water with no external waves is assumed within manoeuvring. The models in this thesis can therefore be described by state space models with Markov property, which means that the fluid memory effects have been neglected \cite{fossenHandbookMarineCraft2011}. 

    \item The maneuvers are assumed to have low-frequency motions so that the added masses can be simplified to have constant values \cite{fossenHandbookMarineCraft2011}.
    
    \item In this paper, the accuracy of the CFD methods used to calculate the VCT data is not investigated. The VCT data is instead assumed to be correct, so that the thesis can instead focus on the system identification given this assumption. Free surface effects were neglected in the VCT calculations assuming that the wave generation is small or has little influence on the manoeuvring performance.

    \item Three degrees of freedom are assumed sufficient to describe the manoeuvring dynamics, neglecting influence of roll, heave, and pitch. 

    \item Only data from standard test types such as turning circles or zigzag tests are used in this thesis, since they are commonly available for ships. 
        
\end{enumerate}



%\section{Reproducibility}
%The research for this thesis has been conducted with the aim of having a high degree of reproducibility. The U.S. National Science Foundation (NSF) subcommittee on replicability in science defines reproducibility as \entryneedsurl{bollen_social_2015}\cite{bollenSocialBehavioralEconomic2015}:
%\begin{quote}
%\vspace{0.2cm}
%\say{Reproducibility refers to the ability of a researcher to duplicate the results of a prior study using the same materials and procedures as were used by the original investigator. ...  Reproducibility is a minimum necessary condition for a finding to be believable and informative.}
%\vspace{0.2cm}
%\end{quote}
%To ensure adequate reproducibility, all code developed in this research has been made available as open source. In addition, the used data has been published as open data, as seen in the references in \autoref{tab:reproducibility}. Publishing the roll decay data from Paper \ref{pap:rolldamping} as open data was not possible due to intellectual property (IP) rights.
%
%\begin{table}[H]
%    \centering
%    \caption{References for reproducibility.}
%    \label{tab:reproducibility}
%    \begin{tabular}{ c l l}
%        \toprule
%         Paper &  Code & Data \\
%         \hline
%         \ref{pap:rolldamping} & \textcite{alexanderssonRolldecayestimators2022} & Unpublished due to IP rights\\
%         \ref{pap:pit} & \textcite{alexanderssonCodePaperSystem2022} & \textcite{alexanderssonWPCCManoeuvringModel2022a}, \textcite{sternExperienceSIMMAN20082011} \\
%         \bottomrule
%    \end{tabular}
%\end{table}

\section{Outline of the thesis}
Chapter \ref{ch:models} presents the models for rigid body ship dynamics used in this thesis. The models for roll motion are introduced in \autoref{sec:roll}, and the manoeuvring motion models are introduced in  \autoref{sec:manoeuvring model}. These models represent the physical principles and thereby the white-box component of the developed grey-box models.
Parameter estimations, representing the black-box component, are used to regress the parameters of the white-box models. The parameter estimations are introduced in \autoref{ch:methods} for the roll motion and the manoeuvring motion in \autoref{sec:_roll} and \autoref{sec:_VMM}. 
A summary of the appended papers, which includes research activities and a selection of the most relevant results, is presented in \autoref{ch:results}, followed by the conclusions in \autoref{ch:conclusions} and plans for future work in \autoref{ch:future_work}.