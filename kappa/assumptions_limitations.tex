\section{Assumptions and limitations}
% Assumptions --> Future work
%Calm waters...
The following assumptions are made in this thesis.
\begin{enumerate}[label=(\Roman*),itemsep=1mm]
    \item The rigid body assumption simplifies the ship to a rigid body that does not deform under the influence of forces.

    \item Calm water with no external waves is assumed within manoeuvring. The models in this thesis can therefore be described by state space models with Markov property, which means that the fluid memory effects have been neglected \cite{fossenHandbookMarineCraft2011}. 

    \item The maneuvers are assumed to have low-frequency motions so that the added masses can be simplified to have constant values \cite{fossenHandbookMarineCraft2011}.

    \item Only data from standard test types such as turning circles or zigzag tests are used in this thesis, since they are commonly available for ships. 

    \item Uncertainties about measurement data from model scale tests and CFD methods have not been studied, which could be a source of error that is not addressed in this thesis.
    
    \item Free surface effects were neglected in the VCT calculations, assuming that the wave generation is small or has little influence on the manoeuvring performance.
    
    \item Three degrees of freedom are assumed sufficient to describe the manoeuvring dynamics, neglecting influence of roll, heave, and pitch. 
        
\end{enumerate}



%\section{Reproducibility}
%The research for this thesis has been conducted with the aim of having a high degree of reproducibility. The U.S. National Science Foundation (NSF) subcommittee on replicability in science defines reproducibility as \entryneedsurl{bollen_social_2015}\cite{bollenSocialBehavioralEconomic2015}:
%\begin{quote}
%\vspace{0.2cm}
%\say{Reproducibility refers to the ability of a researcher to duplicate the results of a prior study using the same materials and procedures as were used by the original investigator. ...  Reproducibility is a minimum necessary condition for a finding to be believable and informative.}
%\vspace{0.2cm}
%\end{quote}
%To ensure adequate reproducibility, all code developed in this research has been made available as open source. In addition, the used data has been published as open data, as seen in the references in \autoref{tab:reproducibility}. Publishing the roll decay data from Paper \ref{pap:rolldamping} as open data was not possible due to intellectual property (IP) rights.
%
%\begin{table}[H]
%    \centering
%    \caption{References for reproducibility.}
%    \label{tab:reproducibility}
%    \begin{tabular}{ c l l}
%        \toprule
%         Paper &  Code & Data \\
%         \hline
%         \ref{pap:rolldamping} & \textcite{alexanderssonRolldecayestimators2022} & Unpublished due to IP rights\\
%         \ref{pap:pit} & \textcite{alexanderssonCodePaperSystem2022} & \textcite{alexanderssonWPCCManoeuvringModel2022a}, \textcite{sternExperienceSIMMAN20082011} \\
%         \bottomrule
%    \end{tabular}
%\end{table}
\clearpage
\section{Outline of the thesis}
Chapter \ref{ch:models} presents the parametric model structures used in this thesis, including the force prediction submodules for the hull, rudder, and propeller. The methods involved in parameter identification are detailed in \autoref{ch:methods}, covering inverse dynamics, added mass estimation, and the proposed method for recursive inverse dynamics regression. The summaries and discussions of the appended papers are then presented in \autoref{ch:results}, followed by conclusions (\autoref{ch:conclusions}) and future work in \autoref{ch:future_work}.