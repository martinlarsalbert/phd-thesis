\section{Summary of Paper \ref{pap:vct}}
\subsection*{"\nameref{pap:vct}"}
\subsection*{Scope and motivations}
The objective of paper \ref{pap:vct} was to propose a parametric model structure based on physical insights from CFD and FRMT inverse dynamics. The reference model from Paper \ref{pap:physics} was assumed to be close to the physically correct true model of ship manoeuvring dynamics. This model was developed with VCT (see \ref{sec:VCT}), which is based on physical first principles through CFD calculations. The paper \ref{pap:vct} investigates the identification of manoeuvring with VCT more closely, with the aim of getting even closer to the true model.

Manoeuvring models were developed for two WAPS test cases with large rudders. The models were identified by conducting VCT to obtain hydrodynamic damping coefficients and by conducting pure yaw and pure sway tests in FNPF to obtain the added masses using the Fourier series method (see \autoref{sec:fourier}). The identified force models were compared with the inverse dynamics forces of the zigzag tests to identify possible weak spots within the models.

\subsection*{Results and concluding remarks}

\begin{figure}[h]
     \centering
     \begin{subfigure}[b]{0.49\textwidth}
         \centering
         \includesvg{figures/results_optiwise_VCT.Y_R_MMG_original.svg}
        \caption{Original MMG rudder model.}
        \label{fig:Y_R_MMG_original}
     \end{subfigure}
     \hfill
     \begin{subfigure}[b]{0.49\textwidth}
         \centering
         \includesvg{figures/results_optiwise_VCT.Y_R_MMG_quadratic.svg}
        \caption{Modified quadratic MMG rudder model.}
        \label{fig:Y_R_MMG_quadratic}
     \end{subfigure}
    \caption{Rudder force during the VCT tests as a function of the effective inflow angle for the original MMG model and the modified quadratic MMG model.}
    \label{fig:MMG_quadratic}
\end{figure}

\clearpage